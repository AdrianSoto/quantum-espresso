\documentclass[12pt,a4paper]{article}
\def\version{5.0}
\def\qe{{\sc Quantum ESPRESSO}}

\usepackage{html}

% BEWARE: don't revert from graphicx for epsfig, because latex2html
% doesn't handle epsfig commands !!!
\usepackage{graphicx}

\textwidth = 17cm
\textheight = 24cm
\topmargin =-1 cm
\oddsidemargin = 0 cm

% \def\htmladdnormallink#1#2{#1}

\begin{document} 
\author{}
\date{}

\def\qeImage{quantum_espresso.pdf}
\def\democritosImage{democritos.pdf}

\begin{htmlonly}
\def\qeImage{quantum_espresso.png}
\def\democritosImage{democritos.png}
\end{htmlonly}

\title{
  \includegraphics[width=5cm]{\qeImage} \hskip 2cm
  \includegraphics[width=6cm]{\democritosImage}\\
  \vskip 1cm
  % title
  \Huge User's Guide for \qe\ \smallskip
  \Large (version \version)
}
%\endhtmlonly

%\latexonly
%\title{
% \epsfig{figure=quantum_espresso.png,width=5cm}\hskip 2cm
% \epsfig{figure=democritos.png,width=6cm}\vskip 1cm
%  % title
%  \Huge User's Guide for \qe \smallskip
%  \Large (version \version)
%}
%\endlatexonly

\maketitle

\tableofcontents

\section{Introduction}

This guide covers the installation and usage of \qe\ (opEn-Source 
Package for Research in Electronic Structure, Simulation,
and Optimization), version \version.

The \qe\ distribution contains the following core packages 
for the calculation of electronic-structure properties within
Density-Functional Theory, using a Plane-Wave basis set and pseudopotentials:
\begin{itemize}
  \item \texttt{PWscf} (Plane-Wave Self-Consistent Field).
  \item \texttt{CP} (Car-Parrinello).
\end{itemize}
It also includes the following more specialized packages:
\begin{itemize}
  \item \texttt{PHonon}:
        phonons with Density-Functional Perturbation Theory.
  \item \texttt{PostProc}: various utilities for data prostprocessing.
  \item \texttt{PWcond}:
        ballistic conductance.
  \item \texttt{GIPAW} 
  (Gauge-Independent Projector Augmented Waves):
        EPR g-tensor and NMR chemical shifts.
  \item \texttt{XSPECTRA}:
        K-edge X-ray adsorption spectra.
  \item \texttt{vdW}:
        (experimental) dynamic polarizability. 
  \item \texttt{Wannier90}:
        maximally localized Wannier functions.
  \item \texttt{GWW}:
        GW calculation using Wannier functions.
  \item \texttt{TD-DFPT}:
        calculations of spectra using Time-Dependent 
        Density-Functional Perturbation Theory.
\end{itemize}
The following auxiliary codes are included as well:
\begin{itemize}
\item \texttt{PWgui} (Graphical User Interface for \texttt{PWscf}): 
      a graphical interface for producing input data files for 
      \texttt{PWscf}.
\item \texttt{atomic}:
      a program for atomic calculations and generation of pseudopotentials.
\end{itemize}
Finally, a copy of required external libraries are included:
\begin{itemize}
\item \texttt{iotk}:
      an Input-Output ToolKit.
\item PMG:
      Multigrid solver for Poisson equation.
\item BLAS and LAPACK
\end{itemize}
This guide documents \texttt{PWscf, CP, PHonon, PostProc, 
PWcond}.
The remaining packages have separate documentation.

The \qe\ codes work on many different types of Unix machines,
including parallel machines using both OpenMP and MPI 
(Message Passing Interface).
Running \qe\ on Mac OS X and MS-Windows is also possible: 
see section \ref{Sec:Installation}.

Further documentation, beyond what is provided in this guide, can be found in:
\begin{itemize}
\item the \texttt{pw\_forum} mailing list (\texttt{pw\_forum@pwscf.org}).
   You can subscribe to this list, browse and search its archives
   (search link in \texttt{http://www.quantum-espresso.org/tools.php}).
   Only subscribed users can post. Please search the archives 
   before posting: your question may have already been answered.
  \item the \texttt{Doc/} directory of the \qe\ distribution,
   containing a detailed description of all input data for all codes
   in the \texttt{INPUT\_*} files (in txt and html format) and a few 
   additional pdf documents; people who want to contribute to
   \qe\ should read the Developer Manual, \texttt{developer\_man.pdf}.
  \item the \qe\ Wiki:\\
   \texttt{http://www.quantum-espresso.org/wiki/index.php/Main\_Page}.
\end{itemize}

This guide does not explain solid state physics and its computational methods.
If you want to learn that, you should read a good textbook, such as e.g.
the book by Richard Martin:
{\em Electronic Structure: Basic Theory and Practical Methods},
Cambridge University Press (2004). See also the bibliography 
section in the Wiki.

This guide assume that you know the basic Unix concepts (shell, 
execution path, directories etc.) and utilities. If you don't, 
you will have a hard time running \qe.

All trademarks mentioned in this guide belong to their respective owners.

\subsection{What can \qe\ do}

\texttt{PWscf} can currently perform the following kinds of calculations:
\begin{itemize}
  \item ground-state energy and one-electron (Kohn-Sham) orbitals;
  \item atomic forces, stresses, and structural optimization;
  \item molecular dynamics on the ground-state Born-Oppenheimer surface,  also with variable cell;
  \item Nudged Elastic Band (NEB) and Fourier String Method Dynamics (SMD)
  for energy barriers and reaction paths;
  \item macroscopic polarization and finite electric fields via 
  the modern theory of polarization (Berry Phases).
\end{itemize}
All of the above works for both insulators and metals, 
in any crystal structure, for many exchange-correlation functionals
(including spin polarization, DFT+U, hybrid functionals), for
norm-conserving (Hamann-Schluter-Chiang) pseudopotentials in 
separable form or Ultrasoft (Vanderbilt) pseudopotentials 
or Projector Augmented Waves (PAW) method.
Non-collinear magnetism and spin-orbit interactions 
are also implemented.  An implementation of finite electric 
fields with a sawtooth potential in a supercell is also available.

\texttt{PHonon} can perform the following types of calculations:
\begin{itemize}
  \item phonon frequencies and eigenvectors at a generic wave vector,
  using Density-Functional Perturbation Theory;
  \item effective charges and dielectric tensors;
  \item electron-phonon interaction coefficients for metals;
  \item interatomic force constants in real space;
  \item third-order anharmonic phonon lifetimes;
  \item Infrared and Raman (nonresonant) cross section.
\end{itemize}
\texttt{PHonon} can be used whenever \texttt{PWscf} can be 
used, with the exceptions of DFT+U and hybrid functionals. 
PAW is not implemented for higher-order response calculations.
Further calculations, in the Quasi-harmonic 
approximations, of the vibrational free energy can be performed
using the \texttt{QHA}  package.

\texttt{PostProc} can perform the following types of calculations:
\begin{itemize}
  \item Scanning Tunneling Microscopy (STM) images;
  \item plots of Electron Localization Functions (ELF);
  \item Density of States (DOS) and Projected DOS (PDOS);
  \item L\"owdin charges;
  \item planar and spherical averages;
\end{itemize}
plus interfacing with a number of graphical utilities and with 
external codes.

\subsection{People}

In the following, the cited affiliation is the one where the last known 
contribution was done and may no longer be valid.

The maintenance and further development of the \qe\ distribution
is promoted by the DEMOCRITOS National Simulation Center 
of IOM-CNR under the coordination of
Paolo Giannozzi (Univ.Udine, Italy) and Layla Martin-Samos 
(Democritos) with the strong support
of the CINECA National Supercomputing Center in Bologna under 
the responsibility of Carlo Cavazzoni.

The \texttt{PWscf} package (which included \texttt{PHonon}
and \texttt{PostProc} in earlier releases)
was originally developed by Stefano Baroni, Stefano
de Gironcoli, Andrea Dal Corso (SISSA), Paolo Giannozzi, and many others.
We quote in particular:
\begin{itemize}
  \item Matteo Cococcioni (MIT) for DFT+U implementation;
  \item David Vanderbilt's group at Rutgers for Berry's phase
  calculations;
  \item Ralph Gebauer (ICTP, Trieste) and Adriano Mosca Conte
  (SISSA, Trieste) for noncolinear magnetism;
  \item Andrea Dal Corso for spin-orbit interactions;
  \item Carlo Sbraccia (Princeton) for NEB, Strings method,
  for improvements to structural optimization
  and to many other parts;
  \item Paolo Umari (Democritos) for finite electric fields;
  \item Renata Wentzcovitch (Univ.Minnesota) for variable-cell
   molecular dynamics;
  \item Lorenzo Paulatto (Univ.Paris VI) for PAW implementation, 
  built  upon previous work by Guido Fratesi (Univ.Milano Bicocca)
  and Riccardo Mazzarello (ETHZ-USI Lugano);
 \item Ismaila Dabo (INRIA, Palaiseau) for electrostatics with
 free boundary conditions.
\end{itemize}
For \texttt{PHonon}, we mention in particular: 
\begin{itemize}
  \item Michele Lazzeri (Paris VI) for the 2n+1 code and Raman 
  cross section calculation with 2nd-order response;
 \item Andrea Dal Corso for Ultrasoft, noncollinear, spin-orbit
 extensions to \texttt{PHonon}.
 \end{itemize}
For \texttt{PostProc}, we mention:
\begin{itemize}
\item Andrea Benassi (SISSA) for the \texttt{epsilon} utility 
(calculation of RPA frequency-dependent complex dielectric
function);
\item Norbert Nemec (U.Cambridge) for the \texttt{pw2casino} 
utility.
\end{itemize}

The \texttt{CP} package is based on the original code written by
 Roberto Car
and Michele Parrinello. CP was developed by Alfredo Pasquarello
(IRRMA, Lausanne), Kari Laasonen (Oulu), Andrea Trave, Roberto
Car (Princeton), Nicola Marzari (MIT), Paolo Giannozzi, and others.
FPMD, later merged with \texttt{CP}, was developed by Carlo
Cavazzoni, 
Gerardo Ballabio (CINECA), Sandro Scandolo (ICTP), 
Guido Chiarotti (SISSA), Paolo Focher, and others.
We quote in particular:
\begin{itemize}
  \item Carlo Sbraccia (Princeton) for NEB;
  \item Manu Sharma (Princeton) and Yudong Wu (Princeton) for
   maximally localized Wannier functions and dynamics with 
   Wannier functions;
  \item Paolo Umari (MIT) for finite electric fields and conjugate
   gradients;
  \item Paolo Umari and Ismaila Dabo for ensemble-DFT;
  \item Xiaofei Wang (Princeton) for META-GGA;
  \item The Autopilot feature was implemented by Targacept, Inc.
\end{itemize}
Other packages in \qe:
\begin{itemize}
\item
\texttt{PWcond} 
was written by Alexander Smogunov (SISSA) and Andrea 
Dal Corso.
\item
\texttt{GIPAW} (\texttt{http://www.gipaw.net})
was written by Davide Ceresoli (MIT), Ari Seitsonen (Univ.Zurich),
Uwe Gerstmann,  Francesco Mauri (Univ. Paris VI).
\item
\texttt{PWgui} was written by Anton Kokalj (IJS Ljubljana) and is 
based on his GUIB concept (\texttt{http://www-k3.ijs.si/kokalj/guib/}).
\item
\texttt{atomic} was written by Andrea Dal Corso and it is the result 
of many additions to the original code by Paolo Giannozzi 
and others. Lorenzo Paulatto wrote the extension to PAW.
\item
\texttt{iotk} (\texttt{http://www.s3.infm.it/iotk}) was written by Giovanni Bussi  (SISSA)  .
\item
Wannier90 (\texttt{http://www.wannier.org/}) was written by A. Mostofi, 
J. Yates, Y.-S Lee (MIT).
\item
\texttt{XSPECTRA} was written by Matteo Calandra (Univ. Paris VI)
and collaborators.
\item \texttt{VdW} was contributed by Huy-Viet Nguyen (SISSA).
\item
\texttt{QHA} was contributed by Eyvaz Isaev (Moscow Steel and Alloy 
Inst. and Linkoping and Uppsala Univ.).
\end{itemize}
Other relevant contributions to \qe:
\begin{itemize}
  \item Andrea Ferretti (MIT) contributed the \texttt{qexml} utility 
  and helped with file formats;
  \item Hannu-Pekka Komsa (CSEA/Lausanne) contributed
  the HSE functional;
  \item Dispersions interaction in the framework of DFT-D were
  contributed by Daniel Forrer (Padua Univ.) and Michele Pavone
  (Naples Univ. Federico II);
 \item Filippo Spiga (Univ. Milano Bicocca) contributed the
  mixed MPI-OpenMP parallelization;
  \item The initial BlueGene porting was done by Costas Bekas and
  Alessandro Curioni (IBM Zurich);
  \item Gerardo Ballabio wrote the first \texttt{configure} for \qe
  \item Audrius Alkauskas (IRRMA), 
Simon Binnie (Univ. College London), Guido Fratesi, Axel Kohlmeyer (UPenn),
Konstantin Kudin (Princeton), Sergey Lisenkov (Univ.Arkansas), 
Nicolas Mounet (MIT), William Parker (Ohio State Univ), 
Guido Roma (CEA), Gabriele Sclauzero (SISSA), Sylvie Stucki (IRRMA), 
Pascal Thibaudeau (CEA), Vittorio Zecca
answered questions on the mailing list, found bugs, helped in 
porting to new architectures, wrote some code.
\end{itemize}

An alphabetical list of further contributors includes: Dario Alf\`e, 
Alain Allouche, Francesco Antoniella, Francesca Baletto,
Mauro Boero, Nicola Bonini, Claudia Bungaro, 
Paolo Cazzato, Gabriele Cipriani, Jiayu Dai, Cesar Da Silva, 
Alberto Debernardi, Gernot Deinzer, Yves Ferro,
Martin Hilgeman,  Yosuke Kanai, Nicolas Lacorne, Stephane Lefranc,
Kurt Maeder, Andrea Marini, 
Pasquale Pavone,  Mickael Profeta, Kurt Stokbro, 
Paul Tangney, 
Antonio Tilocca, Jaro Tobik, 
Malgorzata Wierzbowska, Silviu Zilberman, 
and let us apologize to everybody we have forgotten.
 
This guide was mostly written by Paolo Giannozzi.
Gerardo Ballabio and Carlo Cavazzoni wrote the section on CP.

\subsection{Contacts}

The web site for \qe\ is \texttt{http://www.quantum-espresso.org/}.
Releases and patches can be downloaded from this
site or following the links contained in it. The main entry point for 
developers is the QE-forge web site:
\texttt{http://www.qe-forge.org/}.

The recommended place where to ask questions about installation 
and usage of \qe, and to report bugs, is the \texttt{pw\_forum} 
mailing list: \texttt{pw\_forum@pwscf.org}. Here you can receive
news about \qe\ and obtain help from the developers and from 
knowledgeable users.  
You have to be subscribed in order to post to the list.
Please browse or search the archive -- links are available
in the "Tools" page  of the \qe\ web site,
 \texttt{http://www.quantum-espresso.org/tools.php} -- before 
posting: many questions are asked over and over again.

{\bf Important notice:} only messages that appear to come from the 
registered user's e-mail address, in its {\em exact form}, will be
accepted. Messages "waiting for moderator approval" are
automatically deleted with no further processing (sorry, too 
much spam). In case of trouble, carefully check that your return 
e-mail is the correct one (i.e. the one you used to subscribe).

Since \texttt{pw\_forum} averages $\sim 10$ message a day, an alternative
low-traffic mailing list: \texttt{pw\_users@pwscf.org}, is provided for
those interested only in \qe-related news, such as e.g. announcements 
of new versions, tutorials, etc.. You can subscribe (but not post) to 
this list from the \qe\ web site.

If you need to contact the developers for {\em specific} questions 
about coding, proposals, offers of help, etc., send a message to the
developers' mailing list: \texttt{q-e-developers@qe-forge.org}.
 
\subsection{Terms of use}

\qe\ is free software, released under the 
GNU General Public License. See
\texttt{http://www.gnu.org/licenses/old-licenses/gpl-2.0.txt}, 
or the file License in the distribution).
    
We shall greatly appreciate if scientific work done using this code will 
contain an explicit acknowledgment and the following reference:
\begin{quote}
P. Giannozzi, S. Baroni, N. Bonini, M. Calandra, R. Car, C. Cavazzoni,
D. Ceresoli, G. L. Chiarotti, M. Cococcioni, I. Dabo, A. Dal Corso,
S. Fabris, G. Fratesi, S. de Gironcoli, R. Gebauer, U. Gerstmann,
C. Gougoussis, A. Kokalj, M. Lazzeri, L. Martin-Samos, N. Marzari,
F. Mauri, R. Mazzarello, S. Paolini, A. Pasquarello, L. Paulatto,
C. Sbraccia, S. Scandolo, G. Sclauzero, A. P. Seitsonen, A. Smogunov,
P. Umari, R. M. Wentzcovitch, J.Phys.:Condens.Matter 21, 395502 (2009),
http://arxiv.org/abs/0906.2569
\end{quote}
Note the form \qe\ for textual citations of the code.
Pseudopotentials should be cited as (for instance)

[ ] We used the pseudopotentials C.pbe-rrjkus.UPF
and O.pbe-vbc.UPF from http://www.quantum-espresso.org.

\section{Installing \qe}

\subsection{Download}
 
Presently, \qe\ is only distributed in source form; 
some precompiled executables (binary files) are provided only for PWgui. 
Stable releases of the \qe\ source package (current version 
is \version) can be downloaded from this URL: \\
\texttt{http://www.quantum-espresso.org/download.php}.

Uncompress and unpack the distribution using the command:
\begin{verbatim}
     tar zxvf espresso-5.0.tar.gz
\end{verbatim}
(a hyphen before "zxvf" is optional). If your version of \texttt{tar} 
doesn't recognize the "z" flag:
\begin{verbatim}
     gunzip -c espresso-5.0.tar.gz | tar xvf -
\end{verbatim}
A directory \texttt{espresso-\version/}, containing the distribution, 
will be created. 
Occasionally, patches for the current version, fixing some errors and bugs,
may be distributed as a "diff" file. In order to install a patch (for instance):
\begin{verbatim}
   cd espresso-5.0/
   patch -p1 < /path/to/the/diff/file/patch-file.diff
\end{verbatim}
If more than one patch is present, they should be applied in the correct order.

Daily snapshots of the development version can be downloaded from the
developers' site \texttt{qe-forge.org}: follow the link ''Quantum ESPRESSO'', 
then ''SCM''. Beware: the development version 
is, well, under development: use at your own risk! The bravest 
may access the development version via anonymous CVS 
(Concurrent Version System): see the Developer Manual
(\texttt{Doc/developer\_man.pdf}), section ''Using CVS''.

The \qe\ distribution contains several directories. Some of them are
common to all packages:
\begin{verbatim}
  Modules/    source files for modules that are common to all programs
  include/    files *.h included by fortran and C source files
  clib/       external libraries written in C
  flib/       external libraries written in Fortran
  iotk/       Input/Output Toolkit
  install/    installation utilities
  pseudo/     pseudopotential files used by examples
  upftools/   converters to unified pseudopotential format (UPF)
  examples/   sample input and output files
  tests/      automated tests
  Doc/        documentation
\end{verbatim}
while others are specific to a single package:
\begin{verbatim}
  PW/         PWscf: source files for scf calculations (pw.x)
  pwtools/    PWscf: source files for miscellaneous analysis programs
  PP/         PostProc: source files for post-processing of pw.x data file
  PH/         PHonon: source files for phonon calculations (ph.x) and analysis
  Gamma/      PHonon: source files for Gamma-only phonon calculation (phcg.x)
  D3/         PHonon: source files for third-order derivative calculations (d3.x)
  PWCOND/     PWcond: source files for conductance calculations (pwcond.x)
  vdW/        VdW: source files for calculation of the molecular polarizability
              at finite (imaginary) frequency using approximated Thomas-Fermi
              + von Weizacker scheme
  CPV/        CP: source files for Car-Parrinello code (cp.x)
  atomic/     Source files for the pseudopotential generation package (ld1.x)
  atomic_doc/ Documentation, tests and examples for atomic
  GUI/        PWGui: Graphical User Interface
\end{verbatim}

\subsection{Installation}
\label{Sec:Installation}

To install \qe\ from source, you need first of all a minimal Unix 
environment: basically, a command shell (e.g.,
bash or tcsh) and the utilities \texttt{make, awk, sed}. MS-Windows users need
to have Cygwin (a UNIX environment which runs under Windows) installed:
see \texttt{http://www.cygwin.com/}. Note that the scripts contained in the distribution
assume that the local  language is set to the standard, i.e. "C"; other
 settings 
may break them. Use \texttt{export LC\_ALL=C} (sh/bash) or
\texttt{setenv LC\_ALL C} (csh. tcsh) to prevent any problem 
when running scripts (including installation scripts).

Second, you need C and Fortran-95 compilers. For parallel 
execution, you will also need MPI libraries and a ``parallel''
(i.e. MPI-aware) compiler. For massively parallel machines, or 
for simple multicore parallelization, an OpenMP-aware compiler
and libraries are also required.

Big machines with
specialized hardware (e.g. IBM SP, CRAY, etc) typically have a
Fortran-95 compiler with MPI and OpenMP libraries bundled with 
the software. Workstations or ``commodity'' machines, using PC 
hardware, may or may not have the needed software. If not, you need 
either to buy a commercial product (e.g Portland) or to install
an open-source compiler like gfortran or g95. 
Note that several commercial compilers are available free of charge
under some license for academic or personal usage (e.g. Intel, Sun). 

Instructions for the impatient:
\begin{verbatim}
    cd espresso-\version/
    ./configure
     make all
\end{verbatim}
Executable programs (actually, symlinks to them) will be placed in the
\texttt{bin/}
subdirectory. Note that both C and Fortran compilers must be in your \texttt{PATH} variable.

Additional instructions for CRAY XT, NEC SX, Linux PowerPC machines with
xlf:
\begin{verbatim}
    ./configure ARCH=crayxt4
    ./configure ARCH=necsx
    ./configure ARCH=ppc64-mn
\end{verbatim}
If you have problems or would like to tweak the default settings, read the
detailed instructions below.

\subsection{\texttt{configure}}

To install the \qe\ source package, run the \texttt{configure}
script. This is actually a wrapper to the true \texttt{configure},
located in the \texttt{install/} subdirectory. \texttt{configure}
will (try to) detect compilers and libraries available on
your machine, and set up things accordingly. Presently it is expected
to work on most Linux 32- and 64-bit PCs (all Intel and AMD CPUs) and PC clusters, SGI Altix, IBM SP  machines, NEC SX, Cray XT
machines, Mac OS X, MS-Windows PCs. It may work with
some assistance also on other architectures (see below). 
    
For cross-compilation, you have to specify the target machine with the
--host option (see below). This feature has not been extensively
tested, but we had at least one successful report (compilation for NEC
SX6 on a PC). 
    
Specifically, \texttt{configure} generates the following files:
\begin{verbatim}
    install/make.sys:      compilation rules and flags
    install/configure.msg: a report of the configuration run
    include/fft_defs.h:    defines fortran variable for C pointer
    include/c_defs.h:      defines C to fortran calling convention
                           and a few more things used by C files
\end{verbatim}
\texttt{configure.msg} is only used by \texttt{configure} to print its final 
report and is not needed for compilation. NOTA BENE: unlike previous
versions, \texttt{configure} no longer runs the \texttt{makedeps.sh} 
shell script that updates dependencies. If you modify the  
sources, run \texttt{./install/makedeps.sh} or type \texttt{make depend}
to update files \texttt{make.depend} in the various subdirectories.
    
You should always be able to compile the \qe\ suite
of programs without having to edit any of the generated files. However you
may have to tune \texttt{configure} by specifying appropriate environment variables
and/or command-line options. Usually the tricky part is to get external
libraries recognized and used: see Sec.\ref{Sec:Libraries}
for details and hints.

Environment variables may be set in any of these ways:
\begin{verbatim}
     export VARIABLE=value               # sh, bash, ksh
     ./configure
     setenv VARIABLE value               # csh, tcsh
     ./configure
     ./configure VARIABLE=value          # any shell
\end{verbatim}
Some environment variables that are relevant to \texttt{configure} are:
\begin{verbatim}
     ARCH:         label identifying the machine type (see below)
     F90, F77, CC: names of Fortran 95, Fortran 77, and C compilers
     MPIF90:       name of parallel Fortran 95 compiler (using MPI)
     CPP:          source file preprocessor (defaults to $CC -E)
     LD:           linker (defaults to $MPIF90)
     CFLAGS, FFLAGS, F90FLAGS, CPPFLAGS, LDFLAGS: compilation flags
     LIBDIRS:      extra directories to search for libraries (see below)
\end{verbatim}
For example, the following command line:
\begin{verbatim}
     ./configure MPIF90=mpf90 FFLAGS="-O2 -assume byterecl" \
                  CC=gcc CFLAGS=-O3 LDFLAGS=-static
\end{verbatim}
instructs \texttt{configure} to use \texttt{mpf90} as Fortran 95 compiler 
with flags \texttt{-O2 -assume byterecl}, \texttt{gcc} as C compiler with 
flags \texttt{-O3}, and to link with flag \texttt{-static}. 
Note that the value of \texttt{FFLAGS} must be quoted, because it contains
spaces. NOTA BENE: do not pass compiler names with the leading path
included. \texttt{F90=f90xyz} is ok, \texttt{F90=/path/to/f90xyz} is not. 
Do not use
environmental variables with \texttt{configure} unless they are needed! try
\texttt{configure} with no options as a first step.

If your machine type is unknown to configure, you may use the 
\texttt{ARCH}
variable to suggest an architecture among supported ones. Some large
parallel machines using a front-end (e.g. Cray XT) may need to define
the correct \texttt{ARCH}  even if they are apparently recognized, because
\texttt{configure} cannot figure out that cross-compilation is needed.
Try the one that
looks more similar to your machine type; you'll probably have to do some
additional tweaking. Currently supported architectures are:
\begin{verbatim}
      ia32:    Intel 32-bit machines (x86) running Linux
      ia64:    Intel 64-bit (Itanium) running Linux
      x86_64:  Intel and AMD 64-bit running Linux - see note below
      aix:     IBM AIX machines
      solaris: PC's running SUN-Solaris
      sparc:   Sun SPARC machines
      crayxt4: Cray XT4/5 machines
      macppc:  Apple PowerPC machines running Mac OS X
      mac686:  Apple Intel machines running Mac OS X
      cygwin:  MS-Windows PCs with Cygwin
      necsx:   NEC SX-6 and SX-8 machines
      ppc64:   Linux PowerPC machines, 64 bits
      ppc64-mn:as above, with IBM xlf compiler
\end{verbatim}
{\em Note}: \texttt{x86\_64} replaces \texttt{amd64} since v.4.1. 
Cray Unicos machines, SGI 
machines with MIPS architecture, HP-Compaq Alphas are no longer supported
since v.\version.
Finally, \texttt{configure} recognizes the following command-line options:
\begin{verbatim}
--enable-parallel     compile for parallel execution if possible (default: yes)
--enable-openmp       compile for openmp execution if possible (default: no)
--enable-shared       use shared libraries if available (default: yes)
--disable-wrappers    disable C to fortran wrapper check (default: enabled)
--enable-signals      enable signal trapping (default: disabled)
\end{verbatim}
and the following optional packages:
\begin{verbatim}
  --with-internal-blas    compile with internal blas (default: no)
  --with-internal-lapack  compile with internal lapack (default: no)
  --with-scalapack        use scalapack if available (default: yes)
\end{verbatim}
If you want to modify the \texttt{configure} script (advanced users only!), 
see the Developer Manual.  You will need GNU Autoconf 
(\texttt{http://www.gnu.org/software/autoconf/}) installed.

\subsubsection{Libraries}
\label{Sec:Libraries}

\qe\ makes use of the following external libraries:
\begin{itemize}
\item BLAS (\texttt{http://www.netlib.org/blas/}) and 
\item LAPACK (\texttt{http://www.netlib.org/lapack/}) for linear algebra 
\item FFTW (\texttt{http://www.fftw.org/}) for Fast Fourier Transforms
\end{itemize}
A copy of the needed routines is provided with the distribution. However,
when available, optimized vendor-specific libraries can be used instead: this
often yields huge performance gains.

\paragraph{BLAS and LAPACK} 
\qe\ can use the following architecture-
specific replacements for BLAS and LAPACK:
\begin{verbatim}
      MKL for Intel Linux PCs
      ACML for AMD Linux PCs
      ESSL for IBM machines
      SCSL for SGI Altix
      SUNperf for Sun
\end{verbatim}
If none of these is available, we suggest that you use the optimized ATLAS library: see \\
\texttt{http://math-atlas.sourceforge.net/}. Note that ATLAS is not
a complete replacement for LAPACK: it contains all of the BLAS, plus the
LU code, plus the full storage Cholesky code. Follow the instructions in the
ATLAS distributions to produce a full LAPACK replacement.
    
Sergei Lisenkov reported success and good performances with optimized
BLAS by Kazushige Goto. They can be freely downloaded,
but not redistributed. See the "GotoBLAS2" item at\\
\texttt{http://www.tacc.utexas.edu/tacc-projects/}.

\paragraph{FFT}
\qe\ can use the following vendor-specific FFT libraries:
\begin{verbatim}
      IBM ESSL
      SGI SCSL
      SUN sunperf
      NEC ASL
      AMD ACML
\end{verbatim}
If none of the above is available, you should use FFTW, choosing
before compilation whether to load the built-in copy of FFTW 
or an external v.3 FFTW library.
\texttt{configure} will first search for vendor-specific FFT libraries;
if none is found, it will search for an external FFTW v.3 library;
if none is found, it will fall back to the internal  copy of FFTW.
Appropriate precompiling options will be set in all cases:
\begin{verbatim}
      __FFTW      internal FFTW
      __FFTW3     external FFTW v.3
      __SCSL      SGI SCSL
      __SUNPERF   SUN sunperf 
      __ESSL      IBM ESSL
        ASL       NEC ASL
\end{verbatim}
If you have recent versions of MKL installed, you may try the 
FFTW interface provided with MKL. You will have to compile them
(only sources are distributed with the MKL library)
and to modify file \texttt{make.sys} accordingly (MKL must be linked {\em after}
the FFTW-MKL interface)

If everything else fails, you'll have to modify file \texttt{make.sys}
manually: see Sec.\ref{SubSec:manconf}.
\paragraph{MPI libraries} 
MPI libraries are usually needed for parallel execution 
(unless you are happy with OpenMP multicore parallelization).
In well-configured machines, \texttt{configure} should find the appropriate
parallel compiler for you, and this should find the appropriate
libraries. Since often this doesn't 
happen, especially on PC clusters, see Sec.\ref{SubSec:LinuxPCMPI}.

\paragraph{Other libraries}
\qe\ can use the MASS vector math
library from IBM, if available (only on AIX).
    
The \texttt{configure} script attempts to find optimized libraries, but may fail
if they have been installed in non-standard places. You should examine
the final value of \texttt{BLAS\_LIBS, LAPACK\_LIBS, FFT\_LIBS, MPI\_LIBS} (if needed),
\texttt{MASS\_LIBS} (IBM only), either in the output of \texttt{configure} or in the generated
\texttt{make.sys}, to check whether it found all the libraries that you intend to use.
    
If some library was not found, you can specify a list of directories to search
in the environment variable \texttt{LIBDIRS}, 
and rerun \texttt{configure}; directories in the
list must be separated by spaces. For example:
\begin{verbatim}
   ./configure LIBDIRS="/opt/intel/mkl70/lib/32 /usr/lib/math"
\end{verbatim}
If this still fails, you may set some or all of the \texttt{*\_LIBS} variables manually
and retry. For example:
\begin{verbatim}
   ./configure BLAS_LIBS="-L/usr/lib/math -lf77blas -latlas_sse"
\end{verbatim}
Beware that in this case, \texttt{configure} will blindly accept the specified value,
and won't do any extra search. 
    
{\em Please note}: If you change any settings after a previous (successful or
failed) compilation, you must run \texttt{make clean} before recompiling, 
unless you
know exactly which routines are affected by the changed settings and how to
force their recompilation.

\subsubsection{Manual configuration}
\label{SubSec:manconf}
If \texttt{configure} stops before the end, and you don't find a way to fix
it, you have to write working "make.sys", "include/fft\_defs.h" and
"include/c\_defs.h" files. 
For the latter two files, follow the explanations in 
"include/defs.h.README". 

If \texttt{configure} has run till the end, you should need only to
edit \texttt{install/make.sys}. A few templates (each for a different 
machine type)
are provided in the install/ directory: they have names of the
form \texttt{Make.system}, where "system" is a string identifying the 
architecture and compiler.
  
Most likely (and even more so if there isn't an exact match to your 
machine type), you'll have to tweak make.sys by hand. In particular, 
you must
specify the full list of libraries that you intend to link to.
    
{\em NOTA BENE}:
If you modify the program sources, run the
\texttt{install/makedeps.sh}  script  or type \texttt{make depend} 
to update files \texttt{make.depend} in the various 
subdirectories.

\subsection{Compilation}

There are a few adjustable parameters in \texttt{Modules/parameters.f90}. 
The
present values will work for most cases. All other variables are dynamically
allocated: you do not need to recompile your code for a different system.
    
At your option, you may compile the complete \qe\ suite of programs 
(with \texttt{make all}), or only some specific programs.

\texttt{make} with no arguments yields a list of valid compilation targets. 
Here is a list:
\begin{itemize}
\item \texttt{make pw} produces PW/pw.x\\
pw.x calculates electronic structure, structural optimization, molecular dynamics, barriers with NEB.
\item \texttt{make ph} produces the following codes for phonon calculations:
\begin{itemize}
  \item PH/ph.x\\
    ph.x calculates phonon frequencies and displacement patterns,
    dielectric tensors, effective charges (uses data produced by pw.x). 
  \item dynmat.x\\
    applies various kinds of Acoustic Sum Rule (ASR),
    calculates LO-TO splitting at q = 0 in insulators, IR and Raman
    cross sections (if the coefficients have been properly calculated),
    from the dynamical matrix produced by ph.x 
  \item q2r.x\\
    calculates Interatomic Force Constants (IFC) in real space
    from dynamical matrices produced by ph.x on a regular q-grid 
 \item  matdyn.x\\
    produces phonon frequencies at a generic wave vector
    using the IFC file calculated by q2r.x; may also calculate phonon DOS 
  \end{itemize}
\item \texttt{make d3} produces D3/d3.x\\
  d3.x calculates anharmonic phonon lifetimes (third-order derivatives
  of the energy), using data produced by pw.x and ph.x (Ultrasoft
  pseudopotentials not supported). 
\item \texttt{make gamma} produces Gamma/phcg.x\\
  phcg.x is a version of ph.x that calculates phonons at q = 0 using
  conjugate-gradient minimization of the density functional expanded to
  second-order. Only the $\Gamma$ (q = 0) point is used for Brillouin zone
  integration. It is faster and takes less memory than ph.x, but does
  not support Ultrasoft pseudopotentials. 
\item  \texttt{make pp} produces several codes for data postprocessing, in PP/
  (see list below). 
\item \texttt{make tools} produces several utility programs in pwtools/ (see
  list below).  
\item \texttt{make pwcond} produces PWCOND/pwcond.x\\
 for ballistic conductance calculations.
\item \texttt{make pwall} produces all of the above.
\item \texttt{make ld1} produces code atomic/ld1.x\\
for pseudopotential generation (see specific documentation in atomic\_doc/).
\item \texttt{make upf} produces utilities for pseudopotential conversion in
  directory upftools/.
\item \texttt{make cp} produces the Car-Parrinello code CP in CPV/cp.x
  and the postprocessing code CPV/cppp.x. 
\item \texttt{make all} produces all of the above.
\end{itemize}
For the setup of the GUI, refer to the PWgui-X.Y.Z /INSTALL file, where
X.Y.Z stands for the version number of the GUI (should be the same as the
general version number). If you are using the CVS sources, see
the GUI/README file instead.
   
The codes for data postprocessing in PP/ are:
\begin{itemize}
\item pp.x extracts the specified data from files produced by pw.x,
  prepares data for plotting by writing them into formats that can be
  read by several plotting programs. 
\item bands.x extracts and reorders eigenvalues from files produced by
  pw.x for band structure plotting 
\item projwfc.x calculates projections of wavefunction over atomic
  orbitals, performs L\"wdin population analysis and calculates
  projected density of states. These can be summed using auxiliary
  code sumpdos.x. 
\item dipole.x calculates the dipole moment for isolated systems
  (molecules) and the Makov-Payne correction for molecules in
  supercells (beware: meaningful results only if the charge density is
  completely contained into the Wigner-Seitz cell) 
\item plotrho.x produces PostScript 2-d contour plots
\item plotband.x reads the output of bands.x, produces band structure
  PostScript plots
\item average.x calculates planar averages of quantities produced by
  pp.x (potentials, charge, magnetization densities,...) 
\item dos.x calculates electronic Density of States (DOS)
\item pw2wan.x: interface with code WanT for calculation of transport
  properties via Wannier functions: see \\
  http://www.wannier-transport.org/ 
\item pmw.x generates Poor Man's Wannier functions, to be used in
  DFT+U calculations 
\item pw2casino.x: interface with CASINO code for Quantum Monte Carlo
  calculation (http://www.tcm.phy.cam.ac.uk/\~{}mdt26/casino.html).
  See the header of PP/pw2casino.f90 for instructions on how to use it.
\end{itemize}
Note for people interested in Bader's analysis: there is a software
(link here: \texttt{http://theory.cm.utexas.edu/bader/}) that performs
Bader's analysis starting from charge on a regular grid. The required 
"cube" format can be produced by \qe\ using pp.x (info by G. Lapenna
who has successfully used this technique). This code should perform 
decomposition into Voronoi polyhedra as well, in place of obsolete
code voronoy.x (removed from distribution after v.4.1).

The utility programs in pwtools/ are:
\begin{itemize}
\item lambda.x calculates the electron-phonon coefficient $\lambda$ and the
  function $\alpha^2F(\omega)$ 
\item dist.x calculates distances and angles between atoms in a cell,
  taking into account periodicity 
\item ev.x fits energy-vs-volume data to an equation of state
\item kpoints.x produces lists of k-points
\item pwi2xsf.sh, pwo2xsf.sh process respectively input and output
  files (not data files!) for pw.x and produce an XSF-formatted file
  suitable for plotting with XCrySDen, a powerful crystalline and
  molecular structure visualization program
  (http://www.xcrysden.org/). BEWARE: the pwi2xsf.sh shell script
  requires the pwi2xsf.x executables to be located somewhere in your
  \$PATH. 
\item band\_plot.x: undocumented and possibly obsolete 
\item bs.awk, mv.awk are scripts that process the output of pw.x (not
data files!). Usage: 
\begin{verbatim}
         awk -f bs.awk < my-pw-file > myfile.bs
         awk -f mv.awk < my-pw-file > myfile.mv
\end{verbatim}
The files so produced are suitable for use with xbs, a very simple
X-windows utility to display molecules, available at:\\
http://www.ccl.net/cca/software/X-WINDOW/xbsa/README.shtml 
\item path\_int.sh/path\_int.x: utility to generate, starting from a
  path (a set of images), a new one with a different number of
  images. The initial and final points of the new path can differ from
  those in the original one. Useful for NEB calculations. 
\item kvecs\_FS.x, bands\_FS.x: utilities for Fermi Surface plotting
  using XCrySDen
\end{itemize}

\paragraph{Other utilities}
VdW/ contains the sources for the calculation of the finite (imaginary)
frequency molecular polarizability using the approximated Thomas-Fermi
+ von Weiz\"acker scheme, contributed by H.-V. Nguyen (Sissa and
Hanoi University). Compile with \texttt{make vdw}, executables in VdW/vdw.x, no
documentation yet, but an example in examples/example34.  

\subsection{Running examples}

As a final check that compilation was successful, you may want to run some or
all of the examples contained within the examples directory of the 
\qe\ distribution. Those examples try to exercise all the programs
and features of the \qe\ distribution. A list of examples and
of what each example does is contained in examples/README. For details,
see the README file in each example's directory. If you find that any relevant
feature isn't being tested, please contact us (or even better, write and send
us a new example yourself !).
     
To run the examples, you should follow this procedure:

    
1. Go to the examples directory and edit the environment variables
   file, setting the following variables as needed: 
\begin{verbatim}
   BIN_DIR= directory where executables reside
   PSEUDO_DIR= directory where pseudopotential files reside
   TMP_DIR= directory to be used as temporary storage area
\end{verbatim}
If you have downloaded the full \qe\ distribution, you may set 
BIN\_DIR=\$TOPDIR/bin and PSEUDO\_DIR=\$TOPDIR/pseudo, where =\$TOPDIR
is the root of the \qe\ source tree. In order to be able
to run all the examples, the PSEUDO\_DIR directory must contain all the
needed pseudopotentials. 
If any of these are missing, you can download them (and many others)
from the Pseudopotentials Page of the \qe\ web site
(http://www.quantum-espresso.org/pseudo.php). TMP\_DIR must be a
directory you 
have read and write access to, with enough available space to host the
temporary files produced by the example runs, and possibly offering
high I/O performance (i.e., don't use an NFS-mounted directory). 

2. If you have compiled the parallel version of \qe\ (this
is the default if parallel libraries are detected), you will usually
have to specify a driver program (such as poe or mpiexec) and the
number of processors: see Sec.\ref{SubSec:para} for
details. In order to do that, edit again the environment variables file
and set the PARA\_PREFIX and PARA\_POSTFIX variables as needed. Parallel
executables will be run by a command like this: 
\begin{verbatim}
      $PARA_PREFIX pw.x $PARA_POSTFIX < file.in > file.out
\end{verbatim}
For example, if the command line is like this (as for an IBM SP):
\begin{verbatim}
      poe pw.x -procs 4 < file.in > file.out
\end{verbatim}
you should set PARA\_PREFIX="poe", PARA\_POSTFIX="-procs
4". Furthermore, if your machine does not support interactive use, you
must run the commands specified below through the batch queueing
system installed on that machine. Ask your system administrator for
instructions. 

3. To run a single example, go to the corresponding directory (for
   instance, example/example01) and execute: 
\begin{verbatim}
      ./run_example
\end{verbatim}
This will create a subdirectory results, containing the input and
output files generated by the calculation. Some examples take only a
few seconds to run, while others may require several minutes depending
on your system. To run all the examples in one go, execute:
\begin{verbatim}
      ./run_all_examples
\end{verbatim}
from the examples directory. On a single-processor machine, this
typically takes a few hours. The make\_clean script cleans the
examples tree, by removing all the results subdirectories. However, if
additional subdirectories have been created, they aren't deleted. 

4. In each example's directory, the reference subdirectory contains
verified output files, that you can check your results against. They
were generated on a Linux PC using the Intel compiler. On different
architectures the precise numbers could be slightly different, in
particular if different FFT dimensions are automatically selected. For
this reason, a plain diff of your results against the reference data
doesn't work, or at least, it requires human inspection of the
results. Instead, you can run the  
check\_example script in the examples directory:
\begin{verbatim}
      ./check_example example_dir
\end{verbatim}
where example dir is the directory of the example that you want to
check (e.g., ./check\_example example01). You can specify multiple
directories.  

{\em Note}: check\_example does only a fair job, and only for a few examples. 
For pw.x only, a much better series of automated tests is available in
directory tests/. Edit variables PARA\_PREFIX, PARA\_POSTFIX (if
needed) in file "check\_pw.x.j"; explanations on how to run it 
are in the header of the file.

\subsection{Installation tricks and problems}

\subsubsection{All architectures}

Working Fortran-95 and C compilers are needed in order
to compile \qe. Most ``Fortran-90'' compilers actually
implement the Fortran-95 standard, but older versions 
may not be Fortran-95 compliant. Moreover, 
C and Fortran compilers must be in your PATH.
If configure says that you have no working compiler, you have 
no working compiler.
    
If you get ``Compiler Internal Error'' or similar messages: your
compiler version is buggy. Try to lower the optimization level, or to
remove optimization just for the routine that has problems. If it
doesn't work, or if you experience weird problems at run time, try to 
install patches for your version of the compiler (most vendors release
at least a few patches for free), or to upgrade to a more recent
compiler version.

If you get error messages at the loading phase that look like 
``file XYZ.o: unknown (unrecognized, invalid, wrong, missing, ... ) file
type'', or  ``file format not recognized for file XYZ.a'', one of
the following things have happened:
\begin{enumerate}
\item you have leftover object files from a compilation with another
  compiler: run "make clean" and recompile. 
\item make does not stop at the first compilation error (it may happen
  in some software configurations). Remove file XYZ.o and look for the
  compilation  error. 
\end{enumerate}
If many symbols are missing in the loading phase: you did not specify the
location of all needed libraries (LAPACK, BLAS, FFTW, machine-specific
optimized libraries), in the needed order. 
If only symbols from clib/ are missing, verify that
you have the correct C-to-Fortran bindings, defined in include/c\_defs.h.
Note that \qe\ is self-contained (with the exception of MPI libraries for 
parallel compilation): if system libraries are missing, the problem is in
your compiler/library combination or in their usage, not in \qe.

If you get mysterious errors in the provided tests and examples:
your compiler, or your mathematical libraries, or MPI libraries,
or a combination thereof, is very likely buggy. Although the 
presence of subtle bugs in \qe\ that are not revealed during 
the testing phase can never be ruled out, it is very unlikely
that this happens on the previded tests and examples. 

\subsubsection{Cray XT machines}

Use \texttt{./configure ARCH=crayxt4} or else \texttt{configure} will
not recognize the Cray-specific software environment. Older Cray 
machines: T3D, T3E, X1, are no longer supported.

\subsubsection{IBM AIX}
On IBM machines with ESSL libraries installed, there is a 
potential conflict between a few LAPACK routines that are also part of ESSL, 
but with a different calling sequence. The appearence of run-time errors like
\begin{verbatim}
    ON ENTRY TO ZHPEV  PARAMETER NUMBER  1 HAD AN ILLEGAL VALUE
\end{verbatim}
is a signal that you are calling the bad routine. If you have defined 
\texttt{-D\_\_ESSL} you should load ESSL before LAPACK: see
variable LAPACK\_LIBS in make.sys.

\subsubsection{Linux PC}

Both AMD and Intel CPUs, 32-bit and 64-bit, are supported and work,
either in 32-bit emulation and in 64-bit mode. 64-bit executables 
can address a much larger memory space than 32-bit executable, but
there is no gain in speed.
Beware: the default integer type for 64-bit machine is typically
32-bit long. You should be able to use 64-bit integers as well, 
but it will not give you any advantage and you may run into trouble.

Currently the following compilers are supported by configure:
Intel (ifort), Portland (pgf90), g95, gfortran, Pathscale (pathf95), 
Sun Studio (sunf95),  AMD Open64 (openf95). The ordering approximately
reflects the quality of support. Both Intel MKL and AMD acml mathematical
libraries are suppported. Some combinations of compilers and of libraries
may however require manual editing of make.sys.

It is convenient to create semi-statically linked executables (with only
libc/libm/libpthread linked dynamically). If you want to produce a binary
that runs on different machines, compile it on the oldest machine you have
(i.e. the one with the oldest version of the operating system).

If you get errors like 
\begin{verbatim}
    IPO Error: unresolved : __svml_cos2
\end{verbatim}
at the linking stage, your compiler is optimized to use the SSE
version of sine, cosine etc. contained in the SVML library. Append
'-lsvml' to the list of libraries in your make.sys file (info by Axel
Kohlmeyer, oct.2007). 

\paragraph{Linux PCs with Portland compiler (pgf90)}

\qe\ does not work reliably, or not at all, with many old
versions ($< 6.1$) of the Portland Group compiler 
(http://www.pgroup.com/). Use the latest version of each 
release of the compiler, with patches if available (see
the Portland Group web site).

\paragraph{Linux PCs with Pathscale compiler}

Version 2.99 of the Pathscale compiler(http://www.pathscale.com/)
works and is recognized by
\texttt{configure}, but the preprocessing step:
\begin{verbatim}
   pathcc -E
\end{verbatim}
causes a mysterious error in compilation of iotk and should be replaced by
\begin{verbatim}
   /lib/cpp -P --traditional
\end{verbatim}
The MVAPICH parallel environment with Pathscale compilers also works.
(info by Paolo Giannozzi, July 2008)

\paragraph{Linux PCs with gfortran}

Gfortran v.4.1.2 and later are supported, but only the basic functionalities 
(notably pw.x) have been tested.  Earlier gfortran versions used to produce
nonfunctional phonon executables (segmentation faults and the like). Since
gfortran is evolving quickly, this may have changed meanwhile.

If you experience problems in reading files produced by previous versions
of \qe: "gfortran used 64-bit record markers to allow writing of records 
larger than 2 GB. Before with 32-bit record markers only records $<$2GB 
could be written. However, this caused problems with older files and 
inter-compiler operability. This was solved in GCC 4.2 by using 32-bit 
record markers but such that one can still store $>$2GB records (following 
the implementation of Intel). Thus this issue should be gone. See 4.2 
release notes (item "Fortran") at http://gcc.gnu.org/gcc-4.2/changes.html."
(Info by Tobias Burnus, March 2010).

"Using gfortran v.4.4 (after May 27, 2009) and 4.5 (after May 5, 2009) can 
produce wrong results, unless the environment variable
\texttt{GFORTRAN\_UNBUFFERED\_ALL=1} is set. Newer 4.4/4.5 versions
(later than $\sim$ beginning of April 2010) should be OK. See\\
\texttt{http://gcc.gnu.org/bugzilla/show\_bug.cgi?id=43551}."
(Info by Tobias Burnus, March 2010).

\paragraph{Linux PCs with g95}

G95 v.0.91 and later (http://www.g95.org) works flawlessy. 
The executables it produces are however slower (let us say 20\% or so) 
that those produced by gfortran, which in turn are slower 
(by another 20\% or so) than those produced by ifort.

\paragraph{Linux PCs with Sun Studio compiler}

``The Sun Studio compiler \\
(http://developers.sun.com/sunstudio/) is
free and comes  
with a set of algebra libraries that can be used in place of the slow 
built-in libraries. It also supports openmp, which g95 does not. On the 
other hand, it is a pain to compile mpi with it. Furthermore the most
recent version has a terrible bug that totally miscompiles the iotk 
input/output library (you'll have to compile it with reduced optimization).''
(info by Lorenzo Paulatto, March 2010).

\paragraph{Linux PCs with AMD Open64 suite}

The AMD Open64 compiler suite\\
(http://developer.amd.com/cpu/open64/pages/default.aspx)
can be freely downloaded from the AMD site.
It is recognized by configure but little tested. It sort of works 
but it fails to pass several tests.
(info by Paolo Giannozzi, March 2010).

\paragraph{Linux PCs with Intel compiler (ifort)}

The Intel compiler, ifort\\
 (http://software.intel.com/) is available for 
free for personal usage. It seem to produce the faster executables, 
at least on Intel CPUs, but not all versions work as expected.
ifort v.$<9.1$ are not recommanded, due to the presence of subtle 
and insidious bugs. In case of trouble, update your version with 
the most recent patches,
available via Intel Premier support (registration free of charge for Linux):
http://software.intel.com/en-us/articles/intel-software-developer-support.

If \texttt{configure} doesn't find the compiler, or if you get 
``Error loading shared libraries...'' at run time, you may have 
forgotten to execute the script that
sets up the correct path and library path. Unless your system manager has
done this for you, you should execute the appropriate script -- located in
the directory containing the compiler executable -- in your
initialization files. Consult the documentation provided by Intel. 
    
The warning: ''feupdateenv is not implemented and will always fail'', 
showing up in recent versions, can be safely ignored.
Since each major release of ifort
differs a lot from the previous one. compiled objects from different 
releases may be incompatible and should not be mixed.    

{\bf ifort v.11}: Segmentation faults were reported for the combination 
ifort 11.0.081, MKL 10.1.1.019, openMP 1.3.3. The problem disappeared
with ifort 11.1.056 and MKL 10.2.2.025 (Carlo Nervi, Oct. 2009).

{\bf ifort v.10:} on 64-bit AMD CPUs, at least some versions of ifort 10.1 
miscompile subroutine write\_rho\_xml in Module/xml\_io\_base.f90 with -O2
options. Using -O1 instead solves the problem (info by Carlo
Cavazzoni, March 2008). 

"The intel compiler version 10.1.008 miscompiles a lot of codes (I have proof 
for CP2K and CPMD) and needs to be updated in any case" (info by Axel
Kohlmeter, May 2008).
 
{\bf ifort v.9:} The latest (July 2006) 32-bit version of ifort 9.1
works flawlessy. Earlier versions yielded ``Compiler Internal Error''.

At least some versions of ifort 9.0 have a buggy preprocessor that either
prevents compilation of iotk, or produces runtime errors in cft3. Update
to a more patched version.
    
\paragraph{Linux PCs with MKL libraries}
On Intel CPUs it is very convenient to use Intel MKL libraries. They can be
also used for AMD CPU, selecting the appropriate machine-optimized
libraries, and also together with non-Intel compilers. Note however
that recent versions of MKL (10.2 and following) do not perform
well on AMD machines.

\texttt{configure} should recognize properly installed MKL libraries.
By default the non-threaded version of MKL is linked, unless option
\texttt{configure --with-openmp} is specified. In case of trouble,
refer to the following web page to find the correct way to link MKL:\\
http://software.intel.com/en-us/articles/intel-mkl-link-line-advisor/

MKL contains optimized FFT routines and a FFTW interface, to be separately
compiled. For 64-bit Intel Core2 processors, they are slightly faster than 
FFTW (MKL v.10, FFTW v.3 fortran interface, reported by P. Giannozzi,
November 2008). 

For parallel (MPI) execution on multiprocessor (SMP) machines, set the
environmental variable OMP\_NUM\_THREADS to 1 unless you know what you 
are doing. See the section  on parallelism for more info on this
and on the difference between MPI and OpenMP parallelization. 

\paragraph{Linux PCs with ACML libraries}
For AMD CPUs, especially recent ones, you may find convenient to 
link AMD acml libraries (can be freely downloaded from AMD web site). 
\texttt{configure} should recognize properly installed acml libraries,
together with the compilers most frequently used on AMD systems:
pgf90, pathscale, openf95, sunf95.

\subsubsection{Linux PC clusters with MPI}
\label{SubSec:LinuxPCMPI}
PC clusters running some version of MPI are a very popular
computational platform nowadays. \qe\ is known to work
with at least two of the major MPI implementations (MPICH, LAM-MPI),
plus with the newer MPICH2 and OpenMPI implementation. The number of
possible configurations, in terms of type and version of the MPI
libraries, kernels, system libraries, compilers, is very
large. \qe\ compiles and works on all non-buggy, properly
configured hardware and software combinations. You may have to
recompile MPI libraries: not all MPI installations contain support for
the fortran-90 compiler of your choice (or for any fortran-90 compiler
at all!). Useful step-by-step instructions for MPI comilation can be 
found in the following post by  Javier Antonio Montoya:\\
http://www.democritos.it/pipermail/pw\_forum/2008April/008818.htm . 

If \qe\ does not work for some reason on a PC cluster,
try first if it works in serial execution. A frequent problem with parallel
execution is that \qe\ does not read from standard input,
due to the configuration of MPI libraries: see Sec.\ref{SubSec:para}.

If you are dissatisfied with the performances in parallel execution,
see Sec.\ref{Sec:para} and in particular Sec.\ref{SubSec:badpara}.
See also the following post from Axel Kohlmeyer:\\
http://www.democritos.it/pipermail/pw\_forum/2008-April/008796.html

\subsubsection{Intel Mac OS X}

Newer Mac OS-X machines with Intel CPUs are supported by \texttt{configure},
with gcc4+g95, gfortran, and the Intel compiler ifort with MKL libraries.

\paragraph{Intel Mac OS X with ifort}

"Uninstall darwin ports, fink and developer tools. The presence of all of
those at the same time generates many spooky events in the compilation
procedure.  I installed just the developer tools from apple, the intel
fortran compiler and everything went on great" (Info by Riccardo Sabatini, 
Nov. 2007)

\paragraph{Intel Mac OS X 10.4 and 10.5 with g95 and gfortran}

The stable and unstable versions of g95 are known to work. Recent
gfortran versions also work, but they may require an updated version
of Developer Tools (XCode 2.4.1 or 2.5), that can be downloaded from
Apple. Some tests fails with mysterious errors, that disappear if
fortran BLAS are linked instead of system Atlas libraries. Use: 
\begin{verbatim}
   BLAS_LIBS      = ../flib/blas.a -latlas
\end{verbatim}
(Info by Paolo Giannozzi, jan.2008)

\subsubsection{SGI, Alpha}

SGI Mips machines (e.g. Origin) and HP-Compaq Alpha machines are
no longer supported since v.\version.

\newpage

\section{Parallelism}
\label{Sec:para}

\subsection{Understanding Parallelism in \qe}

Two different parallelization paradigms are currently implemented 
in \qe:
\begin{enumerate}
\item {\em Message-Passing (MPI)}. A copy of the executable runs 
on each CPU; each copy lives in a different world, with its own
private set of data, and communicates with other executables only
via calls to MPI libraries. MPI parallelization requires compilation 
for parallel execution, linking with MPI libraries, execution using 
a launcher program (poe, mpirun, mpiexec, or no launcher at all,
depending upon the specific machine). The number of CPUs used
is specified at run-time either as an option to the launcher or
by the batch queue system. 
\item {\em OpenMP}.  A single executable spawn subprocesses
(threads) that perform in parallel specific tasks. 
OpenMP can be implemented via compiler directives ({\em explicit} 
OpenMP) or via {\em multithreading} libraries  ({\em library} OpenMP).
Explicit OpenMP require compilation for OpenMP execution;
library OpenMP requires only linking to a multithreading
version of mathematical libraries, e.g.:
ESSLSMP, ACML\_MP, MKL (the latter is natively multithreading).
The number of threads is specified at run-time in the environment 
variable OMP\_NUM\_THREADS. 
\end{enumerate}

MPI is the well-established, general-purpose parallelization.
In \qe\ several parallelization levels, specified at run-time
via command-line options to the executable, are implemented
with MPI. This is your first choice for execution a parallel 
machine.

Library OpenMP is a low-effort parallelization suitable for
multicore CPUs. Its effectiveness relies upon the quality of 
the multithreading libraries and the availability of 
multithreading FFTs. If you are using MKL, you may want 
to select FFTW3 and to link with the MKL interface to FFTW3. 
You will get a decent speedup ($\sim 25$\%) on two cores.

Explicit OpenMP is a very recent addition, still at an 
experimental stage, devised to increase scalability on
large multicore parallel machines. Explicit OpenMP is 
devised to be run together with MPI and also together 
with multithreaded libraries. BEWARE: you have to be VERY 
careful to prevent conflicts between the various kinds of
parallelization. If you don't know how to run MPI processes
and OpenMP threads in a controlled manner, forget about mixed 
OpenMP-MPI parallelization.

\subsection{Running on parallel machines}
\label{SubSec:para}

Parallel execution is strongly system- and installation-dependent. 
Typically one has to specify:
\begin{enumerate}
\item a launcher program, such as poe, mpirun, mpiexec, with or
  without  appropriate options 
\item the number of processors, typically as an option to the launcher
  program,  but in some cases to be specified after the program to be
  executed; 
\item the program to be executed, with the proper path if needed: for
  instance, pw.x, or ./pw.x, or \$HOME/bin/pw.x, or whatever applies; 
\item other \qe-specific parallelization options, to be
  read and interpreted by the running code: 
\begin{itemize}
\item the number of ``images'' used by NEB calculations;
\item the number of ``pools'' into which processors are to be grouped
  (pw.x only);
\item the number of ``task groups'' into which processors are to be
  grouped;
\item the number of processors performing iterative diagonalization
  (for pw.x) or orthonormalization (for cp.x).
\end{itemize}
\end{enumerate}
Items 1) and 2) are machine- and installation-dependent, and may be 
different for interactive and batch execution. Note that large
parallel machines are  often  configured so as to disallow interactive
execution: if in doubt, ask your system administrator.
Item 3) also depend on your specific configuration (shell, execution
path, etc). 
Item 4) is optional but may be important: see the following section
for the meaning of the various options.

For illustration, here is how to run pw.x on 16 processors partitioned into
8 pools (2 processors each), for several typical cases. 

IBM SP machines, batch:
\begin{verbatim}
   pw.x -npool 8 < input
\end{verbatim}
This should also work interactively, with environment variables NPROC
set to 16, MP\_HOSTFILE set to the file containing a list of processors.

IBM SP machines, interactive, using poe:
\begin{verbatim}
   poe pw.x -procs 16 -npool 8 < input
\end{verbatim}
PC clusters using mpiexec:
\begin{verbatim}
   mpiexec -n 16 pw.x -npool 8 < input
\end{verbatim}
SGI Altix and PC clusters using mpirun:
\begin{verbatim}   mpirun -np 16 pw.x -npool 8 < input
\end{verbatim}
IBM BlueGene using mpirun:
 \begin{verbatim}
  mpirun -np 16 -exe /path/to/executable/pw.x -args "-npool 8" \
    -in /path/to/input -cwd /path/to/work/directory
\end{verbatim}
If you want to run in parallel the examples distributed with \qe\
(see section ''Run examples''), set PARA\_PREFIX to everything
before the executable (pw.x in the above examples),
PARA\_POSTFIX to what follows it until the first redirection sign 
($<, >, |,..$), if any. For execution using OpenMP, you can
use PARA\_PREFIX="env OMP\_NUM\_THREADS=n"

\subsection{Parallelization levels in \qe}

Data structures are distributed across processors.
Processors are organized in a hierarchy of groups, 
which are identified by different MPI communicators level.
The groups hierarchy is as follow:
\begin{verbatim}
  world _ images _ pools _ task groups
                \_ ortho groups
\end{verbatim}

{\bf world}: is the group of all processors (MPI\_COMM\_WORLD).

{\bf images}: Processors can then be divided into different "images",
corresponding to a point in configuration space (i.e. to
a different set of atomic positions). Such partitioning 
is used when performing Nudged Elastic band (NEB) calculations.

{\bf pools}: When k-point sampling is used, each image group can be 
subpartitioned into "pools", and k-points can distributed to pools.
Within each pool, reciprocal space basis set (plane waves)
and real-space grids are distributed across processors.
This is usually referred to as "plane-wave parallelization".
All linear-algebra operations on array of  plane waves / 
real-space grids are automatically and effectively parallelized.
3D FFT is used to transform electronic wave functions from
reciprocal to real space and vice versa. The 3D FFT is
parallelized by distributing planes of the 3D grid in real
space to processors (in reciprocal space, it is columns of
G-vectors that are distributed to processors). 

{\bf task groups}: 
In order to allow good parallelization of the 3D FFT when 
the number of processors exceeds the number of FFT planes,
data can be redistributed to "task groups" so that each group 
can process several wavefunctions at the same time.

{\bf ortho group}:
A further level of parallelization, independent on
plane-wave (pool) parallelization, is the parallelization of
subspace diagonalization (pw.x) or iterative orthonormalization
(cp.x). Both operations required the diagonalization of 
arrays whose dimension is the number of Kohn-Sham states
(or a small multiple). All such arrays are distributed block-like
across the "ortho group", a subgroup of the pool of processors,
organized in a square 2D grid. The diagonalization is then performed
in parallel using standard linear algebra operations. 
(This diagonalization is used by, but should not be confused with,
the iterative Davidson algorithm).

{\bf Communications}:
Images and pools are loosely coupled and processors communicate
between different images and pools only once in a while, whereas
processors within each pool are tightly coupled and communications
are significant. This means that Gigabit ethernet (typical for
cheap PC clusters) is ok up to 4-8 processors per pool, but {\em fast}
communication hardware (e.g. Mirynet or comparable) is absolutely 
needed beyond 8 processors per pool.

{\bf Choosing parameters}:
To control the number of images, pools and task groups,
command line switch: -nimage -npools -ntg can be used.
The dimension of the ortho group is set to the largest
value compatible with the number of processors and with
the number of electronic states. The user can choose a smaller
value using the command line switch -ndiag (pw.x) or -northo (cp.x) .
As an example consider the following command line:
\begin{verbatim}
mpirun -np 4096 ./pw.x -nimage 8 -npool 2 -ntg 8 -ndiag 144 -input my.input
\end{verbatim}
This execute the \texttt{PWscf} code on 4096 processors, to simulate a system
with 8 images, each of which is distributed across 512 processors.
K-points are distributed across 2 pools of 256 processors each, 
3D FFT is performed using 8 task groups (64 processors each, so
the 3D real-space grid is cut into 64 slices), and the diagonalization
of the subspace Hamiltonian is distributed to a square grid of 144
processors (12x12).

Default values are: -nimage 1 -npool 1 -ntg 1 ; ndiag is chosen
by the code as the fastest $n^2$ (n integer) that fits into the size
of each pool.

\paragraph{Massively parallel calculations}
For very large jobs (i.e. O(1000) atoms or so) or for very long jobs
to be run on massively parallel  machines (e.g. IBM BlueGene) it is
crucial to use in an effective way both the "task group" and the
"ortho group" parallelization. Without a judicious choice of
parameters, large jobs will find a stumbling block in either memory or 
CPU requirements. In particular, the "ortho group" parallelization is
used in the diagonalization  of matrices in the subspace of Kohn-Sham
states (whose dimension is as a strict minumum equal to the number of
occupied states). These are stored as block-distributed matrixes
(distributed across processors) and diagonalized using custom-taylored
diagonalization algorithms that work on block-distributed matrixes.

Since v.4.1, Scalapack can be used to diagonalize block distributed
matrixes, yielding better speed-up than the default algorithms for
large ($ > 1000$) matrices, when using a large number of processors 
($> 512$). If you want to test scalapack  you have to compile adding
-D\_\_SCALAPACK to DFLAGS in make.sys and you have to  
modify the LAPACK\_LIBS variable as in the following example:
\begin{verbatim}
LAPACK_LIBS = -lscalapack -lblacs -lblacsF77init -lblacs -llapack
\end{verbatim}
The repeated -lblacs is not an error, it is needed! after Scalapack,
Blacs, BlacsInit, Blacs again (with their paths if needed), then
Lapack or a suitable replacement.

A further possibility to expand scalability, especially on machines
like IBM BlueGene, is to use mixed MPI-OpenMP. The idea is to have
one (or more) MPI process(es) per multicore node, with OpenMP
parallelization inside a same node. This option is activated by 
preprocessing flag -D\_\_OPENMP, by the following compiler options:
\begin{quote}
 ifort: \texttt{-openmp}\\
 xlf:   \texttt{-qsmp=omp}\\
 PGI:   \texttt{-mp}\\
 ftn:   \texttt{-mp=nonuma}
\end{quote}
and is implemented and tested for the following combinations of FFTs
and libraries:
\begin{quote}
 internal FFTW copy: \texttt{-D\_\_FFTW}\\
 ESSL: \texttt{-D\_\_ESSL} or \texttt{-D\_\_LINUX\_ESSL}, link 
 with \texttt{-lesslsmp}\\
 ACML: \texttt{-D\_\_ACML}, link with \texttt{-lacml\_mp}.
\end{quote}
Currently, ESSL (when available) are faster than internal FFTW,
which in turn are faster than ACML.

\subsubsection{Understanding parallel I/O}
In parallel exeution, each processor has its own slice of wavefunctions, 
to be written to temporary files during the calculation. The way wavefunctions 
are written by pw.x is governed by variable wf\_collect, in namelist control. 
If wf\_collect=.true., the final wavefunctions are collected into a single 
directory, written by a single processor, whose format is independent on 
the number of processors. If wf\_collect=.false. (default) each processor
writes its own slice of the final 
wavefunctions to disk in the internal format used by \texttt{PWscf}. 

The former case requires more
disk I/O and disk space, but produces portable data files; the latter case
requires less I/O and disk space, but the data so produced can be read only
by a job running on the same number of processors and pools, and if
all files are on a file system that is visible to all processors
(i.e., you cannot use local scratch directories: there is presently no
way to ensure that the distribution of processes on processors will
follow the same pattern for different jobs).

cp.x instead always collects the final wavefunctions into a single directory.
Files written by pw.x can be read by cp.x only if wf\_collect=.true. (and if
produced for k=0 case). 

With the new file format (v.3.1 and later) all data (except 
wavefunctions in pw.x if wf\_collect=.false.) is written to and read from
a single directory outdir/prefix.save. A copy of pseudopotential files
is also written there. If some processor cannot access outdir/prefix.save,
it reads the pseudopotential files from the pseudopotential directory
specified in input data. Unpredictable results may follow if those files
are not the same as those in the data directory!

Avoid I/O to network-mounted disks (via NFS) as much as you can! 
Ideally the scratch directory (ESPRESSO\_TMPDIR) should be a modern 
Parallel File System. If you do not have any, you can use local
scratch disks (i.e. each node is physically connected to a disk
and writes to it) but you may run into trouble anyway if you 
need to access your files that are scattered in an unpredictable
way across disks residing on different nodes.

You can use option "disk\_io='minimal'", or even 'none', if you run
into trouble (or angry system managers) with eccessive I/O with pw.x. 
The code will store wavefunctions into RAM during the calculation.
Note however that this will increase your memory usage and may limit 
or prevent restarting from interrupted runs.
\paragraph{Cray XT3}
On the cray xt3 there is a special hack to keep files in
memory instead of writing them without changes to the code.
You have to do a: 
module load iobuf
before compiling and then add liobuf at link time.
If you run a job you set the environment variable 
IOBUF\_PARAMS to proper numbers and you can gain a lot.
Here is one example:
\begin{verbatim}
env IOBUF_PARAMS='*.wfc*:noflush:count=1:size=15M:verbose,\
*.dat:count=2:size=50M:lazyflush:lazyclose:verbose,\
*.UPF*.xml:count=8:size=8M:verbose' pbsyod =\
\~{}/pwscf/pwscfcvs/bin/pw.x npool 4 in si64pw2x2x2.inp > & \
si64pw2x2x232moreiobuf.out &
\end{verbatim}
This will ignore all flushes on the *wfc* (scratch files) using a
single i/o buffer large enough to contain the whole file ($\sim 12$ Mb here).
this way they are actually never(!) written to disk.
The *.dat files are part of the restart, so needed, but you can be
'lazy' since they are writeonly. .xml files have a lot of accesses
(due to iotk), but with a few rather small buffers, this can be
handled as well. You have to pay attention not to make the buffers
too large, if the code needs a lot of memory, too and in this example
there is a lot of room for improvement. After you have tuned those
parameters, you can remove the 'verboses' and enjoy the fast execution.
Apart from the i/o issues the cray xt3 is a really nice and fast machine.
(Info by Axel Kohlmeyer, maybe obsolete)

\subsection{Tricks and problems}

\paragraph{Trouble with input files}
Some implementations of the MPI library have problems with input 
redirection in parallel. This typically shows up under the form of
mysterious errors when reading data. If this happens, use the option 
-in (or -inp or -input), followed by the input file name. 
Example:
\begin{verbatim}
   pw.x -in inputfile npool 4 > outputfile
\end{verbatim} 
Of course the 
input file must be accessible by the processor that must read it
(only one processor reads the input file and subsequently broadcasts
its contents to all other processors).

Apparently the LSF implementation of MPI libraries manages to ignore or to
confuse even the -in/inp/input mechanism that is present in all
\qe\ codes. In this case, use the -i option of mpirun.lsf
to provide an input file.

\paragraph{Trouble with MKL and MPI parallelization}
If you notice very bad parallel performances with MPI and MKL libraries, 
it is very likely that the OpenMP parallelization perfiormed by the latter 
is colliding with MPI. Recent versions of MKL enable autoparallelization
by default on multicore machines.  You must set the environmental variable
OMP\_NUM\_THREADS to 1 to disable it. 
Note that if for some reason the correct setting  of variable
OMP\_NUM\_THREADS  
does not propagate to all processors, you may equally run into trouble. 
Lorenzo Paulatto (Nov. 2008) suggests to use the "-x" option to "mpirun" to 
propagate OMP\_NUM\_THREADS to all processors.
Axel Kohlmeyer suggests the following (April 2008): 
"(I've) found that Intel is now turning on multithreading without any
warning and that is for example why their FFT seems faster than
FFTW. For serial and OpenMP based runs this makes no difference (in
fact the multi-threaded FFT helps), but if you run MPI locally, you
actually lose performance. Also if you use the 'numactl' tool on linux
to bind a job to a specific cpu core, MKL will still try to use all
available cores (and slow down badly). The cleanest way of avoiding
this mess is to either link with
\begin{verbatim}
-lmkl_intel_lp64 -lmkl_sequential -lmkl_core (on 64-bit: x86_64, ia64)
-lmkl_intel -lmkl_sequential -lmkl_core (on 32-bit, i.e. ia32 )
\end{verbatim}
or edit the libmkl\_'platform'.a file (I'm using now a file libmkl10.a with:
\begin{verbatim}
  GROUP (libmkl_intel_lp64.a libmkl_sequential.a libmkl_core.a)
\end{verbatim}
It works like a charm".

\paragraph{Trouble with compilers and MPI libraries}
Many users of \qe, in particular those working on PC clusters,
have to rely on themselves (or on less-than-adequate system managers) for 
the correct configuration of software for parallel execution. Mysterious and
irreproducible crashes in parallel execution are sometimes due to bugs
in \qe, but more often than not are a consequence of buggy
compilers or of buggy or miscompiled MPI libraries. Very useful step-by-step 
instructions to compile and install MPI libraries
can be found in the following post by Javier Antonio Montoya:\\
http://www.democritos.it/pipermail/pw\_forum/2008-April/008818.htm .

On a Xeon quadriprocessor cluster, erratic crashes in parallel
execution have been reported, apparently correlated with ifort 10.1
(info by Nathalie Vast and Jelena Sjakste, May 2008).

\section{Using \texttt{PWscf}}

Input files for the \texttt{PWscf} codes may be either written by hand (the good old
way), or produced via the "PWgui" graphical interface by Anton Kokalj, 
included in the \qe\ distribution. See PWgui-x.y.z/INSTALL
(where x.y.z is the version number) for more info on PWgui, or GUI/README
if you are using CVS sources.
    
You may take the examples distributed with \qe\ as
templates for writing your own input files: see section 2.3, "Run examples".
In the following, whenever we mention "Example N", we refer to those. Input
files are those in the results directories, with names ending in .in 
(they will appear after you have run the examples).
    
Note about exchange-correlation: the type of exchange-correlation used
in the calculation is read from PP files. All PP's must have been generated
using the same exchange-correlation.

\subsection{Electronic and ionic structure calculations}

Electronic and ionic structure calculations are performed by program pw.x.

\paragraph{Input data}

The input data is organized as several namelists, followed by other fields
introduced by keywords.
    
The namelists are
\begin{verbatim}
      &CONTROL: general variables controlling the run
      &SYSTEM: structural information on the system under investigation
      &ELECTRONS: electronic variables: self-consistency, smearing
      &IONS (optional): ionic variables: relaxation, dynamics
      &CELL (optional): variable-cell dynamics
      &EE  (optional): for density counter charge electrostatic corrections
\end{verbatim}    
Optional namelist may be omitted if the calculation to be performed
does not require them. This depends on the value of variable calculation
in namelist \&CONTROL. Most variables in namelists have default values. Only
the following variables in \&SYSTEM must always be specified:
\begin{verbatim}
      ibrav (integer): bravais-lattice index
      celldm (real, dimension 6): crystallographic constants
      nat (integer): number of atoms in the unit cell
      ntyp (integer): number of types of atoms in the unit cell
      ecutwfc (real): kinetic energy cutoff (Ry) for wavefunctions.
\end{verbatim}    
For metallic systems, you have to specify how metallicity is treated
by setting 
variable occupations. If you choose occupations='smearing', you have
to specify the smearing width degauss and optionally the smearing type
smearing. If you choose occupations='tetrahedra', you need to specify 
a suitable uniform k-point grid (card K\_POINTS with option automatic).
Spin-polarized systems must be treated as metallic system, except the 
special case of a single k-point, for which occupation numbers can be fixed
(occupations='from input' and card OCCUPATIONS).
    
Explanations for the meaning of variables ibrav and celldm are in file
INPUT\_PW. Please read them carefully. There is a large number of other
variables, having default values, which may or may not fit your needs.
    
After the namelists, you have several fields introduced by keywords with
self-explanatory names:
\begin{verbatim}
       ATOMIC_SPECIES
       ATOMIC_POSITIONS
       K_POINTS
       CELL_PARAMETERS (optional)
       OCCUPATIONS (optional)
       CLIMBING_IMAGES (optional)
\end{verbatim}
The keywords may be followed on the same line by an option. Unknown
fields (including some that are specific to CP package) are ignored by \texttt{PWscf}.
See file Doc/INPUT\_PW for a detailed explanation of the meaning and format
of the various fields.
    
Note about k points: The k-point grid can be either automatically generated 
or manually provided as a list of k-points and a weight in the Irreducible
Brillouin Zone only of the Bravais lattice of the crystal. The code will 
generate (unless instructed not to do so: see variable nosym) all
required k-point 
and weights if the symmetry of the system is lower than the symmetry of the
Bravais lattice. The automatic generation of k-points follows the convention
of Monkhorst and Pack.

\subsubsection{Typical cases}
We may distinguish the following typical cases for pw.x:

\paragraph{Single-point (fixed-ion) SCF calculation} 
Set calculation='scf'.
Namelists \&IONS and \&CELL need not to be present (this is the default).\\
See Example 01.

\paragraph{Band structure calculation}
First perform a SCF calculation as above;
then do a non-SCF calculation by specifying calculation='bands' or
calculation='nscf', with the desired k-point grid and number nbnd
of bands. If you are interested in calculating only the Kohn-Sham states
for the given set of k-points, use calculation='bands'. If you are
interested in further processing of the results of non-SCF calculations
(for instance, in DOS calculations) use calculations='nscf'.
Specify nosym=.true. to avoid generation of additional k-points in
low symmetry cases. Variables prefix and outdir, which determine
the names of input or output files, should be the same in the two runs.\\
See Example 01.

{\bf Important:} until v.4.0, atomic positions for a non scf calculations 
were read from input, while the scf potential was read from the data file
of the scf calculation. Since v.4.1, both atomic positions and the scf
potential are read from the data file so that consistency is guaranteed.

\paragraph{Structural optimization}
Specify calculation='relax' and add namelist \&IONS.

All options for a single SCF calculation apply, plus a few others. You
may follow a structural optimization with a non-SCF band-structure
calculation (since v.4.1, you do not need any longer to update the 
atomic positions in the input file for non scf calculation).\\
See Example 03.

\paragraph{Molecular Dynamics} 
Specify calculation='md' and time step dt.

Use variable ion dynamics in namelist \&IONS for a fine-grained control
of the kind of dynamics. Other options for setting the initial
temperature and for thermalization using velocity rescaling are
available. Remember: this is MD on the electronic ground state, not
Car-Parrinello MD.\\
See Example 04.

\paragraph{Polarization via Berry Phase}
See Example 10, its README, and the documentation in the header of
PW/bp\_c\_phase.f90. 

\paragraph{Nudged Elastic Band calculation}
Specify calculation='neb' and add namelist \&IONS.

All options for a single SCF calculation apply, plus a few others. In the
namelist \&IONS the number of images used to discretize the elastic band
must be specified. All other variables have a default value. Coordinates
of the initial and final image of the elastic band have to be specified
in the ATOMIC POSITIONS card. A detailed description of all input
variables is contained in the file Doc/INPUT PW. See also Example 17.

An NEB calculation will produce a number of files in the current directory
(i.e. in the directory were the code is run) containing additional information
on the minimum-energy path. The files are organized as following, where \$prefix
is the prefix of the calculation (specified in the input file):
\begin{description}
\item[\$prefix.dat]
is a three-column file containig the position of each image on the reaction
coordinate (arb. units), its energy in eV relative to the energy of the first image
and the residual error for the image in eV/$a_0$.
\item[\$prefix.int]
contains an interpolation of the path energy profile that pass exactly through each
image; it is computed using both the image energies and their derivatives
\item[\$prefix.path]
information used by QE to restart a path calculation, its format depends on the input
details and is undocumented
\item[\$prefix.axsf]
atomic positions of all path images in the XCrySDen animation format,
{\texttt xcrysden -\--axsf \$prefix.axsf} to open it
\item[\$prefix.xyz]
atomic positions of all path images in the generic xyz format, used by
many quantum-chemistry softwares
\item[\$prefix.crd]
path information in the input format used by {\texttt pw.x}, suitable for a manual
restart of the calculation
\end{description}

"NEB calculation are a bit tricky in general and require extreme care to be
setup correctly. NEB also takes easily hunders of iteration to converge,
of course depending on the number of atoms and of images. Here is some
free advice:
\begin{enumerate}
\item 
Don't use Climbing Image (CI) from the beginning. It makes convergence slower, 
especially if the special image changes during the convergence process (this 
may happen if CI\_scheme='auto' and if it does it may mess up everything).
Converge your calculation, then restart from the last configuration with
CI option enabled (note that this will {\em increase} the barrier).
\item
Carefully choose the initial path. Remember that QE assumes continuity
between the first and the last image at the initial condition. In other 
words, periodic images are NOT used; you may have to manually translate
an atom by one or more unit cell base vectors in order to have a meaningful
initial path. You can visualize NEB input files with XCrySDen as animations,
take some time to check if any atoms overlap or get very close in the initial
path (you will have to add intermediate images, in this case).
\item
Try to start the NEB process with most atomic positions fixed, 
in order to converge the more "problematic" ones, before leaving
all atoms move.
\item
Especially for larger systems, you can start NEB with lower accuracy 
(less k-points, lower cutoff) and then increase it when it has
converged to refine your calculation.
\item
Use the Broyden algorithm instead of the default one: it is a bit more
fragile, but it removes the problem of "oscillations" in the calculated
activation energies. If these oscillations persist, and you cannot afford 
more images, focus to a smaller problem, decompose it into pieces.
\item
A gross estimate of the required number of iterations is
(number of images) * (number of atoms) * 3. Atoms that do not
move should not be counted. It may take half that many iterations, 
or twice as many, but more or less that's the order of magnitude, 
unless one starts from a very good or very bad initial guess.
\end{enumerate}
(Courtesy of Lorenzo Paulatto)

\subsubsection{Data files}

The output data files are written in the directory specified by variable
outdir, with names specified by variable prefix (a string that is prepended
to all file names, whose default value is: prefix='pwscf'). The ''iotk''
toolkit is used to write the file in a XML format, whose definition can
be found in the Developer Manual. In order to use the data directory
on a different machine, you need to convert the binary files to formatted
and back, using the ''bin/iotk'' script.

The execution stops if you create a file "prefix.EXIT" in the working 
directory. IMPORTANT NOTE: this is the directory where the program 
is executed, NOT the directory "outdir" defined in input, where files 
are written. Note that with some versions of MPI, the "working directory" 
is the directory where the pw.x executable is! The advantage of this 
procedure is that all files are properly closed, whereas  just killing 
the process may leave data and output files in unusable state.

\subsection{Hartree-Fock and Hybrid functionals}

Calculations in the Hartree-Fock approximation, or using hybrid XC functionals 
that include some Hartree-Fock exchange, can be performed by adding
\texttt{-DEXX} to the preprocessing options \texttt{DFLAGS} in file 
\texttt{make.sys}. Issue command \texttt{make clean} before recompiling. 
Documentation on usage can be found in subdirectory \texttt{EXX\_example/}
of the \texttt{examples/} directory.

The algorithm is quite standard: see for instance Chawla and Voth, 
JCP {bf 108}, 4697 (1998); Sorouri, Foulkes and Hine, JCP {\bf 124}, 
064105 (2006); Spencer and Alavi, PRB {\bf 77}, 193110 (2008). 
Basically, one generates auxiliary densities $\rho_{-q}=\phi^{*}_{k+q}*\psi_k$
in real space and transforms them to reciprocal space using FFT;
the Poisson equation is solved and the resulting potential is transformed 
back to real space using FFT, then multiplied by $\phi_{k+q}$ and the
results are accumulated.
The only tricky point is the treatment of the $q\rightarrow 0$ limit,
which is described in the Appendix A.5 of the \qe\ paper mentioned 
in the Introduction (note the reference to the Gygi and Baldereschi paper). 
See also J. Comp. Chem. {\bf 29}, 2098 (2008);
JACS {\bf 129}, 10402 (2007) for examples of applications.

\section{Phonon calculations}

Phonon calculation is presently a two-step process:
First, you have to find the ground-state atomic and electronic configuration;
Second, you can calculate phonons using Density-Functional Perturbation Theory.
Further processing to calculate Interatomic Force Constants, to add macroscopic
electric field and impose Acoustic Sum Rules at q=0 may be needed.

Since version 4 it is possible to safely stop execution of the phonon code using
the same mechanism of the pw.x code, i.e. by creating a file prefix.EXIT in the 
working directory. Execution can be resumed by setting 'recover=.true.' in the
subsequent input data.

\subsection{Single-q calculation}

The phonon code ph.x calculates normal modes at a given q-vector, starting
from data files produced by pw.x with a simple SCF calculation.
NOTE: the alternative procedure in which a band-structure calculation 
with calculation='phonon' was performed as an intermediate step is no
longer implemented since version 4.1. It is also no longer needed to
specify lnscf=.true. for $q\ne 0$.

The output data file appear in the directory specified by variables outdir,
with names specified by variable prefix. After the output file(s) has been
produced (do not remove any of the files, unless you know which are used
and which are not), you can run ph.x.
    
The first input line of ph.x is a job identifier. At the second line the
namelist \&INPUTPH starts. The meaning of the variables in the namelist
(most of them having a default value) is described in file INPUT PH. Variables
outdir and prefix must be the same as in the input data of pw.x. Presently
you must also specify amass (real, dimension ntyp): the atomic mass of each
atomic type.

After the namelist you must specify the q-vector of the phonon mode.
This must be the same q-vector given in the input of pw.x.
    
Notice that the dynamical matrix calculated by ph.x at q = 0 does not
contain the non-analytic term occuring in polar materials, i.e. there is no
LO-TO splitting in insulators. Moreover no Acoustic Sum Rule (ASR) is
applied. In order to have the complete dynamical matrix at q = 0 including
the non-analytic terms, you need to calculate effective charges by specifying
option epsil=.true. to ph.x.

Use program dynmat.x to calculate the correct LO-TO splitting, IR cross
sections, and to impose various forms of ASR. If ph.x was instructed to 
calculate Raman coefficients, dynmat.x will also calculate Raman cross sections
for a typical experimental setup.
    
A sample phonon calculation is performed in Example 02.

\subsection{Calculation of interatomic force constants in real space}

First, dynamical matrices D(q) are calculated and saved for a suitable uniform 
grid of q-vectors (only those in the Irreducible Brillouin Zone of the
crystal are needed). Although this can be done one q-vector at the time, a
simpler procedure is to specify variable ldisp=.true. and to set variables
nq1,nq2,nq3 to some suitable Monkhorst-Pack grid, that will be automatically
generated, centered at q = 0. Do not forget to specify epsil=.true.
in the input data of ph.x if you want the correct TO-LO splitting in polar
materials.
    
Second, code q2r.x reads the D(q) dynamical matrices produced in the
preceding step and Fourier-transform them, writing a file of Interatomic Force
Constants in real space, up to a distance that depends on the size of the grid
of q-vectors. Program matdyn.x may be used to produce phonon modes and
frequencies at any q using the Interatomic Force Constants file as input.
See Example 06.

\subsection{Calculation of electron-phonon interaction coefficients}

The calculation of electron-phonon coefficients in metals is made difficult by
the slow convergence of the sum at the Fermi energy. It is convenient to 
calculate phonons, for each q-vector of a suitable grid, using a
smaller k-point 
grid, saving the dynamical matrix and the self-consistent first-order variation
of the potential (variable fildvscf). Then a non-SCF calculation with
a larger k-point grid is performed. Finally the electron-phonon calculation is
performed by specifying elph=.true., trans=.false., and the input files
fildvscf, fildyn. The electron-phonon coefficients are calculated using several
values of gaussian broadening (see PH/elphon.f90) because this quickly
shows whether results are converged or not with respect to the k-point grid
and Gaussian broadening. See Example 07.

All of the above must be repeated for all desired q-vectors and the final
result is summed over all q-vectors, using pwtools/lambda.x. The input
data for the latter is described in the header of pwtools/lambda.f90.

\section{Post-processing}

There are a number of auxiliary codes performing postprocessing tasks such
as plotting, averaging, and so on, on the various quantities calculated by
pw.x. Such quantities are saved by pw.x into the output data file(s). 
Postprocessing codes are in the PP/ directory.

\subsection{Plotting selected quantities}
  
The main postprocessing code pp.x reads data file(s), extracts or calculates 
the selected quantity, writes it into a format that is suitable for plotting.

Quantities that can be read or calculated are:
\begin{verbatim}
      charge density
      spin polarization
      various potentials
      local density of states at $E_F$
      local density of electronic entropy
      STM images
      selected squared wavefunction
      ELF (electron localization function)
      planar averages
      integrated local density of states
\end{verbatim}
Various types of plotting (along a line, on a plane, three-dimensional, polar)
and output formats (including the popular cube format) can be specified.
The output files can be directly read by the free plotting system Gnuplot
(1D or 2D plots), or by code plotrho.x that comes with \texttt{PWscf} (2D plots),
or by advanced plotting software XCrySDen and gOpenMol (3D plots).

See file INPUT\_PP.* for a detailed description of the input for code pp.x.
See example05/ in the examples/ directory for a charge density plot.

\subsection{Band structure, Fermi surface}

The code bands.x reads data file(s), extracts eigenvalues,
regroups them into bands (the algorithm used to order bands and to resolve
crossings may not work in all circumstances, though). The output is written
to a file in a simple format that can be directly read by plotting program
plotband.x. Unpredictable plots may results if k-points are not in sequence
along lines. See example05/ in the examples/ directory for a simple band plot.

The code bands.x performs as well a symmetry analysis of the band structure:
see example01/. 

The calculation of Fermi surface can be performed using kvecs\_FS.x and
bands\_FS.x. The resulting file in .xsf format can be read and plotted
using xcrysden. See example08/ for an example of Fermi surface 
visualization (Ni, including the spin-polarized case).

\subsection{Projection over atomic states, DOS}

The code projwfc.x calculates projections of wavefunctions
over atomic orbitals. The atomic wavefunctions are those contained
in the pseudopotential file(s). The L\"owdin population analysis (similar to
Mulliken analysis) is presently implemented. The projected DOS (or PDOS:
the DOS projected onto atomic orbitals) can also be calculated and written
to file(s). More details on the input data are found in file
INPUT\_PROJWFC.*.  
The auxiliary code sumpdos.x (courtesy of Andrea Ferretti) 
can be used to sum selected PDOS, by specifiying the names of files
containing the desired PDOS. Type sumpdos.x -h or look into the source
code for more details. The total electronic DOS is instead calculated by code
PP/dos.x. See example08/ in the examples/ directory for total and projected 
electronic DOS calculations.

\subsection{Tools}

Code sumpdos.x is a small utility that sums the partial DOS of
selected atoms. See the header of  sumpdos.f90 for instructions.

The code path\_int.x is intended to be used in the framework of NEB
calculations. It is a tool to generate a new path (what is actually
generated is the restart file) starting from an old one through
interpolation (cubic splines). The new path can be discretized with a
different number of images (this is its main purpose), images are
equispaced and the interpolation can be also 
performed on a subsection of the old path. The input file needed by
path\_int.x can be easily set up  with the help of the self-explanatory
path\_int.sh shell script. 

\section{Using CP}

This section is intended to explain how to perform basic Car-Parrinello (CP)
simulations using the CP package.
    
It is important to understand that a CP simulation is a sequence of different 
runs, some of them used to "prepare" the initial state of the system, and 
other performed to collect statistics, or to modify the state of the system
itself, i.e. modify the temperature or the pressure.
    
To prepare and run a CP simulation you should:
\begin{enumerate}
\item Define the system:
  \begin{enumerate}
    \item atomic positions
    \item system cell
    \item pseudopotentials
    \item number of electrons and bands
    \item cut-offs
    \item FFT grids
  \end{enumerate}
\item
\end{enumerate}
The first run, when starting from scratch, is always an electronic 
minimization, with fixed ions and cell, to bring the electronic system 
on the ground state (GS) relative to the starting atomic configuration.\\
Example of input file (Benzene Molecule):
\begin{verbatim}
         &control
            title = 'Benzene Molecule',
            calculation = 'cp',
            restart_mode = 'from_scratch',
            ndr = 51,
            ndw = 51,
            nstep = 100,
            iprint = 10,
            isave = 100,
            tstress = .TRUE.,
            tprnfor = .TRUE.,
            dt    = 5.0d0,
            etot_conv_thr = 1.d-9,
            ekin_conv_thr = 1.d-4,
            prefix = 'c6h6',
            pseudo_dir='/scratch/benzene/',
            outdir='/scratch/benzene/Out/'
         /
         &system
            ibrav = 14,
            celldm(1) = 16.0,
            celldm(2) = 1.0,
            celldm(3) = 0.5,
            celldm(4) = 0.0,
            celldm(5) = 0.0,
            celldm(6) = 0.0,
            nat = 12,
            ntyp = 2,
            nbnd = 15,
            ecutwfc = 40.0,
            nr1b= 10, nr2b = 10, nr3b = 10,
            xc_type = 'BLYP'
         /
         &electrons
            emass = 400.d0,
            emass_cutoff = 2.5d0,
            electron_dynamics = 'sd'
         /
         &ions
            ion_dynamics = 'none'
         /
         &cell
            cell_dynamics = 'none',
            press = 0.0d0,
          /
          ATOMIC_SPECIES
          C 12.0d0 c_blyp_gia.pp
          H 1.00d0 h.ps
          ATOMIC_POSITIONS (bohr)
          C     2.6 0.0 0.0
          C     1.3 -1.3 0.0
          C    -1.3 -1.3 0.0
          C    -2.6 0.0 0.0
          C    -1.3 1.3 0.0
          C     1.3 1.3 0.0
          H     4.4 0.0 0.0
          H     2.2 -2.2 0.0
          H    -2.2 -2.2 0.0
          H    -4.4 0.0 0.0
          H    -2.2 2.2 0.0
          H     2.2 2.2 0.0
\end{verbatim} 
You can find the description of the input variables in file INPUT\_CP 
in the Doc/ directory. 

\subsection{Reaching the electronic ground state}

The first step in a CP scheme is to reach the electronic 
ground state (GS), for a given set of nuclear positions.
Sometimes a single run is not enough to reach the GS. In this case,
you need to re-run the electronic minimization stage. Use the input 
of the first run, changing restart\_mode = 'from\_scratch' to 
restart\_mode = 'restart'.
   
\paragraph{Important} Unless you are already experienced with the system 
you are studying or with the internals of the code, you wil usually need 
to tune some input parameters, like emass, dt, and cut-offs. For this 
purpose, a few trial runs could be useful: you can perform short
minimizations (say, 10 steps) changing and adjusting these parameters 
to fit your needs. You can specify the degree of convergence with these
two thresholds:

etot\_conv\_thr: total energy difference between two consecutive steps

ekin\_conv\_thr: value of the fictitious kinetic energy of the electrons
   
Usually we consider the system on the GS when ekin conv\_thr $ < 10^{-5}$.
You could check the value of the fictitious kinetic energy on the standard 
output (column EKINC).

Different strategies are available to minimize electrons, but the most used 
ones are:
\begin{itemize}
\item steepest descent: electron\_dynamics = 'sd'
\item damped dynamics: electron\_dynamics = 'damp',
  electron\_damping = a number typically ranging from 0.1 and 0.5 
\end{itemize}
See the input description to compute the optimal damping factor.

\subsection{Relax the system}

Once your system is in the GS, depending on how you have prepared the starting
atomic configuration:
\begin{itemize}
\item
if you have set the atomic positions "by hand" and/or from a classical code, 
check the forces on atoms, and if they are large ($\sim 0.1 \div 1.0$
atomic units), you should perform an ionic minimization, otherwise the
system could break up during the dynamics.
\item
if you have taken the positions from a previous run or a previous ab-initio 
simulation, check the forces, and if they are too small ($\sim 10^{-4}$ 
atomic units), this means that atoms are already in equilibrium positions 
and, even if left free, they will not move. Then you need to randomize 
positions a little bit (see below).
\end{itemize}

1.) Minimize ionic positions.

As we pointed out in 4) if the interatomic forces are too high, 
the system could "explode" if we switch on the ionic dynamics. 
To avoid that we need to relax the system. Again there are 
different strategies to relax the system, but the most used 
are again steepest descent or damped dynamics for ions and electrons. 
You could also mix electronic and ionic minimization scheme freely, 
i.e. ions in steepest and electron in damping or vice versa.
    
(a) suppose we want to perform a steepest for ions. Then we should specify 
the following section for ions:
\begin{verbatim} 
         &ions
           ion_dynamics = 'sd'
         /
\end{verbatim} 
Change also the ionic masses to accelerate the minimization:
\begin{verbatim} 
         ATOMIC_SPECIES
          C 2.0d0 c_blyp_gia.pp
          H 2.00d0 h.ps
\end{verbatim} 
while leaving other input parameters unchanged.
{\em Note} that if the forces are really high ($> 1.0$ atomic units), you
should always use steepest descent for the first ($\sim 100$
relaxation steps. 

(b) as the system approaches the equilibrium positions, the steepest 
descent scheme slows down, so is better to switch to damped dynamics:
\begin{verbatim} 
         &ions
           ion_dynamics = 'damp',
           ion_damping = 0.2,
           ion_velocities = 'zero'
         /
\end{verbatim}
A typical value of ion damping between 0.05 is good for many systems. 
It is also better to specify to restart with zero ionic and electronic 
velocities, since we have changed the masses.
    
Change further the ionic masses to accelerate the minimization:
\begin{verbatim} 
           ATOMIC_SPECIES
           C 0.1d0 c_blyp_gia.pp
           H 0.1d0 h.ps
\end{verbatim}

(c) when the system is really close to the equilibrium, the damped dynamics 
slow down too, especially because, since we are moving electron and ions 
together, the ionic forces are not properly correct, then it is often better 
to perform a ionic step every N electronic steps, or to move ions only when
electron are in their GS (within the chosen threshold).
    
This can be specified by adding, in the ionic section, the ion\_nstepe
parameter, then the ionic input section become as follows:
\begin{verbatim} 
         &ions
           ion_dynamics = 'damp',
           ion_damping = 0.2,
           ion_velocities = 'zero',
           ion_nstepe = 10
         /
\end{verbatim}
Then we specify in the control input section:
\begin{verbatim} 
           etot_conv_thr = 1.d-6,
           ekin_conv_thr = 1.d-5,
           forc_conv_thr = 1.d-3
\end{verbatim}
As a result, the code checks every 10 electronic steps whether
the electronic system satisfies the two thresholds etot\_conv\_thr,
ekin\_conv\_thr: if it does, the ions are advanced by one step.
The process thus continues until the forces become smaller than
forc\_conv\_thr.

{\em Note} that to fully relax the system you need many run, and different 
strategies, that you shold mix and change in order to speed-up the convergence.
The process is not automatic, but is strongly based on experience, and trial 
and error.

Remember also that the convergence to the equilibrium positions depends on 
the energy threshold for the electronic GS, in fact correct forces (required
to move ions toward the minimum) are obtained only when electrons are in their 
GS. Then a small threshold on forces could not be satisfied, if you do not 
require an even smaller threshold on total energy.

2. Randomization of positions.
   
If you have relaxed the system or if the starting system is already in
the equilibrium positions, then you need to move ions from the equilibrium 
positions, otherwise they will nott move in a dynamics simulation.
After the randomization you should bring electrons on the GS again,
in order to start a dynamic with the correct forces and with electrons 
in the GS. Then you should switch off the ionic dynamics and activate 
the randomization for each species, specifying the amplitude of the 
randomization itself. This could be done with the following ionic input 
section:
\begin{verbatim}
          &ions
            ion_dynamics = 'none',
            tranp(1) = .TRUE.,
            tranp(2) = .TRUE.,
            amprp(1) = 0.01
            amprp(2) = 0.01
          /
\end{verbatim}
In this way a random displacement (of max 0.01 a.u.) is added to atoms of 
species 1 and 2. All other input parameters could remain the same.
Note that the difference in the total energy (etot) between relaxed and
randomized positions can be used to estimate the temperature that will
be reached by the system. In fact, starting with zero ionic velocities,
all the diffrence is potential energy, but in a dynamics simulation, the
energy will be equipartitioned between kinetic and potential, then to
estimate the temperature take the difference in energy (de), convert it
in Kelvins, divide for the number of atoms and multiply by 2/3.
Randomization could be useful also while we are relaxing the system,
especially when we suspect that the ions are in a local minimum or in
an energy plateau.

\subsection{CP dynamics}

At this point after having minimized the electrons, and with ions dis-
placed from their equilibrium positions, we are ready to start a CP
dynamics. We need to specify 'verlet' both in ionic and electronic
dynamics. The threshold in control input section will be ignored, like
any parameter related to minimization strategy. The first time we perform 
a CP run after a minimization, it is always better to put velocities equal
to zero, unless we have velocities, from a previous simulation, to
specify in the input file. Restore the proper masses for the ions. In this
way we will sample the microcanonical ensemble. The input section
changes as follow:
\begin{verbatim}
           &electrons
              emass = 400.d0,
              emass_cutoff = 2.5d0,
              electron_dynamics = 'verlet',
              electron_velocities = 'zero'
           /
           &ions
              ion_dynamics = 'verlet',
              ion_velocities = 'zero'
           /
           ATOMIC_SPECIES
           C 12.0d0 c_blyp_gia.pp
           H 1.00d0 h.ps
\end{verbatim}

If you want to specify the initial velocities for ions, you have to set
ion\_velocities ='from\_input', and add the IONIC\_VELOCITIES
card, after the ATOMIC\_POSITION card, with the list of velocities in 
atomic units.

{\em IMPORTANT:} in restarting the dynamics after the first CP run,
remember to remove or comment the velocities parameters:
\begin{verbatim}
           &electrons
              emass = 400.d0,
              emass_cutoff = 2.5d0,
              electron_dynamics = 'verlet'
              ! electron_velocities = 'zero'
           /
           &ions
              ion_dynamics = 'verlet'
              ! ion_velocities = 'zero'
           /
\end{verbatim}
otherwise you will quench the system interrupting the sampling of the
microcanonical ensemble.

\paragraph{ Varying the temperature }
   
It is possible to change the temperature of the system or to sample the 
canonical ensemble fixing the average temperature, this is done using 
the Nos\'e thermostat. To activate this thermostat for ions you have 
to specify in the ions input section:
\begin{verbatim}
           &ions
              ion_dynamics = 'verletâ',
              ion_temperature = 'nose',
              fnosep = 60.0,
              tempw = 300.0
           /  
\end{verbatim}
where fnosep is the frequency of the thermostat in THz, thatishould be
chosen to be comparable with the center of the vibrational spectrum of
the system, in order to excite as many vibrational modes as possible.
tempw is the desired average temperature in Kelvin.
   
{\em Note:} to avoid a strong coupling between the Nos\'e thermostat 
and the system, proceed step by step. Don't switch on the thermostat 
from a completely relaxed configuration: adding a random displacement
is strongly recommended. Check which is the average temperature via a
few steps of a microcanonical simulation. Don't increase the temperature
too much. Finally switch on the thermostat. In the case of molecular system,
different modes have to be thermalized: it is better to use a chain of 
thermostat or equivalently running different simulations with different 
frequencies. 
 
\paragraph{ No\'se thermostat for electrons }

It is possible to specify also the thermostat for the electrons. This is
usually activated in metals or in systems where we have a transfer of
energy between ionic and electronic degrees of freedom. Beware: the
usage of electronic thermostats is quite delicate. The following information 
comes from K. Kudin: 

''The main issue is that there is usually some "natural" fictitious kinetic 
energy that electrons gain from the ionic motion ("drag"). One could easily 
quantify how much of the fictitious energy comes from this drag by doing a CP 
run, then a couple of CG (same as BO) steps, and then going back to CP.
The fictitious electronic energy at the last CP restart will be purely 
due to the drag effect.''

''The thermostat on electrons will either try to overexcite the otherwise 
"cold" electrons, or it will try to take them down to an unnaturally cold 
state where their fictitious kinetic energy is even below what would be 
just due pure drag. Neither of this is good.''

''I think the only workable regime with an electronic thermostat is a 
mild overexcitation of the electrons, however, to do this one will need 
to know rather precisely what is the fictititious kinetic energy due to the
drag.''

\paragraph{ Self-interaction Correction }

The self-interaction correction (SIC) included in the CP package is based
on the Constrained Local-Spin-Density approach proposed my F. Mauri and 
coworkers (M. D'Avezac et al. PRB 71, 205210 (2005)). It was used for
the first time in \qe\ by F. Baletto, C. Cavazzoni 
and S.Scandolo (PRL 95, 176801 (2005)).

This approach is a simple and nice way to treat ONE, and only one, 
excess charge (EC). It is moreover necessary to check a priori that 
the spin-up and spin-down eigenvalues are not too different, for the 
corresponding neutral system. working in the Local-Spin-Density 
Approximation (setting nspin = 2). If these two conditions are satisfied
and you are interest in charged systems, you can apply the SIC.
This approach is a on-the-fly method to correct the self-interaction 
with the excess charge with itself.

Briefly, both the Hartree and the exchange-correlation part have been 
corrected to avoid the interaction of the EC with tself.

For example, for the Boron atoms, where we have an even number of 
electrons (valence electrons = 3), the parameters for working with
the SIC are:
\begin{verbatim}
           &system
           nbnd= 2,
           total_magnetization=1,
           sic_alpha = 1.d0,
           sic_epsilon = 1.0d0,
           sic = 'sic_mac',
           force_pairing = .true.,

           &ions
           ion_dynamics = 'none',
           ion_radius(1) = 0.8d0,
           sic_rloc = 1.0,

           ATOMIC_POSITIONS (bohr)
           B 0.00 0.00 0.00 0 0 0 1
\end{verbatim}
The two main parameters are:\\
"force\_pairing = .true.", which forces the paired electrons to be the same;\\ 
"sic='sic\_mac'," which instructs the code to use Mauri's correction.\\
Remember to add an extra-column in ATOMIC\_POSITIONS with "1" to activate
SIC for those atoms.

{\bf Warning}: 
This approach has known problems for dissociation mechanism
driven by excess electrons.

Comment 1:
Two parameters, ''sic\_alpha'' and ''sic\_epsilon'', have been introduced 
following the suggestion of M. Sprik (ICR(05)) to treat the radical
(OH)-H$_2$O. In any case, a complete ab-initio approach is followed 
using ''sic\_alpha=sic\_epsilon=1''.

Comment 2:
When you apply this SIC scheme to a molecule or to an atom, which are neutral,
remember to add the correction to the energy level as proposed by Landau: 
in a neutral system, subtracting the self-interaction, the unpaired electron
feels a charged system, even if using a compensating positive background. 
For a cubic box, the correction term due to the Madelung energy is approx. 
given by $1.4186/L_{box} - 1.047/(L_{box})^3$, where $L_{box}$ is the 
linear dimension of your box (=celldm(1)). The Madelung coefficient is 
taken from I. Dabo et al. PRB 77, 115139 (2007).

(info by F. Baletto, francesca.baletto@kcl.ac.uk)

\subsection{ Variable-cell MD }

The variable-cell MD is when the Car-Parrinello technique is also applied 
to the cell. This technique is useful to study system at very high pressure.

\subsection{ Conjugate Gradient }

This page is under construction.

\paragraph{ ensemble-DFT }

The ensemble-DFT (eDFT) is a robust method to simulate the metals in the 
framework of ''ab-initio'' molecular dynamics. It was introduced in 1997 
by Marzari et al.

The specific subroutines for the eDFT are in ensemble\_dft.f90 where you 
define all the quantities of interest. The subroutine inner\_loop\_cold.f90
called by cg\_sub.f90, control the inner loop, and so the minimization of 
the free energy $A$ with respect to the occupation matrix.

To select a eDFT calculations, the user has to choice:
\begin{verbatim}
            calculation = 'cp'
            occupations= 'ensemble' 
            tcg = .true.
            passop= 0.3
            maxiter = 250
\end{verbatim}
to use the CG procedure. In the eDFT it is also the outer loop, where thei
energy is minimized with respect to the wavefunction keeping fixed the 
occupation matrix. While the specific parameters for the inner loop.
Since eDFT was born to treat metals, keep in mind that we want to describe 
the broadening of the occupations around the Fermi energy.
Below the new parameters in the electrons list, are listed.
\begin{itemize}
\item smearing: used to select the occupation distribution;
there are two options: Fermi-Dirac smearing='fd', cold-smearing
smearing='cs' (recommanded) 
\item degauss: is the electronic temperature; it controls the broadening
of the occupation numbers around the Fermi energy. 
\item ninner: is the number of iterative cycles in the inner loop, 
done to minimize the free energy $A$ with respect the occupation numbers.
The typical range is 2-8.
\item conv\_thr: is the threshold value to stop the search of the 'minimum' 
free energy.
\item niter\_cold\_restart: controls the frequency at which a full iterative
inner cycle is done. It is in the range 1-ninner. It is a trick to speed up 
the calculation.
\item lambda\_cold: is the length step along the search line for the best 
value for $A$, when the iterative cycle is not performed. The value is close 
to 0.03, smaller for large and complicated metallic systems.
\end{itemize}
{\em NOTE:} degauss is in Hartree, while in \texttt{PWscf} is in Ry (!!!). 
The tyPical range is 0.01-0.02 Ha.

The input for an Al surface is:
\begin{verbatim}
            &CONTROL
             calculation = 'cp',
             restart_mode = 'from_scratch',
             nstep  = 10,
             iprint = 5,
             isave  = 5,
             dt    = 125.0d0,
             prefix = 'Aluminum_surface',
             pseudo_dir = '~/UPF/',
             outdir = '/scratch/'
             ndr=50
             ndw=51
            /
            &SYSTEM
             ibrav=  14,
             celldm(1)= 21.694d0, celldm(2)= 1.00D0, celldm(3)= 2.121D0,
             celldm(4)= 0.0d0,   celldm(5)= 0.0d0, celldm(6)= 0.0d0,
             nat= 96,
             ntyp= 1,
             nspin=1,
             ecutwfc= 15,
             nbnd=160,
             xc_type = 'pbe'
             occupations= 'ensemble',
             smearing='cs',
             degauss=0.018,
            /
            &ELECTRONS
             orthogonalization = 'Gram-Schmidt',
             startingwfc = 'random',
             ampre = 0.02,
             tcg = .true.,
             passop= 0.3,
             maxiter = 250,
             emass_cutoff = 3.00,
             conv_thr=1.d-6
             n_inner = 2,
             lambda_cold = 0.03,
             niter_cold_restart = 2,
            /
            &IONS
             ion_dynamics  = 'verlet',
             ion_temperature = 'nose'
             fnosep = 4.0d0,
             tempw = 500.d0
            /
            ATOMIC_SPECIES
             Al 26.89 Al.pbe.UPF
\end{verbatim}
{\em NOTA1}  remember that the time step is to integrate the ionic dynamics,
so you can choose something in the range of 1-5 fs. \\
{\em NOTA2} with eDFT you are simulating metals or systems for which the 
occupation number is also fractional, so the number of band, nbnd, has to 
be chosen such as to have some empty states. As a rule of thumb, start
with an initial occupation number of about 1.6-1.8 (the more bands you 
considera, the more the calculation is accurate, but it also takes longer.
The CPU time scales almost linearly with the number of bands.) \\
{\em NOTA3} the parameter emass\_cutoff is used during the preconditioning 
and it has a completely different meaning with respect to plain CP. 
It ranges between 4 and 7.

All the other parameters have the same meaning in the usual CP input, 
and they are discussed above.

\subsection{ About nr1b, nr2b, nr3b}

ecutrho defines the resolution on the real space FFT mesh (as expressed 
by nr1, nr2 and nr3, that the code left on its own sets automatically).
In the ultrasoft case we refer to this mesh as the "hard" mesh, since it 
is denser than the smooth mesh that is needed to represent the square 
of the non-norm-conserving wavefunctions.
  
On this "hard", fine-spaced mesh, you need to determine the size of the
cube that will encompass the largest of the augmentation charges - this
is what nr1b, nr2b, nr3b are.
  
So, nr1b is independent of the system size, but dependent on the size
of the augmentation charge (that doesn't vary that much) and on the
real-space resolution needed by augmentation charges (rule of thumb:
ecutrho is between 6 and 12 times ecutwfc).

In practice, nr1b et al. are often in the region of 20-24-28; testing seems
again a necessity (unless the code started automagically to estimate these).

The core charge is in principle finite only at the core region (as defined
by rcut ) and vanishes out side the core. Numerically the charge is
represented in a Fourier series which may give rise to small charge
oscillations outside the core and even to negative charge density, but
only if the cut-off is too low. Having these small boxes removes the
charge oscillations problem (at least outside the box) and also offers
some numerical advantages in going to higher cut-offs.

The small boxes should be set as small as possible, but large enough
to contain the core of the largest element in your system.
The formula for determining the box size is quite simple: 
$$
   nr1b = (2 \times rcut )/Lx \times nr1
$$
where rcut is the cut-off radius for the largest element and Lx is the
physical length of your box along the x axis. You have to round your
result to the nearest larger integer." (info by Nicola Marzari)

\section{Performance issues (\texttt{PWscf})}

\subsection{CPU time requirements}

The following holds for code pw.x and for non-US PPs. For US PPs there
are additional terms to be calculated, that may add from a few percent 
up to 30-40% to execution time. For phonon calculations, each of the
$3N_{at}$ modes requires a CPU time of the same order of that required by a
self-consistent calculation in the same system. For cp.x, the required CPU 
time of each time step is in the order of the time $T_h + T_{orth} + T_{sub}$ 
defined below.

The computer time required for the self-consistent solution at fixed ionic
positions, $T_{scf}$ , is:
$$T_{scf} = N_{iter} T_{iter} + T_{init}$$
where $N_{iter}$ = niter = number of self-consistency iterations, 
$T_{iter}$ = CPU
time for a single iteration, $T_{init}$ = initialization time for a single
iteration. Usually $T_{init} << N_{iter} T_{iter}$ .
    
The time required for a single self-consistency iteration $T_{iter}$ is:
$$T_{iter} = N_k T_{diag} +T_{rho} + T_{scf}$$
where $N_k$ = number of k-points, $T_{diag}$ = CPU time per 
hamiltonian iterative diagonalization, $T_{rho}$ = CPU time for charge density 
calculation, $T_{scf}$ = CPU time for Hartree and exchange-correlation potential
calculation.
    
The time for a Hamiltonian iterative diagonalization $T_{diag}$ is:
$$T_{diag} = N_h T_h + T_{orth} + T_{sub}$$
where $N_h$ = number of $H\psi$ products needed by iterative diagonalization,
$T_h$ = CPU time per $H\psi$ product, $T_{orth}$ = CPU time for 
orthonormalization, $T_{sub}$ = CPU time for subspace diagonalization.
    
The time $T_h$ required for a $H\psi$ product is
$$T_h = a_1 M N + a_2 M N_1 N_2 N_3 log(N_1 N_2 N_3 ) + a_3 M P N. $$
The first term comes from the kinetic term and is usually much smaller
than the others. The second and third terms come respectively from local
and nonlocal potential. $a_1, a_2, a_3$ are prefactors, M = number of valence
bands, N = number of plane waves (basis set dimension), $N_1, N_2, N_3$ =
dimensions of the FFT grid for wavefunctions ($N_1 N_2 N_3 \sim 8N$ ), 
P = number of projectors for PPs (summed on all atoms, on all values of the
angular momentum l, and m = 1, . . . , 2l + 1)
    
The time $T_{orth}$ required by orthonormalization is
$$T_{orth} = b_1 N M_x^2$$ 
and the time $T_{sub}$ required by subspace diagonalization is
$$T_{sub} = b_2 M_x^3$$
where $b_1$ and $b_2$ are prefactors, $M_x$ = number of trial wavefunctions 
(this will vary between M and a few times M , depending on the algorithm).
    
The time $T_{rho}$ for the calculation of charge density from wavefunctions is
$$T_{rho} = c_1 M N_{r1} N_{r2}N_{r3} log(N_{r1} N_{r2} N_{r3}) + 
            c_2 M N_{r1} N_{r2} N_{r3} + T_{us}$$
where $c_1, c_2, c_3$ are prefactors, $N_{r1}, N_{r2}, N_{r3}$ =
dimensions of the FFT grid for charge density ($N_{r1} N_{r2} N_r3 \sim 8N_g$,
where $N_g$> = number of G-vectors for the charge density), and 
$T_{us}$ = CPU time required by ultrasoft contribution (if any).
 
The time $T_{scf}$ for calculation of potential from charge density is
$$T_{scf} = d_2 N_{r1} N_{r2} N_{r3} + d_3 N_{r1} N_{r2} N_{r3} 
            log(N_{r1} N_{r2} N_{r3} )$$
where $d_1, d_2$ are prefactors.

\subsection{Memory requirements}

A typical self-consistency or molecular-dynamics run requires a maximum
memory in the order of O double precision complex numbers, where
$$ O = m M N + P N + p N_1 N_2 N_3 + q N_{r1} N_{r2} N_{r3}$$
with m, p, q = small factors; all other variables have the same meaning as
above. Note that if the $\Gamma-$point only (q = 0) is used to sample the 
Brillouin Zone, the value of N will be cut into half.

The memory required by the phonon code follows the same patterns, with
somewhat larger factors m, p, q .

\subsection{File space requirements}

A typical pw.x run will require an amount of temporary disk space in the
order of O double precision complex numbers:
$$O = N_k M N + q N_{r1} N_{r2}N_{r3}$$
where $q = 2\times$ mixing\_ndim (number of iterations used in 
self-consistency, default value = 8) if disk\_io is set to 'high' 
or not specified; q = 0 if disk\_io='low' or 'minimal'.

\subsection{Parallelization issues}
\label{SubSec:badpara}

pw.x and cp.x can run in principle on any number of processors.
The effectiveness of parallelization is ultimately judged by the 
''scaling'', i.e. how the time needed to perform a job scales
 with the number of processors, and depends upon:
\begin{itemize}
\item the size and type of the system under study;
\item the judicious choice of the various levels of parallelization 
(detailed in Sec.\ref{SubSec:para});
\item the availability of fast interprocess communications (or lack thereof).
\end{itemize}
Ideally one would like to have linear scaling, i.e. $T_N \sim T_0/N_p$ for 
$N_p$ processors. In addition, one would like to have linear scaling of
the RAM per processor: $O_N \sim O_0/N_p$, so that large-memory systems
fit into the RAM of each processor.

As a general rule, image parallelization:
\begin{itemize}
\item  may give good scaling, but the slowest image will determine
the overall performances (''load balancing'' may be a problem);
\item requires very little communications (suitable for ethernet 
communications);
\item does not reduce the required memory per processor (unsuitable for 
large-memory jobs).
\end{itemize}
Parallelization on k-points:
\begin{itemize}
\item guarantees (almost) linear scaling if the number of k-points
is a multiple of the number of pools;
\item requires little communications (suitable for ethernet communications);
\item does not reduce the required memory per processor (unsuitable for 
large-memory jobs).
\end{itemize}
Parallelization on plane-waves:
\begin{itemize}
\item yields good to very good scaling, especially if the number of processors
in a pool is a divisor of $N_3$ and $N_{r3}$ (the dimensions along the z-axis 
of the FFT grids, which may coincide);
\item requires heavy communications (suitable for Gigabit ethernet up to 
4, 8 CPUs at most, specialized communication hardware needed for 8 or more
processors );
\item yields almost linear reduction of memory per processor with the number
of processors in the pool.
\end{itemize}

A note on scaling: optimal serial performances are achieved when the data are
as much as possible kept into the cache. As a side effect, plane-wave
parallelization may yield superlinear (better than linear) scaling,
thanks to the increase in serial speed coming from the reduction of data size 
(making it easier for the machine to keep data in the cache).

For each system there is an optimal range of number of processors on which to 
run the job.  A too large number of processors will yield performance 
degradation. If the size of pools is especially delicate: $N_p$ should not 
exceed by much $N_3$ and $N_{r3}$. For large jobs, it is convenient to 
further subdivide a pool of processors into ''task groups''.
When the number of processors exceeds the number of FFT planes, 
data can be redistributed to "task groups" so that each group 
can process several wavefunctions at the same time.

The optimal number of processors for the ''ortho'' (cp.x) or ''ndiag'' 
(pw.x) parallelization, taking care of linear algebra operations 
involving $M\times M$ matrices, is automatically chosen by the code.

Actual parallel performances will also depend a lot on the available software 
(MPI libraries) and on the available communication hardware. For
Beowulf-style machines (clusters of PC) the newest version 1.1 and later of 
the OpenMPI libraries (http://www.openmpi.org/) seems to yield better 
performances than other implementations (info by Kostantin Kudin). 
Note however that you need a decent communication hardware (at least 
Gigabit ethernet) in order to have acceptable performances with 
PW parallelization. Do not expect good scaling with cheap hardware: 
plane-wave calculations are by no means an "embarrassing parallel" problem.
   
Also note that multiprocessor motherboards for Intel Pentium CPUs typically 
have just one memory bus for all processors. This dramatically
slows down any code doing massive access to memory (as most codes 
in the \qe\ distribution do) that runs on processors of the same
motherboard.
\section{Troubleshooting}

Almost all problems in \qe\ arise from incorrect input data 
and result in
error stops. Error messages should be self-explanatory, but unfortunately
this is not always true. If the code issues a warning messages and continues,
pay attention to it but do not assume that something is necessarily wrong in
your calculation: most warning messages signal harmless problems.
    
Typical pw.x and/or ph.x (mis-)behavior:

\subsection{pw.x says 'error while loading shared libraries' or
  'cannot open shared object file' and does not start} 
Possible reasons:
\begin{itemize}
\item If you are running on the same machines on which the code was
  compiled, this is a library configuration problem. The solution is
  machine-dependent. On Linux, find the path to the missing libraries;
  then either add it to file /etc/ld.so.conf and run ldconfig (must be
  done as root), or add it to variable LD\_LIBRARY\_PATH and export
  it. Another possibility is to load non-shared version of libraries
  (ending with .a)  instead of shared ones (ending with .so). 
\item If you are {\em not} running on the same machines on which the
  code was compiled: you need either to have the same shared libraries
  installed on both machines, or to load statically all libraries
  (using appropriate \texttt{configure} or loader options). The same applies to
  Beowulf-style parallel machines: the needed shared libraries must be
  present on all PCs. 
\end{itemize}

\subsection{errors in examples with parallel execution}

If you get error messages in the example scripts -- i.e. not errors in
the codes -- on a parallel machine, such as e.g.: \\
"run example: -n: command not found" \\
you may have forgotten 
the "'" in the definitions of PARA\_PREFIX and PARA\_POSTFIX.

\subsection{pw.x prints the first few lines and then nothing happens
  (parallel execution)} 
If the code looks like it is not reading from input, maybe
it isn't: the MPI libraries need to be properly configured to accept input
redirection. See Sec.\ref{SubSec:para}, or inquire with
your local computer wizard (if any).

\subsection{pw.x stops with error while reading data}
There is an error in the input data, typically:
\begin{itemize}
\item a misspelled namelist variable,
\item an empty input file.
\end{itemize}
Unfortunately with most compilers the code just reports "Error while
reading XXX namelist" and no further useful information.
Here are some other more subtle sources of trouble:
\begin{itemize}
\item Out-of-bound indices in dimensioned variables read in the namelist
\item Input data files containing \^M (Control-M) characters at the end
  of lines, or non-ASCII characters (e.g. non-ASCII quotation marks,
  that at a first glance may look the same as the ASCII
  character). Typically, this happens with files coming from Windows
  or produced with "smart" editors.  
\end{itemize}
Both may cause the code to crash with rather mysterious error messages.
If none of the above applies and the code stops at the first namelist
("control") and you are running in parallel: your MPI libraries might
not be properly configured to allow input redirection, see the
previous item above this one. 

\subsection{pw.x mumbles something like 'cannot recover' or 'error
  reading recover file'} 
You are trying to restart from a previous job that either
produced corrupted files, or did not do what you think it did. No luck: you
have to restart from scratch.

\subsection{pw.x stops with 'inconsistent DFT' error}
As a rule, the flavor of DFT used in the calculation should be the
same as the one used in the generation of PPs, and all PPs should be
generated using the same flavor of DFT. This is actually enforced: the
type of DFT is read from PP files and it is checked that the same DFT
is read from all PPs. If this does not hold, the code stops with the
above error message. 

If you really want to use PPs generated with different DFT approximations, 
or to perform a calculation with a DFT that differs from what used in PP 
generation: use input variable \texttt{input\_dft} to force the usage of 
the DFT you like. Use at your own risk.

\subsection{pw.x stops with error in cdiaghg or rdiaghg}
Possible reasons for such behavior are not always clear, but they
typically fall into one of the following cases:
\begin{itemize}
\item serious error in data, such as bad atomic positions or bad
  crystal structure/supercell; 
\item a bad PP, typically with a ghost, but also a US-PP with
  non-positive charge density, leading to a violation of positiveness
  of the S matrix appearing in the US-PP formalism;  
\item a failure of the algorithm performing subspace
  diagonalization. The LAPACK algorithms used by cdiaghg/rdiaghg are
  very robust and extensively tested. Still, it may seldom happen that
  such algorithms fail. Try to use conjugate-gradient diagonalization
  (diagonalization='cg'), a slower but very robust algorithm, and see
  what happens. 
\item buggy libraries. Machine-optimized mathematical libraries are
  very fast but sometimes not so robust from a numerical point of
  view.  Suspicious behavior: you get an error that is not
  reproducible on other architectures or that disappears if the
  calculation is repeated with even minimal changes in
  parameters. Known cases: HP-Compaq alphas with cxml libraries, Mac
  OS-X with system blas/lapack. Try to use compiled BLAS and LAPACK
  (or better, ATLAS) instead of machine-optimized libraries. 
\end{itemize}

\subsection{pw.x crashes with no error message at all}
This happens quite often in parallel execution, or under a batch
queue, or if you are writing the output to a file. When the program
crashes, part of the output, including the error message, may be lost,
or hidden into error files where nobody looks into. It is the fault of
the operating system, not of the code. Try to run interactively 
and to write to the screen. If this doesn't help, move to next point.

\subsection{pw.x crashes with 'segmentation fault' or similarly
  obscure messages} 
Possible reasons:
\begin{itemize}
\item  too much RAM memory requested, or too much stack memory
  requested (see next item). 
\item if you are using highly optimized mathematical libraries, verify
  that they are designed for your hardware. In particular, for Intel
  compiler  and MKL libraries, verify that you loaded the correct set
  of CPU-specific MKL libraries. 
\item buggy compiler. If you are using Portland or Intel compilers on
  Linux PCs or clusters, see the Installation section. 
\end{itemize}

\subsection{pw.x crashes with 'error in davcio'}
'davcio' is the routine that performs most of the I/O operations (read
from disk and write to disk) in pw.x ; 'error in davcio 'means a
failure of an I/O operation.  
\begin{itemize}
\item If the error is reproducible and happens at the beginning of a
  calculation: check if you have read/write permission to the scratch
  directory specified in variable 'outdir'. Also: check if there is
  enough free space available on the disk you are writing to, and
  check your disk quota (if any) 
\item If the error is irreproducible: your might have flaky disks; if
  you are writing via the network using nfs (which you shouldn't do
  anyway), your network connection might be not so stable, or your nfs
  implementation is unable to work under heavy load 
\item if it happens while restarting from a previous calculation: you
  might be restarting from the wrong place, or from wrong  data, or
  the files might be corrupted. 
\end{itemize}

\subsection{pw.x works for simple systems, but not for large systems
  or whenever more RAM is needed}  
Possible solutions:
\begin{itemize}
\item increase the amount of RAM you are authorized to use (which may
  be much smaller than the available RAM). Ask your system
  administrator if you don't know what to do.  
\item reduce nbnd to the strict minimum, or reduce the cutoffs, or the
  cell size 
\item  use conjugate-gradient (diagonalization='cg': slow but very
  robust): it requires less memory than the default Davidson
  algorithm. 
\item in parallel execution, use more processors, or use the same
  number of processors with less pools. Remember that parallelization
  with respect to k-points (pools) does not distribute memory:
  parallelization with respect to R- (and G-) space does. 
\item IBM only (32-bit machines): if you need more than 256 MB you
  must specify it at link time (option -bmaxdata). 
\item buggy or weird-behaving compiler. Some versions of the Portland
  and Intel compilers on Linux PCs or clusters have this problem. For
  Intel ifort 8.1 and later, the problem seems to be due to the
  allocation of large automatic arrays that exceeds the available
  stack. Increasing the stack size (with commands limits or ulimit)
  may (or may not) solve the problem. Versions $> 3.2$ try to avoid this
  problem by removing the stack size limit at startup. See:\\
  http://www.democritos.it/pipermail/pw\_forum/2007-September/007176.html,\\
  http://www.democritos.it/pipermail/pw\_forum/2007-September/007179.html 
\end{itemize}

\subsection{pw.x crashes in parallel execution with an obscure message
  related to MPI errors} 
Random crashes due to MPI errors have often been reported, typically
in Linux PC clusters. We cannot rule out the possibility that bugs in
\qe\ cause such behavior, but we are quite confident that
the most likely explanation is a hardware problem (defective RAM  
for instance) or a software bug (in MPI libraries, compiler, operating
system). 

Debugging a parallel code may be difficult, but you should at least
verify if your problem is reproducible on different
architectures/software configurations/input data sets, and if  
there is some particular condition that activates the bug. If this
doesn't seem to happen, the odds are that the problem is not in
\qe. You may still report your problem, 
but consider that reports like "it crashes with...(obscure MPI error)"
contain 0 bits of information and are likely to get 0 bits of answers.

Concerning MPI libraries in particular, useful information can be
found in Axel's web site: \\
http://www.theochem.rub.de/~axel.kohlmeyer/cpmd-linux.html, and in the
following message by Javier Antonio Montoya: \\
http://www.democritos.it/pipermail/pw\_forum/2008-April/008818.html

\subsection{pw.x runs but nothing happens}
Possible reasons:
\begin{itemize}
\item in parallel execution, the code died on just one
  processor. Unpredictable behavior may follow. 
\item in serial execution, the code encountered a floating-point error
  and goes on producing NaNs (Not a Number) forever unless exception
  handling is on (and usually it isn't). In both cases, look for one
  of the reasons given above. 
\item maybe your calculation will take more time than you expect.
\end{itemize}

\subsection{pw.x yields weird results}
Possible solutions:
\begin{itemize}
\item if this happens after a change in the code or in compilation or
  preprocessing options, try 'make clean', recompile. The 'make'
  command should take care of all dependencies, but do not rely too
  heavily on it. If the problem persists, 'make clean', recompile with
  reduced optimization level.  
\item maybe your input data are weird.
\end{itemize}

\subsection{pw.x stops with error message "the system is metallic,
  specify occupations"} 
You did not specify state occupations, but you need to, since your
system appears to have an odd number of electrons. The variable
controlling how metallicity is treated is occupations in namelist
\&SYSTEM. The default, occupations='fixed', occupies the lowest
(N electrons)/2 states and works only for insulators with a gap. In all other
cases, use 'smearing' or 'tetrahedra'. See file INPUT\_PW for more details.

\subsection{pw.x stops with "internal error: cannot braket Ef " in efermig}
Possible reasons:
\begin{itemize}
\item serious error in data, such as bad number of electrons,
  insufficient number of bands, absurd value of broadening; 
\item the Fermi energy is found by bisection assuming that the
  integrated DOS N(E ) is an increasing function of the energy. This
  is not guaranteed for Methfessel-Paxton smearing of order 1 and can
  give problems when very few k-points are used. Use some other
  smearing function: simple Gaussian broadening or, better,
  Marzari-Vanderbilt 'cold smearing'. 
\end{itemize}

\subsection{pw.x yields 'internal error: cannot braket Ef' message in
  efermit, then stops because 'charge is incorrect'}  
There is either a serious error in data (bad number of electrons,
insufficient number of bands), or too few tetrahedra
(i.e. k-points). The tetrahedron method may become unstable in the
latter case, especially if the bands are very narrow. Remember that 
tetrahedra should be used only in conjunction with uniform k-point grids.

\subsection{pw.x yields 'internal error: cannot braket Ef' message in
  efermit but does not stop} 
This may happen under special circumstances when you are calculating
the band structure for selected high-symmetry lines. The message
signals that occupations and Fermi energy are not correct (but
eigenvalues and eigenvectors are). Remove occupations='tetrahedra' in
the input data to get rid of the message. 

\subsection{the FFT grids in pw.x are machine-dependent}
Yes, they are! The code automatically chooses the smallest grid that
is compatible with the 
specified cutoff in the specified cell, and is an allowed value for the FFT
library used. Most FFT libraries are implemented, or perform well, only
with dimensions that factors into products of small numers (2, 3, 5 typically,
sometimes 7 and 11). Different FFT libraries follow different rules and thus
different dimensions can result for the same system on different machines (or
even on the same machine, with a different FFT). See function allowed in
Modules/fft\_scalar.f90.
    
As a consequence, the energy may be slightly different on different machines. 
The only piece that explicitly depends on the grid parameters is
the XC part of the energy that is computed numerically on the grid. The
differences should be small, though, especially for LDA calculations.

Manually setting the FFT grids to a desired value is possible, but slightly
tricky, using input variables nr1, nr2, nr3 and nr1s, nr2s, nr3s. The
code will still increase them if not acceptable. Automatic FFT grid 
dimensions are slightly overestimated, so one may try {\em very carefully}
to reduce
them a little bit. The code will stop if too small values are required, it will
waste CPU time and memory for too large values.
    
Note that in parallel execution, it is very convenient to have FFT grid
dimensions along $z$ that are a multiple of the number of processors.

\subsection{pw.x does not find all the symmetries you expected} 
pw.x determines first the symmetry operations (rotations) of the
Bravais lattice; then checks which of these are symmetry operations of
the system (including if needed fractional translations). This is done
by rotating (and translating if needed) the atoms in the unit cell and
verifying if the rotated unit cell coincides with the original one.

Assuming that your coordinates are correct (please carefully check!),
you may not find all the symmetries you expect because:
\begin{itemize}
\item the number of significant figures in the atomic positions is not
  large enough. In file PW/eqvect.f90, the variable accep is used to
  decide whether a rotation is a symmetry operation. Its current value
  ($10^{-5}$) is quite strict: a rotated atom must coincide with
  another atom to 5 significant digits. You may change the value of
  accep and recompile. 
\item they are not acceptable symmetry operations of the Bravais
  lattice. This is the case for C$_{60}$, for instance: the $I_h$
  icosahedral group of C$_{60}$ contains 5-fold rotations that are
  incompatible with translation symmetry.  
\item  the system is rotated with respect to symmetry axis. For
  instance: a C$_{60}$ molecule in the fcc lattice will have 24
  symmetry operations ($T_h$ group) only if the double bond is
  aligned along one of the crystal axis; if C$_{60}$ is rotated
  in some arbitrary way, pw.x may not find any symmetry, apart from
  inversion. 
\item they contain a fractional translation that is incompatible with
  the FFT grid (see next paragraph). Note that if you change cutoff or
  unit cell volume, the automatically computed FFT grid changes, and
  this may explain changes in symmetry (and in the number of k-points
  as a consequence) for no apparent good reason (only if you have
  fractional translations in the system, though). 
\item a fractional translation, without rotation, is a symmetry
  operation of the system. This means that the cell is actually a
  supercell. In this case, all symmetry operations containing
  fractional translations are disabled. The reason is that in this
  rather exotic case there is no simple way to select those symmetry
  operations forming a true group, in the mathematical sense of the
  term. 
\end{itemize}

\subsection{warning: 'symmetry operation \# N not allowed'}
This is not an error.  If a symmetry operation contains a fractional
translation that is incompatible with the FFT grid, it is discarded in
order to prevent problems with symmetrization. Typical fractional 
translations are 1/2 or 1/3 of a lattice vector. If the FFT grid
dimension along that direction is not divisible respectively by 2 or
by 3, the symmetry operation will not transform the FFT grid into
itself. 

\subsection{I do not get the same results in different machines!}
If the difference is small, do not panic. It is quite normal for
iterative methods to reach convergence through different paths as soon
as anything changes. In particular, between serial and parallel
execution there are operations that are not performed in the same
order. As the numerical accuracy of computer numbers is finite, this
can yield slightly different results. 

It is also normal that the total energy converges to a better accuracy
than its terms, since only the sum is variational, i.e. has a minimum
in correspondence to ground-state charge density. Thus if the
convergence threshold is for instance $10^{-8}$, you get 8-digit
accuracy on the total energy, but one or two less on other terms
(e.g. XC and Hartree energy). It this is a problem for you, reduce the
convergence threshold for instance to  $10^{-10}$ or  $10^{-12}$. The
differences should go away (but it will probably take a few more
iterations to converge). 

\subsection{the CPU time is time-dependent!}
Yes it is! On most machines and on
most operating systems, depending on machine load, on communication load
(for parallel machines), on various other factors (including maybe the phase
of the moon), reported CPU times may vary quite a lot for the same job.
Also note that what is printed is supposed to be the CPU time per process,
but with some compilers it is actually the wall time.

\subsection{'warning : N eigenvectors not converged ...'}
This is a warning message that can be safely ignored if it is not
present in the last steps of self-consistency. If it is still present
in the last steps of self-consistency, and if the number of
unconverged eigenvector is a significant part of the total, it may
signal serious trouble in self-consistency (see next point) or
something badly wrong in input data.

\subsection{'warning : negative or imaginary charge...', or '...core
  charge ...', or 'npt with rhoup$<0$...' or 'rho dw$<0$...'} 
These are warning messages that can be safely ignored unless the
negative or imaginary charge is sizable, let us say of the order of
0.1. If it is, something seriously wrong is going on. Otherwise, the
origin of the negative charge is the following. When one transforms a
positive function in real space to Fourier space and truncates at some
finite cutoff, the positive function is no longer guaranteed to be
positive when transformed back to real space. This happens only with
core corrections and with ultrasoft pseudopotentials. In some cases it
may be a source of trouble (see next point) but it is usually solved
by increasing the cutoff for the charge density.

\subsection{self-consistency is slow or does not converge}
See the corresponding FAQ item

\subsection{structural optimization is slow or does not converge}
Typical structural optimizations, based on the BFGS algorithm,
converge to the default thresholds ( etot\_conv\_thr and
forc\_conv\_thr ) in 15-25 BFGS steps (depending on the 
starting configuration). This may not happen when your
system is characterized by "floppy" low-energy modes, that make very
difficult (and of little use anyway) to reach a well converged structure, no
matter what. Other possible reasons for a problematic convergence are listed
below.
    
Close to convergence the self-consistency error in forces may become large
with respect to the value of forces. The resulting mismatch between forces
and energies may confuse the line minimization algorithm, which assumes
consistency between the two. The code reduces the starting self-consistency
threshold conv thr when approaching the minimum energy configuration, up
to a factor defined by upscale. Reducing conv\_thr (or increasing upscale)
yields a smoother structural optimization, but if conv\_thr becomes too small,
electronic self-consistency may not converge. You may also increase variables
etot\_conv\_thr and forc\_conv\_thr that determine the threshold for
convergence (the default values are quite strict).
    
A limitation to the accuracy of forces comes from the absence of perfect
translational invariance. If we had only the Hartree potential, our PW
calculation would be translationally invariant to machine
precision. The presence of an exchange-correlation potential
introduces Fourier components in the potential that are not in our
basis set. This loss of precision (more serious for gradient-corrected
functionals) translates into a slight but detectable loss 
of translational invariance (the energy changes if all atoms are displaced by
the same quantity, not commensurate with the FFT grid). This sets a limit
to the accuracy of forces. The situation improves somewhat by increasing
the ecutrho cutoff.

\subsection{pw.x stops during variable-cell optimization in
  checkallsym with 'non orthogonal operation' error} 
Variable-cell optimization may occasionally break the starting
symmetry of the cell. When this happens, the run is stopped because
the number of k-points calculated for the starting configuration may
no longer be suitable. Possible solutions: 
\begin{itemize}
\item start with a nonsymmetric cell;
\item use a symmetry-conserving algorithm: the Wentzcovitch algorithm
  (cell dynamics='damp-w') should not break the symmetry. 
\end{itemize}

\subsection{some codes in PP/ complain that they do not find some files}
For Linux PC clusters in parallel execution: in at least some versions
of MPICH, the current directory is set to the directory where the executable
code resides, instead of being set to the directory where the code is executed.
This MPICH weirdness may cause unexpected failures in some postprocessing
codes that expect a data file in the current directory. Workaround: use
symbolic links, or copy the executable to the current directory.

\subsection{ph.x stops with 'error reading file'}
The data file produced by pw.x
is bad or incomplete or produced by an incompatible version of the code.
In parallel execution: if you did not set wf collect=.true., the number
of processors and pools for the phonon run should be the same as for the
self-consistent run; all files must be visible to all processors.

\subsection{ph.x mumbles something like 'cannot recover' or 'error
  reading recover file'} 
You have a bad restart file from a preceding failed execution.
Remove all files recover* in outdir.

\subsection{ph.x says 'occupation numbers probably wrong' and
  continues; or 'phonon + tetrahedra not implemented' and stops} 
You have a
metallic or spin-polarized system but occupations are not set to 'smearing'.
Note that the correct way to calculate occupancies must be specified in the
input data of the non-selfconsistent calculation, if the phonon code reads
data from it. The non-selfconsistent calculation will not use this information
but the phonon code will.

\subsection{ph.x does not yield acoustic modes with $\omega=0$ at $q=0$}
This may not be an error: the Acoustic Sum Rule (ASR) is never exactly
verified, because the system is never exactly translationally
invariant as it should be.  The calculated frequency of the acoustic
mode is typically less than 10 cm$^{-1}$, but in some cases it may be
much higher, up to 100 cm$^{-1}$. The ultimate test is to diagonalize
the dynamical matrix with program dynmat.x, imposing the ASR. If you
obtain an acoustic mode with a much smaller $\omega$ (let us say 
$< 1 \mbox{cm}^{-1}$ ) 
with all other modes virtually unchanged, you can trust your results.

''The problem is [...] in the fact that the exchange and correlation
energy is computed in real space on a discrete grid and hence the
total energy is invariant (...) only for translation in the FFT
grid. Increasing the charge density cutoff increases the grid density
thus making the integral more exact thus reducing the problem,
unfortunately rather slowly...This problem is usually more severe for
GGA  than with LDA because the GGA functionals have functional forms
that vary more strongly with the position; particularly so for
isolated molecules or system with significant portions of "vacuum"
because in the exponential tail of the charge density a) the finite
cutoff  (hence there is an effect due to cutoff) induces oscillations
in rho and b) the reduced gradient is diverging.''(info by Stefano de
Gironcoli, June 2008) 

\subsection{ph.x yields really lousy phonons, with bad or negative
  frequencies or wrong symmetries or gross ASR violations} 
Possible reasons
\begin{itemize}
\item if this happens only for acoustic modes at $q=0$ that should
  have $\omega=0$: Acoustic Sum Rule violation, see the item before
  this one 
\item wrong data file read.
\item wrong atomic masses given in input will yield wrong frequencies
  (but the content of file fildyn should be valid, since the force
  constants, not the dynamical matrix, are written to file). 
\item convergence threshold for either SCF (conv\_thr) or phonon
  calculation (tr2\_ph) too large: try to reduce them. 
\item maybe your system does have negative or strange phonon
  frequencies, with the approximations you used. A negative frequency
  signals a mechanical instability of the chosen structure. Check that
  the structure is reasonable, and check the following parameters: 
\begin{itemize}
\item The cutoff for wavefunctions, ecutwfc
\item For US PP: the cutoff for the charge density, ecutrho
\item The k-point grid, especially for metallic systems!
\end{itemize}
\end{itemize}
Note that "negative" frequencies are actually imaginary: the negative
sign flags eigenvalues of the dynamical matrix for which $\omega^2 <
0$. 

\subsection{'Wrong degeneracy' error in star\_q}
Verify the q-point for which you are calculating phonons. In order to
check whether a symmetry operation belongs to the small group of q,
the code compares q and the rotated q, with an acceptance tolerance of  
$10^{-5}$ (set in routine PW/eqvect.f90). You may run into trouble if
your q-point differs from a high-symmetry point by an amount in that
order of magnitude.

\section{Frequently Asked Questions (FAQ)}

\subsection{General}

If you search information on \qe, the best starting point is the web site 
\texttt{html://www.quantum-espresso.org}. See in particular the
links ``learn'' for documentation, ``contacts'' if you need 
somebody to talk with. The mailing list \texttt{pw\_forum} is
the typical place where to ask questions about \qe.

% how/where to submit problems 
% whom to contact for which problem (download, web, wiki, qeforge,
% mailing list, bug, help ...)
% how to contact maintainers
% how to submit a bug report

\subsection{Installation}

Most installation problems have obvious origins and can be solved by reading
error messages and acting accordingly. Sometimes the reason for a failure
is less obvious. In such a case, you should look into 
Sec.\ref{Sec:Installation}, and into the pw\_forum archive to
see if a similar problem (with solution) is described. If you get
really weird error messages during installation, look for them with
your preferred Internet search engine (such as Google): very often you
will find an explanation and a workaround. 

\paragraph{What Fortran compiler do I need to compile Q-E?}

Any non-buggy, or not-too-buggy, fortran-95 compiler should work,
with minimal or no changes to the code. \texttt{configure} may not 
be able to recognize your system, though.

\paragraph{Why is \texttt{configure} saying that I have no fortran compiler?}

Because you haven't one (really!); or maybe you have one, but it is not
in your execution path; or maybe it has been given an unusual name by your 
system manager. Install a compiler if you have none; if you have one, fix 
your execution path, or define an alias if it has a strange name.

\paragraph{Why is \texttt{configure} saying that my fortran compiler doesn't work?}

Because it doesn't work (really!); more exactly, \texttt{configure} has tried 
to compile a small test program and didn't succeed. Your compiler may not be 
properly installed. For Intel compiler on PC's: you may have forgotten to run 
the required initialization script for the compiler.

\paragraph{\texttt{configure} doesn't recognize my system, what should I do?}

If compilation/linking works, never mind, Otherwise, try to supply a suitable  
supported architecture, or/and manually edit the \texttt{make.sys} file. 
Detailed instructions in Sec.\ref{Sec:Installation}.

\paragraph{Why doesn't \texttt{configure} recognize that I have a parallel machine?}

You need a properly configured complete parallel environment. If any piece is
missing, \texttt{configure} will revert to serial compilation. In particular: 
\begin{itemize}
\item \texttt{configure} tries to locate a parallel compiler in a logical
  place with a logical name,  but if it has a strange names or it is
  located  in a strange location, you will have to instruct \texttt{configure}
  to find it. Note that in many PC clusters (Beowulf), there is no
  parallel Fortran-95 compiler in default installations:  you have to
  configure an appropriate script, such as mpif90. 
\item \texttt{configure} tries to locate libraries (both mathematical and
  parallel libraries) in the usual places with usual names, but if
  they have strange names or strange locations, you will have to
  rename/move them, or to instruct \texttt{configure} to find them (see
   Sec.\ref{Sec:Installation}). Note that if MPI libraries are not found,
  parallel compilation is disabled. 
\item \texttt{configure} tests that the compiler and the libraries are
  compatible (i.e. the compiler may link the libraries without
  conflicts and without missing symbols). If they aren't and the
  compilation fail, \texttt{configure} will revert to serial compilation. 
\end{itemize}

\paragraph{Compilation fails with "internal error", what should I do?}

Any message during compilation saying something like "internal compiler error" 
and the like means that your compiler is buggy. You should report the problem 
to the compiler maker -- especially if you paid real money for it.
Sometimes reducing the optimization level, or rearranging the code in a
strategic place, will make the problem disappear. In other cases you 
will need to move to a different compiler, or to a less buggy version
(or buggy in a diferent way that doesn't bug you) of the same compiler.

\paragraph{Compilation fails at linking stage: "symbol ... not found"}
If the missing symbols (i.e. routines that are called but not found)
are in the code itself: most likely the fortran-to-C conventions used
in file \texttt{include/c\_defs.h} are not appropriate. Edit this file
and retry.

If the missing symbols are in external libraries (Blas, Lapack, FFT,
MPI libraries): 
there is a name mismatch between what the compiler expects and what the
library provides. See Sec.\ref{Sec:Installation}).

If the missing symbols aren't found anywhere either in the code or in the
libraries: they are system library symbols. i) If they are called by external 
libraries, you need to add a missing system library, or to use a different 
set of external libraries, compiled with the same compiler you are using. 
ii) If you are using no external libraries and still getting missing symbols, 
your compiler and compiler libraries are not correctly installed.

\subsection{Pseudopotentials}

\paragraph{Can I mix Ultrasoft/Norm-Conserving/PAW pseudopotentials?}

Yes, you can (if implemented, of course: a few kinds of calculations
are not available with USPP, a few more are not for PAW). A small
restrictions exists in \texttt{cp.x}, expecting atoms with USPP listed before 
those with NCPP, which in turn are expected before local PP's (if any).
Otherwise you can mix and match, as long as the XC functional used in 
the generation of the PP is the same for all PP's. Note that 
it is the hardest atom that determines the cutoff.

\paragraph{Where can I find pseudopotentials for atom X?}

If you do not find anything suitable in the ``pseudo'' page of the web 
site links, we have bad news for you: you have to produce it by yourself. 
You can use the \texttt{atomic} code, or else any other code for which
a converter to the UPF format exists.
New contributions to the PP table are very appreciated (and very scarce).
If X is one of the rare earths: please consider first if DFT is 
suitable for your system! In many cases, it isn't (at least ``plain'' 
DFT: GGA and the like).

\paragraph{Is there a converter from format XYZ to UPF?}

What is available (no warranty) is in directory \texttt{upftools/}.
You are most welcome to contribute a new converter.

\subsection{Input data}

A large percentage of the problems reported to the mailing list are
caused by incorrect input data. Before reporting a problem with
strange crashes or strange results, {\em please} have
a look at your structure with XCrySDen. XCrySDen can directly
visualise the structure from both \texttt{PWscf} input data:
\begin{verbatim}
   xcrysden --pwi "input-data-file"
\end{verbatim}
and from \texttt{PWscf} output as well:
\begin{verbatim}
   xcrysden --pwo "output-file".
\end{verbatim}
Unlike most other visualizers, XCrySDen is periodicity-aware: you can
easily visualize periodically repeated cells.
You are advised to always use XCrySDen to check your input data! 

\paragraph{Where can I find the crystal structure/atomic positions of XYZ?}

The following site contains a lot of crystal structures: 
\texttt{http://cst-www.nrl.navy.mil/lattice}.\\
"Since this seems to come up often, I'd like to point out that the
American Mineralogist Crystal Structure Database
(\texttt{http://rruff.geo.arizona.edu/AMS/amcsd}) 
is another excellent place to
find structures, though you will have to use it in conjunction with
the Bilbao crystallography server (\texttt{http://www.cryst.ehu.es}),  
and have some understanding of space groups and Wyckoff positions".

\paragraph{Where can I find the Brillouin Zone/high-symmetry
  points/irreps for XYZ?} 

"You might find this web site useful:
\texttt{http://www.cryst.ehu.es/cryst/get\_kvec.html}" (info by Cyrille
Barreteau, nov. 2007). Or else: in textbooks, such as e.g. {\em The
mathematical theory of symmetry in solids} by Bradley and Cracknell.

\paragraph{Where can I find Monkhorst-Pack grids of k-points?}

Auxiliary code \texttt{kpoints.x}, found in \texttt{pwtools/} and 
produced by \texttt{make tools}, generates uniform grids of k-points 
that are equivalent to Monkhorst-Pack grids. 

\paragraph{How do I perform a calculation with spin-orbit interactions?}

The following input variables are relevant for a spin-orbit
calculation: 
\begin{verbatim}
      noncolin=.true./.false.
      lspinorb=.true./.false.
      starting_magnetization (one for each type of atoms)
\end{verbatim}
To make a spin-orbit calculation \texttt{noncolin} must be true. 
If \texttt{starting\_magnetization} is set to zero (or not given) 
the code makes a spin-orbit calculation without spin magnetization 
(it assumes that time reversal symmetry holds and it does not calculate 
the magnetization). The states are still two-component spinors but the
total magnetization is zero. 

If \texttt{starting\_magnetization} is different from zero, it makes a non
collinear spin polarized calculation with spin-orbit interaction. The 
final spin
magnetization might be zero or different from zero depending on the
system. 

Furthermore to make a spin-orbit calculation you must use fully
relativistic pseudopotentials at least for the atoms in which you
think that spin-orbit interaction is large. If all the pseudopotentials 
are scalar
relativistic the calculation becomes equivalent to a noncolinear
calculation without spin orbit. (Andrea Dal Corso, 2007-07-27)

\paragraph{How can I choose parameters for variable-cell molecular
  dynamics?}

"A common mistake many new users make is to set the time step dt
improperly to the same order of magnitude as for CP algorithm, or
not setting dt at all. This will produce a ``not evolving dynamics''.
Good values for the original RMW (RM Wentzcovitch) dynamics are 
dt= $50 \div 70$. The choice of the cell mass is a delicate matter. An
off-optimal mass will make convergence slower. Too small masses, as
well as too long time steps, can make the algorithm unstable. A good
cell mass will make the oscillation times for internal degrees of
freedom comparable to cell degrees of freedom in non-damped
Variable-Cell MD. Test calculations are advisable before extensive
calculation. I have tested the damping algorithm that I have developed
and it has worked well so far. It allows for a much longer time step
(dt=$100 \div 150$) than the RMW one and is much more stable with very
small cell masses, which is useful when the cell shape, not the
internal degrees of freedom, is far out of equilibrium. It also
converges in a smaller number of steps than RMW." (Info from Cesar Da
Silva: the new damping algorithm is the default since v. 3.1).

\subsection{Parallel execution}

Effective usage of parallelism requires some basic knowledge on how
parallel machines work and how parallelism is implemented in
\qe. If you have no experience and no clear ideas (or not
idea at all), consider reading Sec.\ref{Sec:para}.

\paragraph{How do I choose the number of processors/how do I setup my parallel calculation?}

Please see above.

\paragraph{Why is my parallel job running in such a lousy way?}

A frequent reason for lousy parallel performances is a
conflict between MPI parallelization (implemented in \qe)
and the autoparallelizing feature of MKL libraries. Set the
environment variable \texttt{OPEN\_MP\_THREADS} to 1. 
See Sec.\ref{Sec:para} for more info.

\paragraph{Why is my parallel job crashing when reading input data / doing nothing?}

If the same data work in serial execution, use
\texttt{code -inp input\_file} instead of \texttt{code $<$ input\_file}. 
Some MPI libraries do not properly handle input redirection.

\paragraph{Why is my parallel job crashing with mysterious errors?}

Mysterious, unpredictable, erratic errors in parallel execution are
almost always coming from bugs in the compiler or/and in the MPI 
libraries and sometimes even to flacky hardware. Sorry, not our fault.

\subsection{Frequent errors during execution}

\paragraph{Why is the code saying "Wrong atomic coordinates"?}

Because they are: two or more atoms in the list of atoms have
overlapping, or anyway too close, positions. Can't you see why? look better
(or use XCrySDen: see above) and remember that the code checks periodic
images as well. 

\paragraph{The code stops with an "error reading namelist xxxx"}

Most likely there is a misspelled variable in namelist xxxx.
If there isn't any (have you looked carefully? really?? REALLY???), 
beware control characters like DOS control-M: they can confuse
the namelist-reading code. If this happens to the first namelist 
to be read (usually "\&control") in parallel execution, see above.

\paragraph{The code stops with an "error in davcio"}

Possible reasons: disk is full; \texttt{outdir} is not writable for
any reason; you changed some parameter(s) in the input (like 
\texttt{wf\_collect}, or the number of processors/pools) without 
doing a bit of cleanup in your temporary files. 

\paragraph{The code stops with a "wrong charge" error}

In most cases: you are treating a metallic system
as if it were insulating.

\subsection{Self Consistency}

\paragraph{What are the units for quantity XYZ?}

Unless otherwise specified, all \texttt{PWscf} input and output
quantities are in atomic "Rydberg" units, i.e. energies in Ry, lengths
in Bohr radii, etc.. Note that \texttt{CP} uses instead atomic "Hartree" 
units: energies in Ha, lengths in Bohr radii. 

\paragraph{Self-consistency is slow or does not converge at all}

Bad input data will often result in bad scf
convergence. Please check your structure first.
 
Assuming that your input data is sensible :
\begin{enumerate}
\item Verify if your system is metallic or is close to a metallic
  state, especially if you have few k-points. If the highest occupied
  and lowest unoccupied state(s) keep exchanging place during
  self-consistency, forget about reaching convergence. A typical sign
  of such behavior is that the self-consistency error goes down, down,
  down, than all of a sudden up again, and so on. Usually one can
  solve the problem by adding a few empty bands and a small
  broadening. 
\item Reduce \texttt{mixing\_beta} to $\sim 0.3\div
  0.1$ or smaller. Try the \texttt{mixing\_mode} value that is more
  appropriate for your problem. For slab geometries used in surface
  problems or for elongated cells,  \texttt{mixing\_mode='local-TF'}
  should be the better choice, dampening "charge sloshing". You may
  also try to increase \texttt{mixing\_ndim} to more than 8 (default
  value). Beware: this will increase the amount of memory you need. 
\item Specific to US PP: the presence of negative charge density
  regions due to either the pseudization procedure of the augmentation
  part or to truncation at finite cutoff may give convergence
  problems. Raising the \texttt{ecutrho} cutoff for charge density will
  usually help.
\end{enumerate}

\paragraph{How is the charge density (the potential, etc.) stored? 
What position in real space corresponds to an array value?}

The index of arrays used to store functions defined on 3D meshes is
actually a shorthand for three indices, following the FORTRAN convention 
("leftmost index runs faster"). An example will explain this better. 
Suppose you have a 3D array of dimension (nr1x,nr2x,nrx3), say 
\texttt{psi(nr1x,nr2x,nr3x)}. FORTRAN compilers store this array sequentially 
in the computer RAM in the following way:
\begin{verbatim}
        psi(   1,   1,   1)
        psi(   2,   1,   1)
        ...
        psi(nr1x,   1,   1)
        psi(   1,   2,   1)
        psi(   2,   2,   1)
        ...
        psi(nr1x,   2,   1)
        ...
        ...
        psi(nr1x,nr2x,   1)
        ...
        psi(nr1x,nr2x,nr3x)
etc
\end{verbatim}
Let ind be the position of the (i,j,k) element in the above list: the
relation between ind and (i,j,k) is:
\begin{verbatim}
        ind = i + (j - 1) * nr1x + (k - 1) *  nr2x * nr1x
\end{verbatim}
This should clarify the relation between 1D and 3D indexing. In real
space, the (i,j,k) point of the FFT grid with dimensions nr1, nr2, nr3
(nr1$\le$nr1x, nr2$\le$nr2x, nr3$\le$nr3x) is
$$
r_{ijk}=\frac{i-1}{nr1} \tau_1  +  \frac{j-1}{nr2} \tau_2 +
\frac{k-1}{nr3} \tau_3 
$$
where the $\tau_i$ are the basis vectors of the Bravais lattice. 
The latter are stored row-wise in the "AT" array:

$\tau_1 = $ at(:, 1), $\tau_2 = $ at(:, 2), $\tau_3 = $ at(:, 3)

The distinction between the dimensions of the FFT grid
(nr1,nr2,nr3) and the physical dimensions of the array
(nr1x,nr2x,nr3x) is done only because it is computationally
convenient in some cases that the two sets are not the same.
In particular, it is often convenient to have nrx1=nr1+1
to reduce memory conflicts.

\paragraph{What is the difference between total and absolute magnetization?}

The total magnetization is the integral of the magnetization
in the cell: 
$$
M_T = \int (n_{up}-n_{down}) d^3r.
$$
The absolute magnetization is the integral of the absolute value of
the magnetization in the cell:
$$
M_A= \int |n_{up}-n_{down}| d^3r.
$$
In a simple ferromagnetic material they should be equal (except
possibly for an overall sign)`. In simple antiferromagnets (like FeO,
NiO) $M_T$ is zero and $M_A$ is twice the magnetization of each of the
two atoms. (info by Stefano de Gironcoli) 

\paragraph{How can I calculate magnetic moments for each atom?} 

There is no 'right' way of defining the local magnetic moment
around an atom in a multi-atom system. However an approximate way to define
it is via the projected density of states on the atomic orbitals (code
projwfc.x, see example08 for its use as a postprocessing tool). This
code generate many files with the density of states projected on each
atomic wavefunction of each atom and a BIG amount of data on the
standard output, the last few lines of which contain the decomposition
of Lowdin charges on angular momentum and spin component of each atom.

\paragraph{What is the order of $Y_{lm}$ components in projected
  DOS / projection of atomic wavefunctions?}

"The order is, I think:
\begin{verbatim}
    1   $P_{0,0}(t)$
    2   $P_{1,0}(t)$
    3   $P_{1,1}(t)cos\phi$
    4   $P_{1,1}(t)sin\phi$ 
    5   $P_{2,0}(t)$
    6   $P_{2,1}(t)cos\phi$ 
    7   $P_{2,1}(t)sin\phi$
    8   $P_{2,2}(t)cos2\phi$
    9   $P_{2,2}(t)sin2\phi$
\end{verbatim}
and so on; $P_{l,m}$=Legendre Polynomials, $t = cos\theta = z/r$, 
$\phi= atan(y /x)$. No warranty. Anybody really interested in knowing 
''for sure'' which spherical harmonic combination is which should look 
into routine ylmr2 in flib/ylmr2.f90". 

\paragraph{Why is the sum of partial Lowdin charges not equal to
  the total charge?} 

"Lowdin charges (as well as other conventional atomic charges) do not
satisfy any sum rule. You can easily convince yourself that ths is the
case because the atomic orbitals that are used to calculate them are
arbitrary to some extent. If yu like, you can think that the missing
charge is "delocalized" or "bonding" charge, but this would be another
way of naming the conventional (to some extent) character of Lowdin
charge." (Stefano Baroni, Sept. 2008).  

See also the definition of "spilling parameter": Sanchez-Portal et
al., Sol. State Commun. 95, 685 (1995). The spilling parameter
measures the ability of the basis provided by the pseudo-atomic wfc to
represent the PW eigenstates, by measuring how much of the subspace of
the Hamiltonian eigenstates falls outside the subspace spanned by the
atomic basis. 

\paragraph{Why do I get a strange value of the Fermi energy?}

"The value of the Fermi energy (as well as of any energy, for that
matter) depends of the reference level. What you are referring to is
probably the "Fermi energy referred to the vacuum level" (i.e.  
the work function). In order to obtain that, you need to know what the
vacuum level is, which cannot be said from a bulk calculation only"
(Stefano Baroni, Sept. 2008). 

\paragraph{Why I don't get zero pressure/stress at equilibrium?}

It depends. If you make a calculation with fixed cell parameters, you
will never get exactly zero pressure/stress, unless you use the cell
that yields perfect equilibrium for your pseudopotentials,  cutoffs,
k-points, etc.. Such cell will anyway be slightly different from the
experimental one. Note however that pressures/stresses in the order of
a few KBar correspond to very small differences in terms of lattice parameters.

If you obtain the equilibrium cell from a variable-cell optimization,
do not forget that the pressure/stress calculated with the modified
kinetic energy functional (very useful for variable-cell calculations)
slightly differ from those calculated without it. Also note that the
plane-wave basis set used during variable-cell calculations is
determined by the given cutoff and the {\em initial} cell. If you
make a calculation with the final geometry at the same cutoff,
you may get slightly different  results. The difference should
be small, though, unless you are using a too low cutoff for your
system.

\paragraph{Why do I get "negative starting charge"?}
Self-consistency requires an initial guess for the charge density in
order to bootstrap the iterative algorithm. This first guess is
usually built from a superposition of atomic charges, constructed from
pseudopotential data. 

More often than not, this charges are a slightly too hard to be
expanded very accurately in plane waves, hence some aliasing error
will be introduced. Especially if the unit cell is big and mostly
empty, some local low negative charge density will be produced. 

''This is NOT harmful at all, the negative charge density is handled
properly by the code and will disappear during the self-consistent
cycles'', but if it is very high (let's say more than 0.001*number of
electrons) it may be a symptom that your charge density cutoff is too
low. (L. Paulatto - November 2008)

\paragraph{How do I calculate the work function?}

Work function = (average potential in the vacuum) - (Fermi
Energy). The former is estimated in a supercell with the slab
geometry, by looking at the average of the electrostatic potential
(typically without the XC part). See the example in
examples/WorkFct\_example. 

\subsection{ Phonons }

\paragraph{ Is there a simple way to determine the symmetry of a given 
phonon mode?} 

A symmetry analyzer was added in v.3.2 by Andrea Dal Corso. 
The following info may still be of interest to somebody,
though. ISOTROPY package: http://stokes.byu.edu/iso/isotropy.html.

"Please follow http://dx.doi.org/10.1016/0010-4655(94)00164-W and\\
http://dx.doi.org/10.1016/0010-4655(74)90057-5. These are connected to some 
programs found in the Computer Physics Communications Program Library 
(http://www.cpc.cs.qub.ac.uk ) which are described in the articles:\\
ACKJ v1.0 Normal coordinate analysis of crystals, J.Th.M. de Hosson.\\
ACMI v1.0 Group-theoretical analysis of lattice vibrations, T.G. Worlton, J.L. Warren. See erratum Comp. Phys. Commun. 4(1972)382.\\
ACMM v1.0 Improved version of group-theoretical analysis of lattice
dynamics, J.L. Warren, T.G. Worlton." (Info from Pascal Thibaudeau) 

\paragraph{ I am not getting zero acoustic mode frequencies, why? }

"If you treat, e.g., a molecule, the first six frequencies
should vanish. However, due to convergence (number of plane waves,
size of the supercell, etc. ) they often appear as imaginary or small
real, even if all other frequencies are converged with respect to ecut
and celldm. 

If you have a bulk structure, then imaginary frequencies indicate a 
lattice instability. However, they can appear also as a result of a 
non-converged groundstate (Ecut, k-point grid, ...).

Recently I also found that the parameters tr2\_ph for the phonons and 
conv\_thr for the groundstate can affect the quality of the phonon 
calculation, especially the "vanishing" frequencies for molecules."
(Info from Katalyn Gaal-Nagy)

\paragraph{ Why do I get negative phonon frequencies? }

If these occur for acoustic frequencies at Gamma point, see above.
If these occur for rotational modes in a molecule into a supercell: 
it is a fictitious effect of the finite supercell size. If these
occur in other cases, it depends. It may be a problem of bad
convergence (see above) or it may signal a real instability.

An example: large negative phonon frequencies in 1-dimensional chains.
"It is because probably some of atoms are sitting on the saddle points 
of the energy surface. Since QE symmetrizes charge density to avoid small 
numerical oscillation, the system cannot break the symmetry with the help 
of numerical noise. Check your system's stability by displacing one or more 
atoms a little bit along the direction of eigen-vector which has negative 
frequency. The eigen-vector can be found in the output of dynamical matrices 
of ph.x. One example here is: for 1d aluminum chain, the LO mode will be 
negative if you place two atoms at (0.0,0.0,0.0) and (0.0,0.0,0.5) of crystal 
coordinates. To break the symmetry enforced by QE code, change the second atom 
coordinate to (0.0,0.0,0.505). Relax the system. You will find the atom will 
get itself a comfortable place at (0.0,0.0,0.727), showing a typical
dimerization effect." (info by Nicola Marzari).
\end{document}
More FAQS:
- How to find E(V) for a noncubic crystal  
- absolute value of eigenvalues/Fermi energy
- negative phonons
- epsil=.true. and metals
