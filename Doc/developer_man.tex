\documentclass[12pt,a4paper]{article}
\def\version{post4.2}
\def\qe{{\sc Quantum ESPRESSO}}
\textwidth = 17cm
\textheight = 24cm
\topmargin =-1 cm
\oddsidemargin = 0 cm

\usepackage{html}

% BEWARE: don't revert from graphicx for epsfig, because latex2html
% doesn't handle epsfig commands !!!
\usepackage{graphicx}


% \def\htmladdnormallink#1#2{#1}

\def\configure{\texttt{configure}}
\def\configurac{\texttt{configure.ac}}
\def\autoconf{\texttt{autocon}}

\def\qeImage{quantum_espresso.pdf}
\def\democritosImage{democritos.pdf}

\begin{htmlonly}
\def\qeImage{quantum_espresso.png}
\def\democritosImage{democritos.png}
\end{htmlonly}

\begin{document} 
\author{}
\date{}
\title{
  \includegraphics[width=5cm]{\qeImage} \hskip 2cm
  \includegraphics[width=6cm]{\democritosImage}\\
  \vskip 1cm
  % title
  \Huge Developer's Manual for \qe (v. \version) \smallskip
}
\maketitle

\tableofcontents

\section{Introduction}
\subsection{Who should read (and who should {\em write}) this guide}

The intended audience of this guide is everybody who wants to:
\begin{itemize}
\item know how \qe\ works, including its internals;
\item modify/customize/add/extend/improve/clean up \qe;
\item know how to read data produced by \qe.
\end{itemize}
The same category of people should also {\em write} this guide, of course.

\subsection{Who may read this guide but will not necessarily profit from it}

People who want to know about the capabilities of \qe,
or who want just to use it, should read the User Guide.

People who want to know about the methods or the physics
behind \qe\ should read first the relevant  
literature (some pointers in the User Guide).

\subsection{How to contribute to \qe}

You can contribute to a better \qe\ by

\begin{itemize}
\item answering other people's questions on the mailing list (correct
  answers are strongly preferred to wrong ones). 
\item  suggesting changes: note however that suggestions requiring a
  significant amount of work are welcome only if accompanied by
  implementation or by a promise of future implementation (fulfilled
  promises are strongly preferred to forgotten ones). 
\item porting to new/unsupported architectures or configurations: see
  the Installation mechanism section. You shouldn't need new
  preprocessing flags: those already existing should be sufficient. If
  you really need a new one, look into file
  \texttt{include/defs.h.README} to learn what is there and how it is
  used. 
\item pointing out bugs in the software and in the documentation
  (reports of real bugs are strongly preferred to reports of
  nonexistent bugs). Bug reports should preferrablly be filed using
  the bug tracking facility of \texttt{qe-forge.org}. Bug reports
  should include enough information to be reproduced: typically, 
  version number, hardware/software combination(s) for which the
  problem arises, whether it is reproducible or erratic, whether
  it happens in serial or parallel
  execution or both, and, most important, an input and output
  exhibiting such behavior (fast to execute if possible). The error
  message alone is usually not a sufficient piece of information. 
\item adding new features to the code. If you like to have something
  added to \qe, contact the developers via the
 \texttt{q-e-developers} mailing list, hosted on \texttt{qe-forge.org}.
 Unless there are technical reasons not to include your changes, we 
 will try to make you happy (no warranty that we will actually succeed).
\end{itemize}
If you want to get involved as a developer and contribute serious
or nontrivial stuff, you should register for the \qe\ project on 
\texttt{qe-forge.org}. You may also consider the idea of opening
your own project on \texttt{qe-forge.org}. There is a developers' 
mailing list, \texttt{q-e-developers}, to which you should subscribe,
and a mailing list receiving commit message, \texttt{q-e-commits}.
The ideal procedure to add extensive changes is as follows:
\begin{itemize}
\item download the current CVS version (see Using CVS) and work on that version
\item when you are happy with your modified version, make a copy of
  it, then update your copy with \texttt{cvs update} 
\item if you get no conflicts and everything is still working, you
  have won. Send the modified files to the developers, or save
  your changes if you are one.
\item if you get conflicts, or if the updated code doesn't work any
  longer, you haven't yet won. Look into the conflicting section: in
  most cases conflicts are trivial (format changes, white spaces) or
  easily solved (the part of the code you were modifying has been
  moved to another place, for instance). Sometimes, somebody else has
  done changes that are incompatible with yours during the same
  period. Look into the ChangeLog (use the script \texttt{cvs2cl.pl}
  to get an updated ChangeLog) to understand what may have happened,
  or use the web-cvs interface
  (\texttt{http://qe-forge.org/cgi-bin/cvstrac/q-e/index}).
  Use \texttt{cvs update -D 2006-oct-12} to update to the CVS version 
  of 12 oct 2006: this may be useful if you suspect that the incompatibile 
  change happened after that date. CVS versions can be {\em tagged}; 
  tags are visible with \texttt{cvs status -v}. Use \texttt{cvs update -r TAG}
  to update to the CVS version with tag TAG. In all cases, use 
  \texttt{cvs update -A} to revert to the last version.
\end{itemize}

\section{ Structure of the distribution}
\subsection{Contents of the various directories}
\subsubsection{ Modules}
\subsubsection{ Sources}
\subsubsection{ Utilities}
\subsubsection{ Libraries}

Subdirectory \texttt{flib/} contains libraries written in fortran77 
(\texttt{*.f}) and in fortran-90 (\texttt{*.f90}).
The latter should not depend on any module, except for modules
\texttt{kinds} and \texttt{constants}.

Subdirectory \texttt{clib/} contains libraries written in C 
(\texttt{*.c}). Functions that are called by fortran
should be preprocessed using the macros:
\begin{enumerate}
\item \texttt{F77\_FUNC (func,FUNC)} for function \texttt{func}, not containing underscore(s) in name 
\item \texttt{F77\_FUNC\_(f\_nc,F\_NC)} for function \texttt{f\_nc}, containing underscore(s) in name
\end{enumerate}
These macros are defined in file \texttt{include/c\_defs.h}. This file must be included
by all \texttt{*.c} files. The macros are automagically generated by 
\configure\ and 
choose the correct case 
(lowercase or uppercase) and the correct number of final underscores. 
See file \texttt{include/defs.h.README} for more info.

\subsection{Installation mechanism}

\label{subsec:inst}
The code contains C-style preprocessing directives. There are two ways to do preprocessing of fortran files:
\begin{itemize}
\item directly with the fortran compiler, if supported;
\item by first pre-compiling with the C preprocessor \texttt{cpp}.
\end{itemize}

In the first case, one needs to specify in the \texttt{make.sys} file the fortran compiler option that tells the compiler to pre-process first. In the second case, one needs to
specify the C precompiler and options (if needed) in \texttt{make.sys}.
Normally, \configure\ should take care of this.


\subsubsection{ How to edit the \configure\ script}

The \configure\ script is generated from its source file
\configurac\ by the GNU \autoconf\ utility
(\texttt{http://www.gnu.org/software/autoconf/}).  Don't edit \configure\
directly: whenever it gets regenerated, your changes will be lost.
Instead, go to the \texttt{install/} directory, edit \configurac, 
then run \autoconf\ to regenerate \configure. If you want 
to keep the old \configure, make a copy
first.

GNU \autoconf\ is installed by default on most Unix/Linux systems.  If
you don't have it on your system, you'll have to install it. You will
need a recent version (e.g. v.2.65) of \autoconf, because our \configurac
file uses recent syntax.

\configurac\ is a regular Bourne shell script (i.e., "sh" -- not csh!), 
except that:
\begin{itemize}
\item[--] capitalized names starting with "AC\_" are \autoconf\ macros.  Normally you shouldn't have to touch them.
\item[--] square brackets are normally removed by the macro processor.  If you need a square bracket (that should be very rare), you'll have to write two.
\end{itemize}

You may refer to the GNU \autoconf\ Manual for more info.

\texttt{make.sys.in} is the source file for \texttt{make.sys}, that \configure
generates: you might want to edit that file as well.  The generation
procedure is as follows: if \configurac\ contains the macro
"AC\_SUBST(name)", then every occurrence of "@name@" in the source
file will be substituted with the value of the shell variable "name"
at the point where AC\_SUBST was called.

Similarly, \configure\texttt{msg} is generated from \configure\texttt{.msg.in}: this
file is only used by \configure\ to print its final report, and isn't
needed for the compilation.  We did it this way so that our
\configure\ may also be used by other projects, just by replacing the
\qe-specific \configure\texttt{.msg.in} by your own.

\configure\ writes a detailed log of its operation to "config.log".
When any configuration step fails, you may look there for the relevant
error messages.  Note that it is normal for some checks to fail.

\subsubsection{How to add support for a new architecture}

In order to support a previously unsupported architecture, first you
have to figure out which compilers, compilation flags, libraries
etc. should be used on that architecture.
In other words, you have to write a \texttt{make.sys} that works: you may use
the manual configuration procedure for that (see the 
User Guide).  Then, you have to modify \configure\ so that it can
generate that \texttt{make.sys} automatically.

To do that, you have to add the case for your architecture in several
places throughout \configurac:
\begin{enumerate}
\item Detect architecture

Look for these lines:
\begin{verbatim}
  if test "$arch" = ""
  then
          case $host in
                  ia64-*-linux-gnu )      arch=ia64   ;;
                  x86_64-*-linux-gnu )    arch=x86_64 ;;
                  *-pc-linux-gnu )        arch=ia32   ;;
                  etc.
\end{verbatim}
Here you must add an entry corresponding to your architecture and
operating system.  Run \texttt{config.guess} to obtain the string identifying
your system.
For instance on a PC it may be "i686-pc-linux-gnu", while on IBM SP4
"powerpc-ibm-aix5.1.0.0".  It is convenient to put some asterisks to
account for small variations of the string for different machines of
the same family.  For instance, it could be "aix4.3" instead of
"aix5.1", or "athlon" instead of "i686"...

\item  Select compilers

Look for these lines:

\begin{verbatim}
  # candidate compilers and flags based on architecture
  case $arch in
  ia64 | x86_64 )
        ...
  ia32 )
        ...
  aix )
        ...
  etc.
\end{verbatim}

Add an entry for your value of \$arch, and set there the appropriate
values for several variables, if needed (all variables are assigned
some reasonable default value, defined before the "case" block):

- "try\_f90" should contain the list of candidate Fortran 90 compilers,
in order of decreasing preference (i.e. configure will use the first
it finds).  If your system has parallel compilers, you should list
them in "try\_mpif90".

- "try\_ar", "try\_arflags": for these, the values "ar" and "ruv" should
be always fine, unless some special flag is required (e.g., -X64
With sp4).  

- you should define "try\_dflags" if there is
any
"\#ifdef" specific to your machine: for instance, on IBM machines,
"try\_dflags=-D\_\_AIX" . A list of such flags can be found in file 
\texttt{include/defs.h.README}.

You shouldn't need to define the following:
- "try\_iflags" should be set to the appropriate "-I" option(s)
needed by the preprocessor or by the compiler to locate *.h files
to be included; try\_iflags="-I../include" should be good for most cases

For example, here's the entry for IBM machines running AIX:
\begin{verbatim}
   aix )
        try_mpif90="mpxlf90_r mpxlf90"
        try_f90="xlf90_r xlf90 $try_f90"
        try_arflags="-X64 ruv"
        try_arflags_dynamic="-X64 ruv"
        try_dflags="-D__AIX -D__XLF"
        ;;
\end{verbatim}
The following step is to look for both serial and parallel fortran
compilers:
\begin{verbatim}
  # check serial Fortran 90 compiler...
  ...
  AC_PROG_F77($f90)
  ...
        # check parallel Fortran 90 compiler
  ...
        AC_PROG_F77($mpif90)
  ...
  echo setting F90... $f90
  echo setting MPIF90... $mpif90
\end{verbatim}
A few compilers require some extra work here: for instance, if the
Intel Fortran compiler was selected, you need to know which version
because different versions need different flags.

At the end of the test,
 
- \$mpif90 is the parallel compiler, if any; if no parallel compiler is found or if \texttt{--disable-parallel} was specified, \$mpif90 is the serial compiler

- \$f90 is the serial compiler

Next step: the choice of (serial) C and Fortran 77 compilers.
Look for these lines:
\begin{verbatim}
  # candidate C and f77 compilers good for all cases
  try_cc="cc gcc"
  try_f77="$f90"

  case "$arch:$f90" in
  *:f90 )
        ....
  etc.
\end{verbatim}
Here you have to add an entry for your architecture, and since the 
correct choice of C and f77 compilers may depend on the fortran-90 
compiler, you may need to specify the f90 compiler as well.
Again, specify the compilers in try\_cc and try\_f77 in order of 
decreasing preference.  At the end of the test, 

- \$cc is the C compiler

- \$f77 is the Fortran 77 compiler, used to compile *.f files
(may coincide with \$f90)

\item Specify compilation flags.

Look for these lines:
\begin{verbatim}
  # check Fortran compiler flags
  ...
  case "$arch:$f90" in
  ia64:ifort* | x86_64:ifort* )
        ...
  ia64:ifc* )
        ...
  etc.
\end{verbatim}
Add an entry for your case and define:

- "try\_fflags": flags for Fortran 77 compiler.

- "try\_f90flags": flags for Fortran 90 compiler.
In most cases they will be the same as in Fortran 77 plus some
others.  In that case, define them as "\$(FFLAGS) -something\_else".

- "try\_fflags\_noopt": flags for Fortran 77 with all optimizations
turned off: this is usually "-O0".
These flags must be used for compiling flib/dlamch.f (part of our
version of Lapack): it won't work properly with optimization.

- "try\_ldflags": flags for the linking phase (not including the list
of libraries: this is decided later). 

- "try\_ldflags\_static": additional flags to select static compilation
(i.e., don't use shared libraries).

- "try\_dflags": must be defined if there is in the code any \#ifdef
specific to your compiler (for instance, -D\_\_INTEL for Intel
compilers).  Define it as "\$try\_dflags -D..." so that pre-existing
flags, if any, are preserved.

- if the Fortran 90 compiler is not able to invoke the C preprocessor
automatically before compiling, set "have\_cpp=0" (the opposite case
is the default). The appropriate compilation rules will be generated 
accordingly. If the compiler requires that any flags be specified in
order to invoke the preprocessor (for example, "-fpp " -- note the 
space), specify them in "pre\_fdflags".

For example, here's the entry for ifort on Linux PC:
\begin{verbatim}
  ia32:ifort* )
          try_fflags="-O2 -tpp6 -assume byterecl"
          try_f90flags="\$(FFLAGS) -nomodule"
          try_fflags_noopt="-O0 -assume byterecl"
          try_ldflags=""
          try_ldflags_static="-static"
          try_dflags="$try_dflags -D__INTEL"
          pre_fdflags="-fpp "
          ;;
\end{verbatim}
Next step: flags for the C compiler. Look for these lines:
\begin{verbatim}
  case "$arch:$cc" in
  *:icc )
        ...
  *:pgcc )
        ...
  etc.
\end{verbatim}
Add an entry for your case and define:

- "try\_cflags": flags for C compiler.

- "c\_ldflags": flags for linking, when using the C compiler as linker.
This is needed to check for libraries written in C, such as FFTW.

- if you need a different preprocessor from the standard one (\$CC -E),
define it in "try\_cpp".

For example for XLC on AIX:
\begin{verbatim}
  aix:mpcc* | aix:xlc* | aix:cc )
          try_cflags="-q64 -O2"
          c_ldflags="-q64"
          ;;
\end{verbatim}
Finally, if you have to use a nonstandard preprocessor, look for these
lines:
\begin{verbatim}
  echo $ECHO_N "setting CPPFLAGS... $ECHO_C"
  case $cpp in
        cpp) try_cppflags="-P -traditional" ;;
        fpp) try_cppflags="-P"              ;;
        ...
\end{verbatim}
and set "try\_cppflags" as appropriate.

\item Search for libraries

To instruct \configure\ to search for libraries, you must tell it two
things: the names of libraries it should search for, and where it
should search.

The following libraries are searched for:

- BLAS or equivalent. 
Some vendor replacements for BLAS that are supported by \qe\ are:
\begin{quote}
    MKL on Linux, 32- and 64-bit Intel CPUs\\
    ACML on Linux, 64-bit AMD CPUs\\
    essl on AIX\\
    SCSL on sgi altix\\
    SUNperf on sparc
\end{quote}
Moreover, ATLAS is used over BLAS if available.

- LAPACK or equivalent. Some vendor replacements for LAPACK that are supported by \qe\ are:
\begin{quote}
    mkl on linux
    SUNperf on sparc
\end{quote}

- FFTW (version 3) or another supported FFT library. The latter include:
\begin{quote}
    essl on aix
    ACML on Linux, 64-bit AMD CPUs
    SUNperf on sparc
\end{quote}

- the MASS vector math library on aix

- an MPI library. This is often automatically linked by the compiler

If you have another replacement for the above libraries, you'll have
to insert a new entry in the appropriate place.

This is unfortunately a little bit too complex to explain.
Basic info: "AC\_SEARCH\_LIBS(function, name, ...)" looks for symbol
"function" in library "libname".  If that is found, "-lname" is
appended to the LIBS environment variable (initially empty).
The real thing is more complicated than just that because the
"-Ldirectory" option must be added to search in a nonstandard
directory, and because a given library may require other libraries as
prerequisites (for example, Lapack requires BLAS).
\end{enumerate}

\subsection{Adding new directories or routines}

\section{ Algorithms}
\subsection{Diagonalization}
\subsection{Self-consistency}
\subsection{Structural optimization}
\subsection{Symmetrization}
\subsection{Gamma tricks}

In calculations using only the $\Gamma$ point (k=0),
the Kohn-Sham orbitals can be chosen to be real functions in 
real space, so that 
$
  \psi(G) = \psi^*(-G).
$
This allows us to store only half of the Fourier components.
Moreover, two real FFTs can be performed as a single complex FFT.
The auxiliary complex function $\Phi$ is introduced:
$
    \Phi(r) = \psi_j(r)+ i \psi_{j+1}(r)
$
whose Fourier transform $\Phi(G)$ yields

$
   \psi_j    (G) =  {\Phi(G) + \Phi^*(G)\over 2},
   \psi_{j+1}(G) =  {\Phi(G) - \Phi^*(G)\over 2i}.
$

A side effect on parallelization is that $G$ and $-G$ must
reside on the same processor. As a consequence, pairs of columns
with $G_{n'_1,n'_2,n'_3}$ and $G_{-n'_1,-n'_2,n'_3}$
(with the exception of the case $n'_1=n'_2=0$),
must be assigned to the same processor.

\section{ Structure of the code}
\subsection{Modules and global variables}
\subsection{Meaning of the most important variables}
\subsection{Conventions for indices}
\subsection{Preprocessing}

The code contains C-style preprocessing directives. Most fortran compilers directly support them; some don't, and preprocessing is ''hand-made'' by the makefile using the C preprocessor \texttt{cpp}. The C preprocessor may:
\begin{itemize}
\item assign a value to a given expression. For instance, command \texttt{\#define THIS that}, or the option in the command line: \texttt{-DTHIS=that}, will replace all occurrence of \texttt{THIS} with \texttt{that}.
\item include file (command \texttt{\#include})
\item expand macros (command \texttt{\#define})
\item execute conditional expressions such as
\begin{verbatim}
  #ifdef __expression
    ...code A...
  #else
    ...code B...
  #endif
\end{verbatim}
If ''expression'' is defined (with a \texttt{\#define} command 
or from the command line with option \texttt{\-D\_\_expression}), 
then  \texttt{...code A...} is sent to output; otherwise 
\texttt{...code B...} is sent to output.

\end{itemize}
The file \texttt{include/defs.h.README} contains a list of definitions that are used in the code. In order to make  preprocessing options easy to see, preprocessing variables should start with 
two underscores, as \texttt{\_\_expression} in the above example. Traditionally ''preprocessed'' variables are also written in uppercase.

\subsection{Performance issues}
\subsection{Portability issues}

\section{ Parallelization}

In parallel execution, PW starts N independent processes (do not start more than one per processor!) that communicate via calls to MPI libraries. Each process has its own set of variables and knows nothing about other processes' variables. Variables that take little memory are replicated, those that take a lot of memory (wavefunctions, G-vectors, R-space grid) are distributed.
    
Beware: replicated calculations may either be performed independently on each processor, or performed on one processor and broadcast to all
others. The first approach requires less programming, but it is unsafe: in principle all processors should yield exactly the same results, if they work on the same data, but sometimes they don't (depending on the machine, compiler, and libraries). Even a tiny difference in the last significant digit can eventually cause serious trouble if allowed to build up, especially when a replicated check is performed (in which
case the code may ''hang'' if the check yields different results on different processors). Never assume that the value of a variable produced by replicated calculations is exactly the same on all processors: when in doubt, broadcast the value calculated on a specific processor (the ''root'' processor) to all others.

\subsection{Paradigms}
\subsection{Implementation}
\subsubsection{ Data distribution}

Quantum ESPRESSO employ arrays whose memory requirements fall 
into three categories.
\begin{itemize}
\item {\em Fully Scalable}: 
Arrays that are distributed across processors of a pool.
Fully scalable arrays are typically large to very large and contain one 
of the following dimensions:
\begin{itemize}
\item number of plane waves, npw (or max number, npwx)
\item number of Gvectors, ngm
\item number of grid points in the R space, nrxx
\end{itemize}
Their size decreases linearly with the number of processors in a pool. 

\item {\em Partially Scalable}: 
Arrays that are distributed across processors of the
ortho or diag group. Typically they are much smaller than fully scalable
array, and small in absolute terms for moderate-size system. Their size
however increases quadratically with the number of atoms in the system,
so they have to be distributed for large systems (hundreds to thousands
atoms). Partially scalable arrays contain none of the dimensions listed 
above, two of the following dimensions:
\begin{itemize}
\item number of states, nbnd
\item number of projectors, nkb
\end{itemize}
Their size decreases linearly with the number of processors in a ortho
or diag group. 

\item
{\em Nonscalable}: All the remaining arrays, that are not distributed across
processors. These are typically small arrays, having dimensions like for
instance:
\begin{itemize}
\item number of atoms, nat
\item number of species of atoms, nsp
\end{itemize}
The size of these arrays is independent on the number of processors.
\end{itemize}

\subsubsection{ Parallel fft}

\section{ File Formats}
\subsection{Data file(s)}

\qe\ restart file specifications:
Paolo Giannozzi scripsit AD 2005-11-11,
Last modified by Andrea Ferretti 2006-10-29

\subsubsection{Rationale}

Requirements: the data file should be
\begin{itemize}
\item efficient (quick to read and write)
\item easy to read, parse and write without special libraries
\item easy to understand (self-documented)
\item portable across different software packages
\item portable across different computer architectures 
\end{itemize}
Solutions:
\begin{itemize}
\item use binary I/O for large records
\item exploit the file system for organizing data
\item use XML
\item use a small specialized library (iotk) to read, parse, write 
\item ensure the possibility to convert to a portable formatted file
\end{itemize}
Integration with other packages:
\begin{itemize}
\item provide a self-standing (code-independent) library to read/write this format
\item the use of this library is intended to be at high level, hiding low-level details
\end{itemize}

\subsubsection{General structure}

Format name: '''QEXML''' \\
Format version: '''1.4.0''' \\

The "restart file" is actually a "restart directory", containing several files and sub-directories. 
For CP/FPMD, the restart directory is created as "\$prefix\_\$ndw/", where \$prefix is the value of the 
variable "prefix". \$ndw the value of variable ndw, both read in input; it is read from "\$prefix\_\$ndr/", 
where \$ndr the value of variable ndr, read from input.
For PWscf, both input and output directories are called 
"\$prefix.save/".

The content of the restart directory is as follows:
\begin{verbatim}
    ''data-file.xml''          which contains:
                               - general information that doesn't require large data set: 
                                 atomic structure, lattice, symmetries, parameters of the run, ...
                               - pointers to other files or directories containing bulkier data:
                                 such as grids, wavefunctions, charge density, potentials, ...
      
    ''charge_density.dat''     contains the charge density
    ''spin_polarization.dat''  contains the spin polarization (rhoup-rhodw) (LSDA calculations)
    ''magnetization.x.dat''    
    ''magnetization.y.dat''    contain the spin polarization along x,y,z (noncollinear calculations)  
    ''magnetization.z.dat'' 
    ''lambda.dat''             contains occupations (Car-Parrinello dynamics only
    ''mat_z.1''                contains occupations (ensemble-dynamics only)
    
    <pseudopotentials>     A copy of all pseudopotential files given in input
    
    <k-point dirs>         One or more subdirectories ''K00001/'', ''K00002/'', etc, one per k-point.
\end{verbatim}
Each k-point directory contains:
\begin{verbatim}
    ''evc.dat''                containing the wavefunctions for spin-unpolarized calculations, OR
    ''evc1.dat''
    ''evc2.dat''               containing the spin-up and spin-down wavefunctions, respectively, 
                           for spin polarized (LSDA) calculations;
  
                           in a molecular dynamics run, also wavefunctions at the preceding time step:
    ''evcm.dat''               for spin-unpolirized calculations OR
    ''evcm1.dat''
    ''evcm2.dat''              for spin polarized calculations;
     
    ''gkvectors.dat''          with the details of specific k+G grid;
    ''eigenval.xml''           containing the eigenvalues for the corresponding k-point for spin-unpolarized calculations, OR
    ''eigenval1.xml''      
    ''eigenval2.xml''          for spin-polarized calculations;
\end{verbatim}

\begin{itemize}
\item All files ''*.xml'' are XML-compliant, formatted file;
\item Files ''mat\_z.1'', ''lambda.dat'' are unformatted files, containing a single record;
\item All other files ''*.dat'', are XML-compliant files, but they contain an unformatted record.
\end{itemize}

\subsubsection{ Structure of file "data-file.xml"}
\begin{verbatim}
* ''XML Header'': whatever is needed to have a well-formed XML file

* ''Body'': introduced by <Root>, terminated by </Root>. Contains first-level tags only. These contain only other tags, not values. XML syntax applies.

* ''First-level tags'': contain either
** second-level tags
** "data tags": tags containing data (values for a given variable)
** "file tags": tags pointing to a file

''data tags syntax'' ( [...] = optional ) :
      <TAG type="vartype" size="n" [UNIT="units"] [LEN="k"]>
      values (in appropriate units) for variable corresponding to TAG:
      n elements of type vartype (if character, of lenght k)
      </TAG>
where TAG describes the variable into which data must be read;<br>
"vartype" may be "integer", "real", "character", "logical";<br>
if type="logical", LEN=k" must be used to specify the length
of the variable character; size="n" is the dimension.<br>
Acceptable values for "units" depend on the specific tag.

''Short syntax'', used only in a few cases:
      <TAG attribute="something"/> . 
For instance:
      <FFT_GRID nr1="NR1" nr2="NR2" nr3="NR3"/>
defines the value of the FFT grid parameters nr1, nr2, nr3
for the charge density
\end{verbatim}

\subsubsection{Sample}
\begin{verbatim}
* Header:

 <?xml version="1.0"?>
 <?iotk version="1.0.0test"?>
 <?iotk file_version="1.0"?>
 <?iotk binary="F"?> 

These are meant to be used only by iotk (actually they aren't)

* First-level tags:

  - <HEADER>         (global information about fmt version)
  - <CONTROL>        (miscellanea of internal information)
  - <STATUS>         (information about the status of the CP simulation)
  - <CELL>           (lattice vector, unit cell, etc)
  - <IONS>           (type and positions of atoms in the unit cell etc)
  - <SYMMETRIES>     (symmetry operations)
  - <ELECTRIC_FIELD> (details for an eventual applied electric field)
  - <PLANE_WAVES>    (basis set, cutoffs etc)
  - <SPIN>           (info on spin polarizaztion)
  - <MAGNETIZATION_INIT>     (info about starting or constrained magnetization)
  - <EXCHANGE_CORRELATION>
  - <OCCUPATIONS>    (occupancy of the states)
  - <BRILLOUIN_ZONE> (k-points etc)
  - <PHONON>         (info for phonon calculations)  
  - <PARALLELISM>    (specialized info for parallel runs)
  - <CHARGE-DENSITY>
  - <TIMESTEPS>      (positions, velocities, nose' thermostats)
  - <BAND_STRUCTURE_INFO>    (dimensions and basic data about band structure)
  - <EIGENVALUES>    (eigenvalues and related data)
  - <EIGENVECTORS>   (eigenvectors and related data)

  
* Tag description

  <HEADER> 
     <FORMAT>    (name and version of the format)
     <CREATOR>   (name and version of the code generating the file)
  </HEADER>

  <CONTROL>
     <PP_CHECK_FLAG>    (whether the file can be used for post-processing)
     <LKPOINT_DIR>      (whether kpt-data are written in sub-directories)
     <Q_REAL_SPACE>     (whether augmentation terms are used in real space)
  </CONTROL>

  <STATUS>  (optional)
     <STEP>   (number $n of steps performed, i.e. we are at step $n)
     <TIME>   (total simulation time)
     <TITLE>  (a job descriptor)
     <ekin>   (kinetic energy)
     <eht>    (hartree energy)
     <esr>    (Ewald term, real-space contribution)
     <eself>  (self-interaction of the Gaussians)
     <epseu>  (pseudopotential energy, local)
     <enl>    (pseudopotential energy, nonlocal)
     <exc>    (exchange-correlation energy)
     <vave>   (average of the potential)
     <enthal> (enthalpy: E+PV)
  </STATUS>

  <CELL>
     <BRAVAIS_LATTICE>
     <LATTICE_PARAMETER>
     <CELL_DIMENSIONS>  (cell parameters)
     <DIRECT_LATTICE_VECTORS>
        <UNITS_FOR_DIRECT_LATTICE_VECTORS>
        <a1>
        <a2>
        <a3>
     <RECIPROCAL_LATTICE_VECTORS>
        <UNITS_FOR_RECIPROCAL_LATTICE_VECTORS>
        <b1>
        <b2>
        <b3>
  </CELL>

  <IONS>
     <NUMBER_OF_ATOMS>
     <NUMBER_OF_SPECIES>
     <UNITS_FOR_ATOMIC_MASSES>
     For each $n-th species $X:
        <SPECIE.$n>
           <ATOM_TYPE>
           <MASS>
           <PSEUDO>
        </SPECIE.$n>
     <PSEUDO_DIR>
     <UNITS_FOR_ATOMIC_POSITIONS>
     For each atom $n of species $X:
        <ATOM.$n SPECIES="$X">
  </IONS>

  <SYMMETRIES>
     <NUMBER_OF_SYMMETRIES>
     <INVERSION_SYMMETRY>
     <NUMBER_OF_ATOMS>
     <UNITS_FOR_SYMMETRIES>
     For each symmetry $n:
        <SYMM.$n>
           <INFO>
           <ROTATION>
           <FRACTIONAL_TRANSLATION>
           <EQUIVALENT_IONS>
        </SYMM.$n>
  </SYMMETRIES>

  <ELECTRIC_FIELD>  (optional)
     <HAS_ELECTRIC_FIELD> 
     <HAS_DIPOLE_CORRECTION>
     <FIELD_DIRECTION>
     <MAXIMUM_POSITION>
     <INVERSE_REGION>
     <FIELD_AMPLITUDE>
  </ELECTRIC_FIELD>  

  <PLANE_WAVES>
     <UNITS_FOR_CUTOFF>
     <WFC_CUTOFF>
     <RHO_CUTOFF>
     <MAX_NUMBER_OF_GK-VECTORS>
     <GAMMA_ONLY>
     <FFT_GRID>
     <GVECT_NUMBER>
     <SMOOTH_FFT_GRID>
     <SMOOTH_GVECT_NUMBER>
     <G-VECTORS_FILE>       link to file "gvectors.dat"
     <SMALLBOX_FFT_GRID>
  </PLANE_WAVES>

  <SPIN>
     <LSDA>
     <NON-COLINEAR_CALCULATION>
     <SPIN-ORBIT_CALCULATION>
     <SPIN-ORBIT_DOMAG>
  </SPIN>

  <EXCHANGE_CORRELATION>
     <DFT>
     <LDA_PLUS_U_CALCULATION>
     if LDA_PLUS_U_CALCULATION
        <NUMBER_OF_SPECIES>
        <HUBBARD_LMAX>
        <HUBBARD_L>
        <HUBBARD_U>
        <HUBBARD_ALPHA>
     endif
  </EXCHANGE_CORRELATION>

  if hybrid functional
      <EXACT_EXCHANGE>
        <x_gamma_extrapolation>
        <nqx1>
        <nqx2>
        <nqx3>
        <exxdiv_treatment>
        <yukawa>
        <ecutvcut>
        <exx_fraction>
        <screening_parameter>
      </EXACT_EXCHANGE>
  endif 

  <OCCUPATIONS>
     <SMEARING_METHOD>
     if gaussian smearing
        <SMEARING_TYPE>
        <SMEARING_PARAMETER>
     endif
     <TETRAHEDRON_METHOD>
     if use tetrahedra
        <NUMBER_OF_TETRAHEDRA>
        for each tetrahedron $t
           <TETRAHEDRON.$t>
     endif
     <FIXED_OCCUPATIONS>
     if using fixed occupations
        <INFO>
        <INPUT_OCC_UP>
        if lsda
           <INPUT_OCC_DOWN>
        endif
     endif
  </OCCUPATIONS>

  <BRILLOUIN_ZONE>
     <NUMBER_OF_K-POINTS>
     <UNITS_FOR_K-POINTS>
     <MONKHORST_PACK_GRID>
     <MONKHORST_PACK_OFFSET>
     For each k-point $n:
        <K-POINT.$n>
  </BRILLOUIN_ZONE>

  <PHONON> 
     <NUMBER_OF_MODES>
     <UNITS_FOR_Q-POINT>
     <Q-POINT>
  </PHONON>

  <PARALLELISM>
     <GRANULARITY_OF_K-POINTS_DISTRIBUTION>
  </PARALLELISM>

  <CHARGE-DENSITY>
      link to file "charge_density.rho"
  </CHARGE-DENSITY>

  <TIMESTEPS>  (optional)
     For each time step $n=0,M
       <STEP$n>
          <ACCUMULATORS>
          <IONS_POSITIONS>
             <stau>
             <svel>
             <taui>
             <cdmi>
             <force>
          <IONS_NOSE>
             <nhpcl>
             <nhpdim>
             <xnhp>
             <vnhp>
          <ekincm>
          <ELECTRONS_NOSE>
             <xnhe>
             <vnhe>
          <CELL_PARAMETERS>
             <ht>
             <htve>
             <gvel>
          <CELL_NOSE>
             <xnhh>
             <vnhh>
          </CELL_NOSE>
  </TIMESTEPS>

  <BAND_STRUCTURE_INFO>
     <NUMBER_OF_BANDS>
     <NUMBER_OF_K-POINTS>
     <NUMBER_OF_SPIN_COMPONENTS>
     <NON-COLINEAR_CALCULATION>
     <NUMBER_OF_ATOMIC_WFC>
     <NUMBER_OF_ELECTRONS>
     <UNITS_FOR_K-POINTS>
     <UNITS_FOR_ENERGIES>
     <FERMI_ENERGY>
  </BAND_STRUCTURE_INFO>

  <EIGENVALUES>
     For all kpoint $n:
         <K-POINT.$n>
             <K-POINT_COORDS>
             <WEIGHT>
             <DATAFILE>                  link to file "./K$n/eigenval.xml"
         </K-POINT.$n>
  </EIGENVALUES>

  <EIGENVECTORS>
     <MAX_NUMBER_OF_GK-VECTORS>
     For all kpoint $n:
         <K-POINT.$n>
             <NUMBER_OF_GK-VECTORS>
             <GK-VECTORS>                link to file "./K$n/gkvectors.dat"
             for all spin $s
                <WFC.$s>                 link to file "./K$n/evc.dat"
                <WFCM.$s>                link to file "./K$n/evcm.dat" (optional)
                                         containing wavefunctions at preceding step
         </K-POINT.$n>
  </EIGENVECTORS>
\end{verbatim}

\subsection{Restart files}

\section{ Modifying/adding/extending \qe}
\subsection{Hints, Caveats, Do's and Dont's}

\begin{itemize}
\item Before doing anything, inquire whether it is already there, or under development.
\item Before starting writing code, inquire whether you can reuse code that is already available in the distribution. Avoid redundancy: the only bug-free software line is the one that doesn't exist.
\item When you make some change:
\begin{itemize}
\item Check that are not spoiling other people's work. In particular, search the distribution for codes using the routine or module you are modifying and change its usage or its calling sequence everywhere.
\item Do not forget to add/update documentation and examples as well.
\item Do not forget that your change must work on many different combinations of hardware and software, in both serial and parallel execution.
\end{itemize}
\item Please do not include files with DOS \^M characters or 
tabulators \^I.
\end{itemize}
Important: when you modify the program sources, run the
\texttt{install/makedeps.sh}  script  or type \texttt{make depend} 
to update files \texttt{make.depend} in the various 
subdirectories.

\subsection{Programming style (or lack of it)}

Guidelines for developers:
\begin{itemize}
\item preprocessing options should be capitalized and start with two underscores. Examples: \_\_AIX, \_\_LINUX, ...
\item fortran commands should be capitalized: 
CALL something( )
\item variable names should be lowercase: \texttt{foo = bar/2}
\item indent DO's and IF's with three white spaces (editors like emacs will do this automatically for you)
\item do not write crammed code: leave spaces, insert empty separation lines
\item comments (introduced by a !) should be used to explain what is not obvious from the code, not to repeat what is already evident. Obscure comments serve no purpose.
\item do not use machine-dependent extensions or sloppy syntax. Standard f90 requires that a \& is needed both at end of line AND at the beginning of continuation line if there is a ' ' or " " spanning two lines. Some compilers do not complain if the latter \& is missing, others do.
\item use DP (defined in module ''kinds'') to define the type of real and complex variables
\item all constants should be defined to be of kind DP.  Preferred syntax: 0.0\_dp.
\item use "generic" intrinsic functions: SIN, COS, etc.
\item conversions should be explicitely indicated. For conversions to real, 
use DBLE, or else REAL(...,KIND=dp). For conversions to complex, use 
CMPLX(...,...,KIND=dp). For complex conjugate, use CONJG.  For imaginary part, 
use AIMAG.  IMPORTANT: Do not use REAL or CMPLX without KIND=dp, or else you 
will lose precision (except when you take the real part of a 
double precision complex number).
\end{itemize}

\subsection{Adding or modifying input variables}

New input variables should be added to 
''Modules/input\_parameters.f90'',
then copied to the code internal variables in the ''input.f90''
subroutine. The namelists and cards parsers are in :
''Modules/read\_namelists.f90'' and ''Modules/read\_cards.f90''.
Files ''input\_parameters.f90'', ''read\_namelists.f90'',
''read\_cards.f90'' are shared by all codes, while each code
has its own version of ''input.f90''  used to copy input values
into internal variables

EXAMPLE:
suppose you need to add a new input variable called ''pippo''
to the namelist control, then:

\begin{enumerate}
\item add pippo to the input\_parameters.f90 file containing the
namelist control 
\begin{verbatim}
              INTEGER :: pippo = 0
              NAMELIST / control / ....., pippo
\end{verbatim}
Remember: always set an initial value!

\item add pippo to the control\_default subroutine (contained in
module read\_namelists.f90 ) 
\begin{verbatim}
               subroutine control_default( prog )
              ...
              IF( prog == 'PW' ) pippo = 10
              ...
              end subroutine
\end{verbatim}
This routine sets the default value for pippo (can be different in
different codes) 

\item add pippo to the control\_bcast subroutine (contained in module
read\_namelists.f90 ) 
 \begin{verbatim}
                subroutine control_bcast( )
                ...
                call mp_bcast( pippo )
                ...
                end subroutine
\end{verbatim}
\end{enumerate}
 
\section{ Using CVS}

The package is available read-only using anonymous CVS. Developers may
have read-write access if needed. Note that the latest (development)
version may not work properly, and sometimes not even compile
properly. Use at your own risk. 

CVS (Concurrent Version System) is a software that allows many
developers to work and maintain a single copy of a software in a
central location (repository). It is installed by default on most Unix
machines, or otherwise it can be very easily installed: see
http://www.cvshome.org. For a tutorial, see: 
http://www.loria.fr/~molli/cvs/cvs-tut/cvs\_tutorial\_toc.html .
You will also need a working installation of ssh (secure shell) to use CVS.

\subsection{Anonymous CVS}

You have to define the environment variables CVS\_RSH and CVSROOT.
For csh/tcsh, use
\begin{verbatim}
setenv CVS_RSH ssh
setenv CVSROOT :pserver:anonymous@scm.qe-forge.org:/cvsroot/q-e
\end{verbatim}
For sh/ksh/bash, use
\begin{verbatim}
export CVS_RSH=ssh
export CVSROOT=:pserver:anonymous@scm.qe-forge.org:/cvsroot/q-e 
\end{verbatim}
Then:
\begin{verbatim}
cvs login
\end{verbatim}
Do not specify any password, just press "Enter".

\subsection{Read/Write CVS}

The environment variable CVS\_RSH is defined as above, but CVSROOT is
set to a different value. For csh/tcsh, use
\begin{verbatim}
setenv CVS_RSH ssh
setenv CVSROOT :ext:your-account@scm.qe-forge.org:/cvsroot/q-e 
\end{verbatim}
For sh/ksh/bash use
\begin{verbatim}
export CVS_RSH=ssh
export CVSROOT=:ext:your-account@scm.qe-forge.org:/cvsroot/q-e 
\end{verbatim}
You need to have an account on http://www.qe-forge.org
(''your-account'' above) with provileges as Quantum ESPRESSO developer.
You will be prompted for your password at each cvs operation.

\subsection{CVS operations}

For the first code download:
\begin{verbatim}
cvs checkoout -P espresso
\end{verbatim}
The code appears in directory ''espresso/''. To update the code to the
current version: 
\begin{verbatim}
cvs update -dP
\end{verbatim}
in the directory containing the distribution. Option ''-d'' ensures
that newly added directories are downloaded, ''-P'' that empty
directories are removed.  It is possible to download the version at a
given date, or corresponding to  a given ''tag''  (set by the
developers, usually just before extensive changes or at public
releases). 

When updating, you should get lines looking like
\begin{verbatim}
cvs server: Updating Modules
P Modules/Makefile
U Modules/control_flags.f90
\end{verbatim}
where P means ''patched'', U means ''updated'' (a new file is added); or
\begin{verbatim}
M PW/Makefile
\end{verbatim}
where M means ''locally modified'' (i.e., wrt the CVS repository
version); or
\begin{verbatim}
cvs server: FPMD/control.f90 is no longer in the repository
\end{verbatim}
if a file has been meanwhile deleted, or moved, or renamed; but no lines like
\begin{verbatim}
C Somedir/SomeFile
cvs server: conflict while updating Somedir/SomeFile
\end{verbatim}
This means that somebody else has modified the same parts of the code
that you have locally modified; conflicting files will contain lines
like 
\begin{verbatim}
>>>>>>>>>>>>>
   something
=============
   something else
<<<<<<<<<<<<<
\end{verbatim}
If this happen, you must edit the file manually to remove conflicts.

You can compare your local copy with any version of the repository
using
\begin{verbatim}
cvs diff -r tag
\end{verbatim}
or with the repository at a given date using
\begin{verbatim}
cvs diff -D date
\end{verbatim}
You can also compare two different versions/snapshots of the
repository, by specifying two different tags or dates. Options ''-i''
(ignore case), ''-w'' (ignore all white spaces), ''-b'' (ignore
changes in the number of white spaces) may be useful to distinguish
true changes from esthetic ones (such as changes in indentation). 

READ-WRITE ACCESS ONLY:

In order to save your changes to the repository, use ''cvs
commit''. In order to add a file, first use ''cvs add'', then ''cvs
commit''. In order to delete a file, first use ''cvs delete'', then
''cvs commit''. In order to rename a file, delete the file with the
old name, add the file with the new name. 

In order to add a new directory (let us say ''dir/'', and if you have
the permission to do so: 
\begin{itemize}
\item create directory''dir/''; do ''cvs add dir'' (this will create
  the CVS subdirectory in the new directory ''dir/'') 
\item copy all files into ''dir/'', then from inside ''dir/'' add
  files: ''cvs add *.f90 Makefile'' (for instance) 
\item ''cvs commit'' will save the new directory
\end{itemize}

\subsection{CVS hints}

It is convenient to create a \texttt{.cvsrc} file in your
home directory containing the following lines:
\begin{verbatim}
cvs -z3
diff -wib
update -dP
checkout -P
\end{verbatim}
These options will be automatically used.

You can use the same repository read-only with anonymous cvs,
read-write with your qe-forge account. Download your repository 
using anonymous cvs as explained above; then re-define the
\texttt{\$CVSROOT} environment variable to point to the
appropriate value for read-write access. You will keep
using anonymous cvs because cvs will look for the contents of
file CVS/Root, but if you use syntax:
\begin{verbatim}
cvs -d$CVSROOT commit ...
\end{verbatim}
you can perform write operations (you will be prompted for your 
qe-forge password). In this way you will need to type 
your password only once in a while.

You will find very convenient to use the Web-CVS interface:
http://qe-forge.org/cgi-bin/cvstrac/q-e/index, in particular the 
"Timeline" and "Browse" options, useful to follow what
has been done and what is going on in the development version.

\section{bibliography}

Fortran books:
\begin{itemize}
\item 
M. Metcalf, J. Reid, Fortran 95/2003 Explained, Oxford University Press (2004) 
\item
S. J. Chapman, Fortran 95/2003 for Scientists and Engineers, McGraw Hill (2007) 
\item
J. C. Adams, W. S. Brainerd, R. A. Hendrickson, R. E. Maine, J. T. Martin,
B. T. Smith, The Fortran 2003 Handbook, Springer (2009) 
\item
W. S. Brainerd, Guide to Fortran 2003 Programming, Springer (2009)
\end{itemize}
On-line tutorials:
\begin{itemize}
\item Fortran:
http://www.cs.mtu.edu/\~{}shene/COURSES/cs201/NOTES/fortran.html
\item Make:  
http://en.wikipedia.org/wiki/Make\_(software)
\item Configure script:
http://en.wikipedia.org/wiki/Configure\_script
\end{itemize}
(info courtesy of Goranka Bilalbegovic)
\end{document}
