\documentclass[a4paper,twocolumn,12pt]{article}
\usepackage{times}
\usepackage{mathptm}
\usepackage{rotating}
\usepackage{amsmath}
\usepackage{amsfonts}
\newcommand{\var}[1]{{\tt #1}}
\renewcommand{\phi}{\varphi}
\renewcommand{\theta}{\vartheta}

\title{PAW in \var{pw.x} using regular grids\\for computing one-center
integrals\\WORKING NOTES, USE WITH CARE}

\author{Guido Fratesi and Riccardo Mazzarello}

\date{Last modified: April 26, 2006}

% 21cm = 2x\hoffset + \textwidth
\setlength{\hoffset}{-1in}\addtolength{\hoffset}{1.6cm} %GF>
\setlength{\textwidth}{17.8cm} %GF>
\setlength{\columnsep}{.8cm} %GF>
% A4
% 29.7cm = \topmargin + \textheight + \footskip + bottom margin
\setlength{\voffset}{-1in} %<GF>
\setlength{\topmargin}{1.5cm} %<GF>
\setlength{\headheight}{0cm} %<GF>
\setlength{\headsep}{0cm} %<GF>
\setlength{\textheight}{26cm} %<GF>
\setlength{\footskip}{0.7cm} %<GF>

\begin{document}
\maketitle

\section*{Current status}

\var{ld1.x} can generate PAW dataset.  Two modules have been added to
perform PAW calculations with \var{pw.x}, with Kresse's notations
adopted throughout: \var{grid\_paw\_variables} and
\var{grid\_paw\_routines} mainly contain clones of variables and
routines already present in \var{pw.x} at Oct. 4, 2005 (when code was
branched with tag \var{develop\_PAW}), slightly adapted for doing
similar things on one-center densities rather than US one
(atom-by-atom, separately).

Simple tests have been performed for magnetic and non-magnetic O and O$_2$,
obtaining agreement with reference NC calculations within 1mRyd (see
\var{examples/PAWexample}), the origin of this small discrepancy yet
to be investigated.

\section*{Next steps to do}

Numerics on grids: Check the 1mRyd accuracy; Beware that \var{alpha}
parameter for gaussian charges added and removed is fixed.

Compensation functions currently reproduce only zeroth moments.  This
should be fixed, by splitting them in separate $L$ channels, see PRB
{\bf 59}, 1758 (1999), Eq.~(27).

Compensation functions should be allowed to not mimic the true
augmentation functions.  Currently, \var{ld1.x} can generate PAW with
pure gaussian compensation functions, but right now the case $l\neq0$
is not properly done because in this case one should multiply the
gaussian times some $r^l$ or similar---otherwise the function is
ill-defined in the origin.

After that, we should proceed substituting routines using radial grids
(or better, analytic summations) for routines using regular (hard FFT)
grid.

\section*{Introduction}

\noindent{\bf Disclaimer:} One-center integrals should be
eventually evaluated on radial grids. However, as a first step, the use of
regular grids should enable one to test PAW implementation
introducing the smallest amount of new code.

\begin{table*}
%\begin{center}
\hspace{-1.2cm}
\begin{small}
\begin{tabular}{|c@{}c@{}l|p{4.5cm}|p{4.5cm}|p{4.0cm}|}
\hline
Qt. & & Definition & Description & Variable & Routine \\
\hline
%
%=================================================================
%
\hline $V^{PS}_{ion}(r)$ &&& PS descreened potential & \var{vloc\_at}, \var{vloc}, \var{vltot} & \var{init\_vloc}, \var{setlocal} \\
%
$Q_{ij}(r)$ &$\sim$& $\phi^*_i(r)\phi_j(r)-\tilde{\phi}^*_i(r)\tilde{\phi}_j(r)$ & augmentation functions (pseudized somehow) & \var{qrad}, \var{qgm} & \var{init\_us\_1}, \var{qvan2} \\
%
$\tilde{n}_c(r)$ && & PS core charge density & \var{rho\_core} & \var{set\_rhoc}\\
%
\hline
%
$\tilde{n}(r)$ &$=$& $\sum_n f_n |\tilde{\Psi}_n(r)^2|$ &the
non-augmented smooth charge density&\var{rho}=$\tilde{n}$+$\hat{n}$&\var{sum\_band}\\
%
$\rho_{ij}$ &$=$& $\sum_n f_n
\langle\tilde{\Psi}_n(r)|\beta_i\rangle\langle\beta_j|\tilde{\Psi}_n(r)\rangle$
& augmentation channel occupations & \var{becsum} &\var{sum\_band} \\
%
$\hat{n}(r)$ &$=$& $\sum_{ij}\rho_{ij}Q_{ij}(r)$ & compensation
charge & added to \var{rho} & \var{addusdens} \\
%
$\tilde{v}_{eff}(r)$ &$=$& $V^{PS}_{ion}+v_{Hxc}[\tilde{n}+\hat{n}+\tilde{n}_c]$ & effective potential &\var{vr} & \var{v\_of\_rho} \\
%
%=================================================================
\hline
%=================================================================
%
\hline $Q_{ij}(r)$ &$\sim$& $\phi^*_i(r)\phi_j(r)-\tilde{\phi}^*_i(r)\tilde{\phi}_j(r)$ & augmentation functions (pseudized somehow) & added to \var{ptfunc} & --\\
%
$P_{ij}(r)$ &$=$& $\phi^*_i(r)\phi_j(r)\theta(|r-r_c|)$ & AE atomic wavefunctions & \var{pfunc}, \var{prad} & as \var{init\_us\_1}, \var{qvan2}\\
%
$\tilde{P}_{ij}(r)$ &$=$& $\tilde{\phi}^*_i(r)\tilde{\phi}_j(r)\theta(|r-r_c|)$ & PS atomic & \var{ptfunc}=$\tilde{P}_{ij}$+$Q_{ij}$, \var{ptrad} & as \var{init\_us\_1}, \var{qvan2}\\
%
%=================================================================
%
\hline $I_{(ij),(ij)'}$ &$=$& $\int \frac{P_{ij}(r)P_{(ij)'}(r')}{|r-r'|}dr dr'$ & integrals of products of AE P-functions & \var{prodp}, \var{prod0p} & \var{paw\_prod\_p} \\
$\tilde{I}_{(ij),(ij)'}$ &$=$& $\int \frac{\tilde{P}_{ij}(r)\tilde{P}_{(ij)'}(r')}{|r-r'|} dr dr'$ & integrals of products of PS P-functions & \var{prodpt}, \var{prod0pt} & \var{paw\_prod\_p} \\
%
%=================================================================
%
\hline ${n}^1(r)$ &$=$& $\sum_{ij}\rho_{ij}{P}_{ij}(r)$ & AE 1-c charge & \var{rho1} & as \var{addusdens} \\
%
$\tilde{n}^1(r)$ &$=$& $\sum_{ij}\rho_{ij}\tilde{P}_{ij}(r)$ & PS 1-c charge & \var{rho1t}=$\tilde{n}^1$+$\hat{n}$& as \var{addusdens} \\
%
$\hat{n}(r)$ &$=$& $\sum_{ij}\rho_{ij}Q_{ij}(r)$ & compensation charge & added to \var{rho1t}
& -- \\
%
%=================================================================
%
\hline $n_c(r)$ && & AE core charge density & \var{aerho\_atc}, \var{aerho\_core} & as \var{set\_rhoc}\\
%
$\tilde{n}_c(r)$ && & PS core charge density & \var{psrho\_atc}, \var{psrho\_core} & as \var{set\_rhoc}\\
%
%=================================================================
%
\hline $V^{AE}_{ion}(r)$ 
&$\rightarrow$&$V^{AE}_{ion}(r)\theta(|r-r_c|)$ 
& AE descreened potential &  \var{aevloc\_at}, \var{aevloc}, \var{aevloc\_r}& \var{init\_paw\_vloc}, \var{paw\_grid\_setlocal} \\
%
$V^{PS}_{ion}(r)$  
&$\rightarrow$&$V^{PS}_{ion}(r)\theta(|r-r_c|)$
& PS descreened potential &\var{psvloc\_at}, \var{psvloc}, \var{psvloc\_r}& \var{init\_paw\_vloc}, \var{paw\_grid\_setlocal} \\
%
%=================================================================
%
\hline ${v}_{Hxc}^1(r)$ &$=$& $v_{Hxc}[{n}^1+{n}_c]$ & AE 1-c H+xc
potential & \var{vr1} & \var{v\_xc}, \var{v\_h\_grid} \\
%
$\tilde{v}_{Hxc}^1(r)$ &$=$& $v_{Hxc}[\tilde{n}^1+\hat{n}+\tilde{n}^c]$ & PS 1-c H+xc potential & \var{vr1t} & \var{v\_xc}, \var{v\_h\_grid} \\
%%
%\hline ${v}_{eff}^1(r)$ &$=$& $V^{AE}_{ion}+v_{Hxc}[{n}^1+{n}_c]$ & AE 1-c effective potential & & as \var{v\_of\_rho} \\
%%
%$\tilde{v}_{eff}^1(r)$ &$=$& $V^{PS}_{ion}+v_{Hxc}[\tilde{n}^1+\hat{n}+\tilde{n}^c]$ & PS 1-c effective potential & & as \var{v\_of\_rho} \\
%
%=================================================================
%
\hline $E_1$ &$=$& $E_{xc}[n^1+n_c]$ & AE 1-c XC energy & \var{etxc1} & \var{v\_xc} \\
%
$ $ &+& $E_{H}[n^1]$ & AE 1-c Hartree energy & \var{ehart1} & \var{v\_h\_grid} \\
%
$ $ &$-$& $\int{n^1 v_{Hxc}^{1}}$ & avoid double counting & {\em beware, new definition of \var{etot} by P.U.}& as \var{delta\_e} \\
%
$ $ &$+$& $\int{n^1 (v_{Hxc}^{1}-v_{Hxc}^{1,old})}$ & correction for self-consistency & {\em beware, new definition of \var{etot} by P.U.} & as \var{delta\_escf} \\
%
%=================================================================
%
\hline $\tilde{E}_1$ &$=$& $E_{xc}[\tilde{n}^1+\hat{n}+\tilde{n}_c]$ & PS 1-c XC energy & \var{etxc1t} & \var{v\_xc} \\
%
$ $ &+& $E_{H}[\tilde{n}^1+\hat{n}]$ & PS 1-c Hartree energy & \var{ehart1t} & \var{v\_h\_grid} \\
%
$ $ &$-$& $\int{(\tilde{n}^1+\hat{n}) \tilde{v}_{Hxc}^{1}}$ & avoid double counting & {\em beware, new definition of \var{etot} by P.U.}& as \var{delta\_e} \\
%
$ $ &$+$& $\int{(\tilde{n}^1+\hat{n}) (\tilde{v}_{Hxc}^{1}-\tilde{v}_{Hxc}^{1,old})}$ & correction for self-consistency & {\em beware, new definition of \var{etot} by P.U.}& as \var{delta\_escf} \\
%
%=================================================================
%
\hline $D_{ij}$ &$=$& $\int{\tilde{v}_{eff}Q_{ij}}$ & US update of NL coefficients & \var{deeq} & \var{newd} \\
%
$ $ &$+$& $\langle\phi_i|T|\phi_j\rangle-\langle\tilde{\phi}_i|T|\tilde{\phi}_j\rangle$ & Kinetic energy differences & \var{kdiff}, added to \var{deeq} & read from file \\
%
$ $ &$+$& $\int(V^{AE}_{ion}P_{ij}-V^{PS}_{ion}\tilde{P}_{ij})$ & $\langle\phi_i|V^{AE}_{ion}|\phi_j\rangle-\langle\tilde{\phi}_i|V^{PS}_{ion}|\tilde{\phi}_j\rangle$ & \var{dpaw\_ae}/\var{ps}, $\rightarrow$\var{deeq} & \var{newd\_paw\_grid}\\
%
$ $ &$-$& $\int{V^{PS}_{ion}Q_{ij}}$ & compensate for US update & \var{dpaw\_ae}/\var{ps}, $\rightarrow$\var{deeq} & \var{newd\_paw\_grid}\\
%
$ $ &$+$& $\int(v^1_{Hxc}P_{ij}-\tilde{v}^1_{Hxc}\tilde{P}_{ij})$ & $\langle\phi_i|v^1_{Hxc}|\phi_j\rangle-\langle\tilde{\phi}_i|\tilde{v}^1_{Hxc}|\tilde{\phi}_j\rangle$ & \var{dpaw\_ae}/\var{ps}, $\rightarrow$\var{deeq} & \var{newd\_paw\_grid}\\
%
$ $ &$-$& $\int{\tilde{v}^1_{Hxc}Q_{ij}}$ & compensate for US update & \var{dpaw\_ae}/\var{ps}, $\rightarrow$\var{deeq} & \var{newd\_paw\_grid}\\
%
%\hline $ $ &=& $ $ & & &\\
\hline
\end{tabular}
\end{small}
%\end{center}
\caption{\label{tab:quantities}Quantities needed to perform PAW
calculation on the regular grid}
\end{table*}


Since the solution of the Schr\"odinger equation is the same for US
and PAW, one has just to implement the calculation of the total
energy and the hamiltonian (i.e., of nonlocal part $D_{ij}$).
Forces will be discussed at a later stage.  One also needs for SCF
procedure a starting point for the augmentation channel occupations
$\rho_{ij}$ and an algorithm for mixing them, consistently with the
corresponding charge density.

To evaluate PAW corrections to $E$ and $D_{ij}$ one needs the
quantities reported in Table~\ref{tab:quantities}.  The first group
are quantities defining the US pseudopotential, and the second are
quantites already used for US calculation.  In the following groups
are quantities which should be added for PAW calculation.  Some of
them are repeated from the first two groups to remind the reader that their
evaluation will eventually be performed on radial grids.  It is
understood that all one-center (1-c) integrals have to be performed
individually for all atoms, and then summed.

Notice the spatial cutoff $\theta(|r-r_c|)$ applied to $P$,
$\tilde{P}$, $V^{AE}_{ion}$, and $V^{PS}_{ion}$.  This will not be
strictly needed when integrating on the radial grid, due to
cancellations between AE and PS quantities.  Right now it is necessary
in order to avoid overposition of functions with those of the same
atom in periodically repeated cells.  The use of $\theta$ is purely
symbolic, as a softer cutoff function should be used.  As a
consequence of the cutoff, charges $n^1$, $\tilde{n}^1$ and potentials
$v^1_{xc}$, $\tilde{v}^1_{xc}$ will also be vanishing for $r>r_c$
since LDA/GGA are quasilocal.

\section*{Hartree terms ($n^1$, $\tilde{n}^1$)}

We define the Fourier transform (FT) 
of a function $f(\mathbf{r})$, and the anti-FT, as
\begin{eqnarray}
\label{eq:FTL2R3}
f(\mathbf{q}) &=& \int \mathrm{d}^3 \mathbf{r} \mathrm{e}^{ -\mathrm{i} \mathbf{q} \cdot \mathbf{r}} f(\mathbf{r}),\\
f(\mathbf{r}) &=& \int \frac{\mathrm{d}^3 \mathbf{q}}{(2\pi)^3} \mathrm{e}^{ \mathrm{i} \mathbf{q} \cdot \mathbf{r}} f(\mathbf{q}).
\end{eqnarray}
In \var{pw.x} one adopts the slightly different notation
$f(\mathbf{q})=\frac{1}{\Omega} \int \mathrm{d}^3 \mathbf{r}
\mathrm{e}^ { -\mathrm{i} \mathbf{q} \cdot \mathbf{r}} f(\mathbf{r})$,
$\Omega$ being the volume of the periodically repeated cell.
We will discretize FT with the latter convention, leading to:
\begin{eqnarray}
\label{eq:discFT}f_\mathbf{G}=&\frac{1}{\Omega} \sum_\mathbf{r} \frac{\Omega}{N} \mathrm{e}^ { -\mathrm{i} \mathbf{G} \cdot \mathbf{r}} f_\mathbf{r}& =\frac{1}{N} \sum_\mathbf{r} \mathrm{e}^ { -\mathrm{i} \mathbf{G} \cdot \mathbf{r}} f_\mathbf{r},\\
f_\mathbf{r}=&{\Omega} \sum_\mathbf{r} \frac{\Omega'}{(2\pi)^3N} \mathrm{e}^ { \mathrm{i} \mathbf{G} \cdot \mathbf{r}} f_\mathbf{r}& = \sum_\mathbf{r} \mathrm{e}^ { \mathrm{i} \mathbf{G} \cdot \mathbf{r}} f_\mathbf{G},
\end{eqnarray}
where $\Omega'=\frac{(2\pi)^3N}{\Omega}$ is as usual the volume in
reciprocal space.  Now notice that
\begin{eqnarray}
f_\mathbf{r} &\equiv& f(\mathbf{r}),\\
f_\mathbf{G} &=& \frac{1}{\Omega} f(\mathbf{q}=\mathbf{G}),\\
f_{\mathbf{G}=0} &=& \frac{1}{\Omega} \int \mathrm{d}^3\mathbf{r} f(\mathbf{r})= \frac{f_\text{Tot}}{\Omega}=f_\text{Av}.
\end{eqnarray}

As for Coulomb potential related functions, we remind the following equations:
\begin{eqnarray}
f(\mathbf{r})=\frac{1}{r}&\Rightarrow& f(\mathbf{q})=\frac{4\pi}{q^2}\\
\nabla^2\frac{1}{|\mathbf{r}-\mathbf{r}'|}&=&-4\pi\delta^{(3)}(\mathbf{r}-\mathbf{r}').
\end{eqnarray}
We define ($\frac{1}{2}\times$) Hartree energy density as
\begin{eqnarray}
V_\text{H}(\mathbf{r})&\equiv&\frac{e^2}{4\pi\epsilon_0} \int \mathrm{d}^3\mathbf{r}' \frac{n(\mathbf{r}')}{|\mathbf{r}-\mathbf{r}'|}\nonumber\\
\label{eq:VHcont}&=& \frac{e^2}{4\pi\epsilon_0} \int \frac{\mathrm{d}^3\mathbf{q}}{(2\pi)^3} n(\mathbf{q}) \frac{4\pi}{q^2} \mathrm{e}^{ \mathrm{i} \mathbf{q} \cdot \mathbf{r}},
\end{eqnarray}
where $n(\mathbf{r})$ is the electron {\em number} density (not the
{\em charge} density).  Following this definition, Hartree energy is
simply given as $E_\text{H}=\frac{1}{2} \int \mathrm{d}^3\mathbf{r}
n(\mathbf{r}) V_\text{H}(\mathbf{r})$, with all constants embedded
into $V_\text{H}$.

Now we want to evaluate $V_\text{H}$ for a spatially-bounded charge
distribution, taking advantage of discretized FT, but without
introducing spurious effect due to periodically repeated replicas.
There is a problem though: one cannot straightforwardly
discretize Eq.~(\ref{eq:VHcont}) using Eq.~(\ref{eq:discFT}) because
the Coulomb potential is singular at $q=0$ (the singularity is
not a problem in the continuous formulation because it is canceled by
the volume element).  A possible solution is to exploit linearity in
charge: remove gaussian charges so to cut-off the FT of the density
and cancel singularity via $n(\mathbf{q})$; then the potential
generated by gaussian charges is added analytically in real space.
Let $Q$ be the total charge, $Q=n(\mathbf{q}=0)$.  We substract and
add the charge
\begin{eqnarray}
n_W(\mathbf{q})&=&Q\mathrm{e}^{-\alpha q^2}, \\
n_W(\mathbf{r})&=&Q \frac{1}{(2\pi)^3} \mathrm{e}^{-\frac{r^2}{4\alpha}}\left(\frac{\pi}{\alpha}\right)^{3/2},
\end{eqnarray}
which generates the potential\footnote{To derive $W(\mathbf{r})$, one
takes FT of $W(\mathbf{q})$ and notices that $$\int_0^\infty
\mathrm{d}q \frac{\sin{qr}}{qr}\mathrm{e}^{-\alpha q^2}
=\frac{\pi}{2r}\text{erf}\left(\frac{r}{2\sqrt{\alpha}}\right).$$}
\begin{eqnarray}
W(\mathbf{q})&=&\frac{e^2}{4\pi\epsilon_0}\frac{4\pi}{q^2}Q\mathrm{e}^{-\alpha q^2},\\
W(\mathbf{r})&=&\frac{e^2}{4\pi\epsilon_0}\frac{Q}{r}\text{erf}\left(\frac{r}{2\sqrt{\alpha}}\right).
\end{eqnarray}
Then we split $V$, and take advantage from the fact that
$\lim_{q\rightarrow\mathbf{G}}(V-W)(\mathbf{q})$ is well defined also
for $\mathbf{G}=0$:
\begin{eqnarray}
V(\mathbf{r})&=&W(\mathbf{r})+(V-W)(\mathbf{r})\\
&=&W(\mathbf{r}) + \int \frac{\mathrm{d}^3\mathbf{q}}{(2\pi)^3} \mathrm{e}^{ \mathrm{i} \mathbf{q} \cdot \mathbf{r}}(V-W)(\mathbf{q})\nonumber\\
&=&W(\mathbf{r}) + \sum_\mathbf{G} \mathrm{e}^{\mathrm{i}\mathbf{G}\cdot\mathbf{r}}\frac{1}{\Omega}\lim_{q\rightarrow\mathbf{G}}(V-W)(\mathbf{q})\nonumber\\
%&=&W(\mathbf{r}) + \sum_\mathbf{G\neq0} \mathrm{e}^{\mathrm{i}\mathbf{G}\cdot\mathbf{r}} (V-W)_\mathbf{G} + \frac{1}{\Omega}\lim_{q\rightarrow0}(V-W)(\mathbf{q})
&=&W(\mathbf{r}) + \sum_\mathbf{G} \mathrm{e}^{\mathrm{i}\mathbf{G}\cdot\mathbf{r}} \bar{V}_\mathbf{G},
\end{eqnarray}
where we introduced $\bar{V}_\mathbf{G}$, defined by
\begin{eqnarray}
\bar{V}_{\mathbf{G}\neq0}&=&\frac{e^2}{4\pi\epsilon_0}\frac{4\pi}{G^2} \left(n_\mathbf{G}-\frac{Q\mathrm{e}^{-\alpha G^2}}{\Omega}\right),\\
\label{eq:VbG=0}\bar{V}_{\mathbf{G}=0}&=&\frac{1}{\Omega}\lim_{q\rightarrow0}(V-W)(\mathbf{q}).
\end{eqnarray}
(Notice that the distance $r$ is ill-defined in PBC.)  Finally, we
have to evaluate $\lim_{q\rightarrow0}(V-W)(\mathbf{q})$.  Let us
first expand $n(\mathbf{q})$ in powers of $q$:
\begin{eqnarray}
\nabla_\mu n(\mathbf{q}) = -\mathrm{i} \int \mathrm{d}^3\mathbf{r} r_\mu \mathrm{e}^{\mathrm{i}\mathbf{q}\cdot\mathbf{r}} n(\mathbf{r});
\end{eqnarray}
(this term vanishes for $q=0$ {\bf if} $n(\mathbf{r})=n(-\mathbf{r})$)
\begin{eqnarray}
\nabla_\nu \nabla_\mu n(\mathbf{q}) &=& - \int \mathrm{d}^3\mathbf{r} r_\nu r_\mu \mathrm{e}^{\mathrm{i}\mathbf{q}\cdot\mathbf{r}} n(\mathbf{r})\\
&=&- \int \mathrm{d}^3\mathbf{r} r_\nu r_\mu n(\mathbf{r})\text{~if $q=0$}.
\end{eqnarray}
Now define
\begin{eqnarray}
I(\mathbf{q})&=&\sum_{\mu\nu} q_\mu q_\nu \nabla_\nu \nabla_\mu
n(\mathbf{q})_{\mathbf{q}=0} \\&=& -\int \mathrm{d}^3\mathbf{r}
(\mathbf{q}\cdot\mathbf{r})^2 n(\mathbf{r}).
\end{eqnarray}
The integral of $I(\mathbf{q})$ is
\begin{eqnarray}
\int \mathrm{d}^3\mathbf{q}
I(\mathbf{q})&=&-\int\mathrm{d}^3\mathbf{r}n(\mathbf{r})
\int\mathrm{d}^3\mathbf{q} (\mathbf{q}\cdot\mathbf{r})^2 \nonumber\\&=&
-\frac{1}{3}\int\mathrm{d}^3\mathbf{r}n(\mathbf{r})
\int\mathrm{d}^3\mathbf{q}q^2\nonumber\\
&=&
\int\left(-\frac{1}{3}{\cal{M}}q^2\right)\mathrm{d}^3\mathbf{q},\\
{\cal{M}}&\equiv&\int\mathrm{d}^3\mathbf{r}n(\mathbf{r})r^2,\\
&=&\sum_{ij}\rho_{ij}{\cal{M}}_{ij}.
\end{eqnarray}
[The coefficients
${\cal{M}}_{ij}=\int{d{\mathbf{r}}r^2P_{ij}({\mathbf{r}})}
=\delta_{l_il_j}\delta_{m_im_j}\int{4{\pi}r^2drr^2P_{ij}(r)}$ are
easily evaluated on the radial mesh once for all at the beginning of
the calculation.]
%
So, upon integration, $I(\mathbf{q})$ is equivalent to
$-\frac{1}{3}{\cal{M}}q^2$.  If we insert this latter quantity into the
expansion of $n(\mathbf{q})$, we obtain for ${q\rightarrow0}$
\begin{eqnarray}
(V-W)(\mathbf{q})&\approx&
\frac{e^2}{4\pi\epsilon_0}\frac{4\pi}{q^2} \left[ Q - \frac{1}{6}{\cal{M}}q^2 - Q +Q\alpha q^2\right]\nonumber\\
&=&\frac{e^2}{4\pi\epsilon_0} \left(4\pi\alpha Q - \frac{2\pi}{3}{\cal{M}}\right),
\end{eqnarray}
so that
\begin{equation}
\bar{V}_{\mathbf{G}=0}=\frac{e^2}{4\pi\epsilon_0} \left(4\pi\alpha Q - \frac{2\pi}{3}{\cal{M}}\right)\frac{1}{\Omega}.
\end{equation}

Let us now determine the Hartree energy corresponding to
$n(\mathbf{r})$,
\begin{eqnarray}
E_\text{H} &=& \frac{1}{2} \frac{e^2}{4\pi\epsilon_0} \int\mathrm{d}^3\mathbf{r}\int\mathrm{d}^3\mathbf{r}' \frac{n(\mathbf{r})n(\mathbf{r})'}{|\mathbf{r}-\mathbf{r}'|}\nonumber\\
&=& \frac{1}{2} \frac{e^2}{4\pi\epsilon_0} \int\mathrm{d}^3\mathbf{q} \frac{4\pi n(\mathbf{q})n(\mathbf{q})}{q^2}.
\end{eqnarray}
Let us introduce
\begin{equation}
F(\mathbf{q})=\frac{1}{2}\frac{e^2}{4\pi\epsilon_0}\frac{4\pi
n(\mathbf{q})n(\mathbf{q})}{q^2},
\end{equation}
so that Hartree energy can be computed as anti-FT of $F(\mathbf{q})$:
\begin{equation}
E_\text{H}=F(\mathbf{r}=0).
\end{equation}
We substract and add again $n_W$.  Define
\begin{eqnarray}
n(\mathbf{q})&=&a(\mathbf{q})+b(\mathbf{q}),\\
a(\mathbf{q})&=&n(\mathbf{q})-n_W(\mathbf{q}),\\
b(\mathbf{q})&=&n_W(\mathbf{q}).
\end{eqnarray}
Then we separate $F(\mathbf{q})$ into three terms:
\begin{eqnarray}
F(\mathbf{q})=F^{aa}(\mathbf{q})+2F^{ab}(\mathbf{q})+F^{bb}(\mathbf{q}).
\end{eqnarray}
Since $F^{aa}(\mathbf{q})\rightarrow0$ as $q\rightarrow0$ we can
directly discretize the anti-FT of $F^{aa}(\mathbf{q})$ to the one of
$F^{aa}_\mathbf{G}$ restricting the summation to $\mathbf{G}\neq0$:
\begin{eqnarray}
F^{aa}(\mathbf{r}=0)&=& \sum_{\mathbf{G}\neq0}F^{aa}_\mathbf{G} =
\frac{1}{\Omega}\sum_{\mathbf{G}\neq0}F^{aa}(\mathbf{q}=\mathbf{G})\nonumber\\
&=&\frac{1}{2}\frac{e^2}{4\pi\epsilon_0}\sum_{\mathbf{G}\neq0}\frac{4\pi}{G^2}a_\mathbf{G}^2\,\,\,\Omega,\\
a_\mathbf{G}&=&n_\mathbf{G}-\frac{Q\mathrm{e}^{-\alpha G^2}}{\Omega}
\end{eqnarray}
%
We do the same for $F^{ab}$, but now the $\mathbf{G}=0$ contribution,
$\frac{1}{\Omega}\lim_{\mathbf{q}\rightarrow0}F^{ab}(\mathbf{q})$, is
different from zero and has to be added as in Eq.~(\ref{eq:VbG=0}):
\begin{eqnarray}
F^{ab}_{\mathbf{G}=0}&=&\frac{1}{\Omega}\lim_{\mathbf{q}\rightarrow0}F^{ab}(\mathbf{q})\nonumber\\
&=& \frac{1}{2\Omega}\lim_{\mathbf{q}\rightarrow0}
\frac{e^2}{4\pi\epsilon_0} \frac{4\pi}{q^2} [n(\mathbf{q})-n_W(\mathbf{q})]n_W(\mathbf{q})\nonumber\\
&=& \frac{1}{2\Omega}Q\lim_{\mathbf{q}\rightarrow0}(V-W)(\mathbf{q})\nonumber\\
&=& \frac{1}{2\Omega}Q\frac{e^2}{4\pi\epsilon_0} \left(4\pi\alpha Q - \frac{2\pi}{3}{\cal{M}}\right).
\end{eqnarray}
The $\mathbf{G}\neq0$ contribution to $F^{ab}$ sums with the one of
$F^{aa}$; the total corresponds to squared density
$n_\mathbf{G}^2-\frac{Q^2\mathrm{e}^{-2\alpha G^2}}{\Omega^2}$.\\
Finally, the contribution from $F^{bb}$ is analytical:
\begin{eqnarray}
F^{bb}(\mathbf{r}=0)&=&\frac{1}{2} \frac{e^2}{4\pi\epsilon_0}
\int\frac{\mathrm{d}^3 \mathbf{q}}{(2\pi)^3} \frac{4\pi}{q^2} Q^2
\mathrm{e}^{-2\alpha q^2} \nonumber\\&=& \frac{1}{2} \frac{e^2}{4\pi\epsilon_0}
\frac{Q^2}{\sqrt{2\pi\alpha}}.
\end{eqnarray}
Collecting all terms, we have for $E_\text{H}^1$ and
$\tilde{E}_\text{H}^1$
\begin{eqnarray}
E_\text{H}&=&
\sum_{\mathbf{G}\neq0}\frac{1}{2}\frac{e^2}{4\pi\epsilon_0}\frac{4\pi}{G^2}
\left(n_\mathbf{G}^2- \frac{Q^2\mathrm{e}^{-2\alpha
G^2}}{\Omega^2}\right)\nonumber\\
&&+\frac{1}{2\Omega}Q\frac{e^2}{4\pi\epsilon_0} \left(4\pi\alpha Q - \frac{2\pi}{3}{\cal{M}}\right) \nonumber\\
&&+ \frac{1}{2} \frac{e^2}{4\pi\epsilon_0}
\frac{Q^2}{\sqrt{2\pi\alpha}}.
\end{eqnarray}


\section*{Hartree terms ($\tilde{n}$)}


{\bf From this point it is not completely correct, to be changed:}
{\small\sl%
Should Hartree potentials $v^1_{H}$, $\tilde{v}^1_{H}$ be constructed
in order to coincide outside $r_c$, and $\tilde{v}^1_{H}$ so to
coincide with $\tilde{v}_{H}$ inside?  Let us shift the three
potentials by a constant (which is left free when solving Poisson's
equation), independently:
\begin{equation}
\tilde{v}_{H}=\tilde{V}_{H}+\tilde{c},\ 
\tilde{v}^1_{H}=\tilde{V}^1_{H}+\tilde{c}^1,\ 
{v}^1_{H}={V}^1_{H}+{c}^1.
\end{equation}
Remember that we have to define $v^1_{H}$ and $\tilde{v}^1_{H}$ for
each atom in the cell, thus we have $N_{at}$ couples $\{\tilde{c}^1_{\mathbf{R}},{c}^1_{\mathbf{R}}\}$.

For the potential to be treated with plane waves, it is easily seen
that
\begin{eqnarray}
\frac{\partial H}{\partial\tilde{c}}&=&1+
\sum_{ij}|\beta_i\rangle\langle\beta_j|
\frac{\partial}{\partial\tilde{c}}\int{Q_{ij}\tilde{v}_{eff}dr}
\nonumber\\&=&1+\sum_{ij}|\beta_i\rangle\langle\beta_j|q_{ij}=S.
\end{eqnarray}
For $\tilde{c}^1_{\mathbf{R}}$ and $c^1_{\mathbf{R}}$ we have to
take the derivative of the NL coefficients $D_{ij}$ corresponding to
atom in ${\mathbf{R}}$ only:
\begin{eqnarray}
\frac{\partial H}{\partial\tilde{c}^1_{\mathbf{R}}}&=&-
\sum_{ij\in{\mathbf{R}}}|\beta_i\rangle\langle\tilde{\phi}_i|\tilde{\phi}_j\rangle\langle\beta_j|
-\sum_{ij\in{\mathbf{R}}}|\beta_i\rangle\langle\beta_j|q_{ij}
\nonumber\\&=&
-1_{\mathbf{R}}-\sum_{ij\in{\mathbf{R}}}|\beta_i\rangle\langle\beta_j|q_{ij}=-S_{\mathbf{R}},
\end{eqnarray}
where we have used the property that for a complete set of projectors
$\sum_{i\in{\mathbf{R}}}|\tilde{\phi}_i\rangle\langle\beta_i|=1_{\mathbf{R}}$,
the identity on the augmentation region surrounding ${\mathbf{R}}$.
Finally,
\begin{eqnarray}
\frac{\partial H}{\partial{c}^1_{\mathbf{R}}}&=&
\sum_{ij\in{\mathbf{R}}}|\beta_i\rangle\langle{\phi}_i|{\phi}_j\rangle\langle\beta_j|
\nonumber\\&=&
\sum_{ij}|\beta_i\rangle\langle\tilde{\phi}_i|\tilde{\phi}_j\rangle\langle\beta_j|+\sum_{ij\in{\mathbf{R}}}|\beta_i\rangle\langle\beta_j|q_{ij}
\nonumber\\&=&
1_{\mathbf{R}}+\sum_{ij\in{\mathbf{R}}}|\beta_i\rangle\langle\beta_j|q_{ij}=S_{\mathbf{R}}.
\end{eqnarray}
Having introduced $\tilde{c}$, $\tilde{c}^1$ and $c^1$, the
generalized eigenvalue problem $H\psi=\epsilon{S}\psi$ is transformed
into $H'\psi'=\epsilon'{S}\psi'$, with hamiltonian
$H'=H+\tilde{c}S+\sum_{\mathbf{R}}(c^1-\tilde{c}^1)S_{\mathbf{R}}$,
i.e. into
\begin{equation}\label{eq:newH}
H\psi'=[\epsilon'-\tilde{c}]S\psi'+\sum_{\mathbf{R}}(c^1_{\mathbf{R}}-\tilde{c}^1_{\mathbf{R}})S_{\mathbf{R}}\psi',
\end{equation}
from which we see that upon changes in $\tilde{c}$ {\em eigenfunctions
are the same while eigenvalues are shifted to energies
$\epsilon'=\epsilon+\tilde{c}$}.  Changes in $c^1_{\mathbf{R}}$ and
$\tilde{c}^1_{\mathbf{R}}$ can be more dramatic, changing also the
shape of the hamiltonian, unless one fixes
$c^1_{\mathbf{R}}=\tilde{c}^1_{\mathbf{R}}$.

We shall then consider the effect of $\tilde{c}$ constant on the
electronic energy.  From what we have seen above, the sum of the
eigenvalues trivially changes as
%$\sum{(\tilde{c}+c^1-\tilde{c}^1)}=(\tilde{c}+c^1-\tilde{c}^1)N_{el}$.
$\sum{\tilde{c}}=\tilde{c}N_{el}$.
%
Then consider the plane-wave (US) part: $\tilde{E}_H$ does not change,
since it is built directly from the charge density and not the
potential; the DC correction is affected as
$\int{(\tilde{n}+\hat{n})\tilde{c}}=\tilde{c}N_{el}$, thus
compensating the same term contained in the sum of eigenvalues.
%
%We easily get analogous results for the dependence of $E^1$ and
%$\tilde{E}^1$ upon $c^1$ and $\tilde{c}^1$, respectively.
%
So also {\em electronic energy is invariant}, as it should be.

Let's go back to $c^1_{\mathbf{R}}$ and $\tilde{c}^1_{\mathbf{R}}$.
Eq.~(\ref{eq:newH}) indicates that it is necessary to have
$v^1_{H}=\tilde{v}^1_{H}$ outside $r_c$, possibly with the addition of
a common constant to both terms.  When performing the integration on a
radial grid, this constant is perhaps most conveniently chosen so that
$v(\infty)=0$.  This condition cannot be set for calculations on the
periodically repeated finite cell of plane-wave code.  There,
Poisson's equation is solved in $G$-space by \var{v\_of\_rho}, where
one puts $v_G=n_G/G^2$ and $v_{G=0}=0$.  Notice that
$c\propto{}v_{G=0}\times[{\textrm{volume}}]$.  One way of proceeding
could be to Fourier transform the potentials back to real space and to
shift $v^1$ so to minimize some sort of ``distance'' from
$\tilde{v}^1$ {\em outside} the augmentation region, e.g., by
requiring
\begin{equation}
\int_{^c\Omega_{\mathbf{R}}}\tilde{v}^1({\mathbf{r}})d{\mathbf{r}}=\int_{^c\Omega_{\mathbf{R}}}{v}^1({\mathbf{r}})d{\mathbf{r}}.
\end{equation}
For evaluating these integrals, one has to build a routine to identify
all points in the regular grid which are outside the augmentation
sphere.


The problem connected to the arbitrariness of $c^1_{\mathbf{R}}$ and
$\tilde{c}^1_{\mathbf{R}}$ is due to the fact that, rigorously, the
Hartree potential (for the atomic densities) should be
\begin{equation}
v_H({\mathbf{r}})=\int{d{\mathbf{r}}'\frac{n({\mathbf{r}}')}{|{\mathbf{r}}-{\mathbf{r}}'|}},
\end{equation}
which also fixes the value of the constants (that is the average value
of the potential, and hence the $v_{G=0}$).  So a better way to
proceed is to construct $v_{G=0}$ in order to give the proper average.
This can be done by observing that, under the hypothesis i) $n$ is
globally neutral ii) $r^2n({\mathbf{r}})\rightarrow0$ for
$r\rightarrow\infty$:
\begin{equation}
\int{d{\mathbf{r}}v_H({\mathbf{r}})}=-\frac{2\pi}{3}\int{d{\mathbf{r}}r^2n({\mathbf{r}})}\equiv-\frac{2\pi}{3}{\cal{M}},
\end{equation}
(this result is based on Gauss' theorem; alternatively, it can be
derived by using Fourier Transforms and by imposing that the summation
over discrete $G$-vectors is equivalent to integration over continous
ones).  Hypothesis (ii) is verified by considering into $n$ also the
positive charge (for a neutral atom), and this does not contribute to
${\cal{M}}$---the additional term is centered in the origin.  If the
atom is not neutral, one can force neutrality by ficticiously changing
the nuclear charge: since the deviation from neutrality is the same
for $n^1$ and $\tilde{n}^1$, this should change the average potential
by the same amount (and remember $H$ depends on the {\em difference}
of $c^1$ and $\tilde{c}^1$, not their absolute values).

%To be applied to the case of $n^1$ and $\tilde{n}^1+\hat{n}$, remember
%that we are cutoffing these functions to make them decrease well
%within the unit cell, so the monopole of the electric field (for
%neutral atom) should be zero and the integral of the potential
%converged with respect to size.  (Multipolar contributions outside the
%compensation cell are also possible, but should be the same for $n^1$
%and $\tilde{n}^1+\hat{n}$ by construction of $\hat{n}$.)

Finally, the definition of the Fourier Transform used by \var{pw.x} is
\begin{equation}
f_{\mathbf{G}}=\frac{1}{\Omega}\int{d{\mathbf{r}} f({\mathbf{r}})
\exp(-i{\mathbf{G}}\cdot{\mathbf{r}})},
\end{equation}
so we should simply set (analogous for $\tilde{v}^1_H$)
\begin{eqnarray}
v^1_{H,{\mathbf{G}}=0}&=&\frac{1}{\Omega}\int{d{\mathbf{r}}v^1_H({\mathbf{r}})}=-\frac{1}{\Omega}\frac{2\pi}{3}{\cal{M}}
\nonumber\\&=&-\frac{2\pi}{3\Omega}\sum_{ij}\rho_{ij}{\cal{M}}^1_{ij},
\end{eqnarray}
where the coefficients
${\cal{M}}^1_{ij}=\int{d{\mathbf{r}}r^2P_{ij}({\mathbf{r}})}
=\delta_{l_il_j}\delta_{m_im_j}\int{4{\pi}r^2drr^2P_{ij}(r)}$ are
easily evaluated on the radial mesh once for all at the beginning of
the calculation.
}


%{\em This have to be taken
%care of when solving Poisson's equation.}  \var{pw.x} code is doing
%this for $n=\tilde{n}+\hat{n}$ in $G$-space (\var{v\_of\_rho}),
%putting $V_G=n_G/G^2$ and $V_{G=0}=0$.  If one does the same for $n^1$
%and $\tilde{n}^1+\hat{n}$ he obtains $v_H^1$ and $\tilde{v}_H^1$
%having the same average value (0) on the full crystal cell.  {\em Does
%this also mean that they coincide outside compensation region?}

\section*{Initialization}

To start the calculation, we need a guess for density
$\tilde{n}$=\var{rho} and augmentation channel occupations
$\rho_{ij}$=\var{becsum}.  The following is the structure of
\var{init\_run}, with ``\var{*}'' indicating added routines for PAW:
\begin{verbatim}
  setup()
  allocate_fft()
  ggen()
  summary()
  allocate_nlpot()
 *allocate_paw_internals()
  allocate_locpot()
  allocate_wfc()
  openfil()
  hinit0()
    local and NL PS, core charge
   *init_paw_vloc
   *init_prad()
   *paw_grid_setlocal
   *set_paw_rhoc()
   *paw_prod_p()
  potinit()
    rho = sum rho_at  ->  vr
   *atomic_becsum()
   *compute_onecenter_charges()
   *compute_onecenter_potentials()
  newd()
   *newd_paw_grid()
  wfcinit()
    solve H in the basis of
    atomic wfcs
\end{verbatim}
Tentative idea: use as \var{becsum} at first iteration the atomic one,
hopefully corresponding to the atomic density (new routine
\var{atomic\_becsum}).  Notice however that this is not fully consistent
to starting wavefunctions, which should be used only as guess for
diagonalization, not for self-consistency.

\section*{SCF loop}

The only difference with US case is: we have to mix also occupations
of augmentation channels, because they affect the new hamiltonian
through one-center charges. This has been implemented in 
mix\_rho.f90. Simple Thomas-Fermi screening and local-density-dependent 
TF screening seem to work without modification.


\end{document}
